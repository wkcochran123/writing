\documentclass[12pt]{article}
\usepackage{amsthm,amssymb,amsmath,mathtools,setspace}
\onehalfspacing

% --- Module Dependencies (Must align with main P.tex definitions) ---
\newtheorem{theorem}{Theorem}
\newtheorem{definition}{Definition}
\newtheorem{proposition}{Proposition}

% Shared G2S/P.tex Symbols
\newcommand{\U}{\mathbf{U}}        
\newcommand{\V}{\mathbf{V}}        
\newcommand{\DTwo}{\mathbf{U}''}   
\newcommand{\DFour}{\mathbf{U}^{(4)}} 
\newcommand{\Acal}{\mathcal{A}}     
\newcommand{\Bform}{\mathsf{B}}     
\newcommand{\Hspace}{\mathcal{H}}   
\newcommand{\Domain}{\Omega}

% G3S Discrete Symbols
\newcommand{\Uh}{\mathbf{U}_{\mathbf{h}}} 
\newcommand{\Vh}{\mathbf{V}_{\mathbf{h}}} 
\newcommand{\Acalh}{\mathcal{A}_{\!\mathbf{h}}} 
\newcommand{\Bformh}{\mathsf{B}_{\!\mathbf{h}}} 
\newcommand{\Sh}{\mathcal{S}_{\!\mathbf{h}}}  
\newcommand{\mesh}{\mathsf{h}}      
\newcommand{\DeltaTwoh}{\Delta_{\!\mathbf{h}}^{(2)}} 
\newcommand{\DeltaFourh}{\Delta_{\!\mathbf{h}}^{(4)}} 
\newcommand{\norm}[1]{\left\lVert #1 \right\rVert} 

% --- Optional Rosetta Layer Input ---
\IfFileExists{V_macros.tex}{% V_macros.tex — ultra-minimal Rosetta (Gemini-safe)
% Only simple \providecommand definitions. No packages, no \makeatletter.

% --- Core symbols used across G1S/G2S/G3S/P/V_article ---
\providecommand{\U}{\mathbf{U}}         % Admissible history / universe field
\providecommand{\V}{\mathbf{V}}         % Test / variation (generic handle)
\providecommand{\ph}{\varphi}           % Test function (lowercase alias)
\providecommand{\Test}{\mathcal{V}}     % Test space
\providecommand{\Bform}{\mathsf{B}}     % Bilinear/reciprocity pairing (weak form)
\providecommand{\grad}{\nabla}          % Gradient / abstract differential operator
\providecommand{\C}{\mathcal{C}}        % Causal class / partition carrier

% --- Entropy & closures (title-safe via \ensuremath) ---
\providecommand{\EntropySymbol}{\mathrm{S}}
% \Entropy[sub] -> S_sub (subscript optional), safe in text/headings
\providecommand{\Entropy}[1][]{\ensuremath{\EntropySymbol_{\!#1}}}
% Fourth derivative shorthand (1D strong closure)
\providecommand{\DFour}{\U^{(4)}}

% --- Named statement handles (contracts), safe in text/headings ---
\providecommand{\GOneStatement}{\ensuremath{\Delta\,\Entropy \ge 0}}
\providecommand{\GTwoStatement}{\ensuremath{\DFour = 0}}
\providecommand{\GTwoWeakForm}{\ensuremath{\Bform(\U,\ph)=0\ \forall\,\ph\in\Test}}
\providecommand{\GThreeStatement}{\ensuremath{\text{Discrete–continuum reciprocity via }\Bform}}

% --- Noether bridge (index-free API primitives) ---
\providecommand{\Xi}{\boldsymbol{\xi}}                                      % symmetry generator (abstract)
\providecommand{\NoetherTensor}[3]{\mathsf{N}\!\left[#1,#2;#3\right]}       % N[L,U;Xi]
\providecommand{\StressEnergy}{\mathbf{T}}                                   % stress–energy (abstract)
\providecommand{\Current}{\mathbf{J}}                                        % Noether current (abstract)
\providecommand{\Div}[1]{\ensuremath{\grad\!\cdot\! #1}}                     % divergence via ∇· (uses \grad)
}{}

\newenvironment{discreteobligation}[1][]{%
  \par\medskip\noindent\textbf{G3 Proof Obligation.} #1\;\emph{(Provide discrete stability and convergence.)}%
  \begin{enumerate}[leftmargin=2.2em,label=\alph*.]
}{%
  \end{enumerate}\medskip
}

\title{\textbf{Module 3: Discrete Kinematics and Consistency} \\
\large Validating G3: The Discrete Spline Energy and Convergence}
\author{G3S — Module Contract Fulfillment (V.tex Compliant)}
\date{\today}

\begin{document}
\maketitle

\section{Input Node: The Discrete Energy Functional}
The minimized continuum action $\Acal[\U]$ is the smooth limit of a discrete action $\Acalh[\Uh]$[cite: 1362].

\begin{definition}[Discrete Curvature Action $\Acalh$]
Let $\Uh$ be the vector of admissible history values on a uniform grid with spacing $\mesh$. The discrete energy functional (informational bending energy) is the Riemannian sum approximation of $\Acal[\U]$[cite: 1363, 1364]:
\[
\Acalh[\Uh] = \frac{1}{2} \sum_{i=1}^{N} \left( \frac{\DeltaTwoh \U_i}{\mesh^2} \right)^2 \mesh.
\]
Here, $\DeltaTwoh \U_i = \U_{i+1} - 2\U_i + \U_{i-1}$ is the centered second-order finite difference operator[cite: 1365].
\end{definition}

\begin{definition}[Discrete Spline Space $\Sh$]
The finite-dimensional space $\Sh$ is the space of unique piecewise cubic polynomials $\U_{\mathbf{h}}$ that interpolate the fixed event nodes and maintain $C^2(\Domain)$ smoothness across knots, minimizing $\Acal[\U]$[cite: 1366, 1367, 1368].
\end{definition}

\section{Theorem: The Kinematic Closure Chain (G3.Chain)}

The stationary point of the discrete action, $\delta \Acalh[\Uh] = 0$, satisfies the discrete Euler-Lagrange equation $\DeltaFourh \Uh = 0$[cite: 1370].

\begin{theorem}[Discrete Kinematic Closure and Convergence]
The discrete Euler-Lagrange operator converges to the continuous solution:
\[
\DeltaFourh \Uh = 0 \quad \implies \quad \lim_{\mesh\to 0} \frac{\DeltaFourh \Uh}{\mesh^2} = \DFour.
\]
This convergence formally links the discrete Event Selection to the continuum minimal curvature[cite: 1371, 1372].
\end{theorem}

\begin{proof}[\textbf{G3 Proof Obligation Fulfillment (Convergence Chain)}]
\begin{discreteobligation}
  \item \textbf{Discrete Weak Form $\Bformh$:} Stationarity $\delta \Acalh[\Uh] = 0$ is equivalent to finding $\Uh^\star \in \Sh$ such that $\Bformh(\Uh^\star, \Vh) = 0$ for all admissible variations $\Vh \in \Sh$[cite: 1375].
  \item \textbf{Consistency (Truncation Error):} The truncation error $T_{\mathbf{h}}(\U^\star) = \mathcal{O}(\mesh^2)$, ensuring the discrete problem accurately reflects the continuum limit[cite: 1377, 1378].
  \item \textbf{Stability (Discrete Coercivity):} The discrete bilinear form $\Bformh$ is coercive, satisfying the discrete Poincaré inequality: $\Bformh(\Uh, \Uh) \ge C_{\mathbf{h}} \norm{\Uh}^2_{l^2}$[cite: 1379, 1380].
  \item \textbf{Convergence (G3.Chain Closure):} Consistency and stability imply that the discrete solution converges to the continuous one in the energy norm: $\norm{\Uh^\star - \U^\star}_{H^2} \to 0$ as $\mesh\to 0$, fulfilling the G3.Chain contract[cite: 1381, 1382].
\end{discreteobligation}
\end{proof}

\end{document}