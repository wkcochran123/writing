\documentclass[12pt]{article}
\usepackage{amsthm,amssymb,amsmath,mathtools,setspace}
\onehalfspacing

% --- Module Dependencies (Must align with main P.tex definitions) ---
\newtheorem{axiom}{Axiom}
\newtheorem{equivalence}{Equivalence}
\newcommand{\U}{\mathbf{U}}        % Universe tensor
\newcommand{\Ek}{\mathbf{E}_k}      % Event tensor

\title{\textbf{Module 1: Foundational Duality and Logical Continuity} \\
\large Validating G5: The Reciprocity Law and the Continuity Boundary}
\author{Independent Logical Section: Validating G5}
\date{\today}

\begin{document}
\maketitle

\section{Input Node: The Defined Universe Tensor}

This module begins with the foundation established in Chapter 1: the universe is a sequence of discrete, distinguishable events, aggregated into the ordered cumulative record $\mathbf{U}_n = \sum_{k=1}^n \mathbf{E}_k$.

\section{The Reciprocity Law: Duality of Measurement and Variation}

The system of finite events must define an internal coherence. This coherence arises from the necessary duality between an observer's two core actions: making a distinction and observing a change.

\begin{equivalence}[The Reciprocity Law]
Every physically admissible variation ($\delta \U$) corresponds to a measurable distinction ($\Phi$), and vice versa. Measurement and variation are exact inverse operations on the space of distinguishable events.
\end{equivalence}

\section{The Axiom of Event Selection (Validating G5)}

The Reciprocity Law holds locally, but requires a non-trivial condition to extend globally. To prevent pathological extensions (overcounting or contradictions), we must impose a necessary logical constraint on the causal set $\mathcal{C}$.

\begin{axiom}[Axiom of Event Selection (MA-like)]
Every countable family of admissible local causal choices admits a globally consistent extension.
\end{axiom}

\textbf{Interpretation as the Continuity Boundary (G5):} This axiom is the logical guarantor of non-contradiction. It is non-constructive, asserting the \emph{existence} of a single, coherent history ($\mathbf{G5}$). This existence is the **minimal regularity condition** required to transition from the countable, discrete algebra of Chapter 1 to the continuous calculus required in Chapter 3. If this axiom failed, the analytic closure of $\mathbf{U}^{(4)}=0$ would not be logically guaranteed.

\section{Output Node: Input to Kinematic Closure}

The **Axiom of Event Selection** is the structural input for the next module. It ensures the space of possible field configurations is restricted to the subset that can be continuously represented.

\begin{center}
\textit{Conclusion for Module 1: The condition for smooth motion is logically guaranteed to exist.}
\end{center}

\end{document}
```
eof

## How to Glue `G3S.tex` to `P.tex`

This module is designed to replace the contents of **Chapter 2** in your main `P.tex` document.

### 1. Identify the Splice Point

The module should be inserted immediately after **Chapter 1** (`\chapter{The Calculus of Measurement: Defining Order}`) and immediately before **Chapter 3** (`\chapter{Smoothness and Kinematic Closure}`).

### 2. Prepare the Core File

Extract the content of `G3S.tex` (everything between `\begin{document}` and `\end{document}`) into a separate file, for example, `G3S_core.tex`.

### 3. Inserting the Module (Chapter 2)

In `P.tex`, you will replace the existing content of **Chapter 2** with the core content file, ensuring the chapter and section headers are correctly positioned for the `book` class:

```latex
...
\chapter{The Calculus of Measurement: Defining Order}
\label{chap:measurement}
... (End of Chapter 1)
% -----------------------------------------------------------------
% START MODULE 1 / CHAPTER 2: Reciprocity and Continuity (Validating G5)
% -----------------------------------------------------------------
\chapter{The Reciprocity Law: Duality of Observation and Change}
\label{chap:reciprocity}
\section*{Core Idea: Observation and Change are Mirror Operations (Validating G5)}

% --- Begin G3S.tex Content ---
\input{G3S_core.tex}
% --- End G3S.tex Content ---

\chapter{Smoothness and Kinematic Closure}
\label{chap:smoothness}
...
