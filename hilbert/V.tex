% V.tex — Verification Tensor: Module Contract and Harness for G1S
% This file is designed to improve G1S.tex while leaving P_clean.tex unchanged.
% All shared symbols are guarded with \providecommand to avoid overriding globals.

\documentclass[12pt]{article}

% ===== Packages (minimal, non-invasive) =====
\usepackage[T1]{fontenc}
\usepackage[utf8]{inputenc}
\usepackage{lmodern}
\usepackage{geometry}
\geometry{margin=1in}
\usepackage{amsmath,amssymb,amsthm}
\usepackage{mathtools}
\usepackage{enumitem}
\usepackage{microtype}
\usepackage{hyperref}
\hypersetup{colorlinks=true,linkcolor=blue,citecolor=blue,urlcolor=blue}

% ===== Theorem-like environments (non-invasive) =====
\newtheorem{theorem}{Theorem}[section]
\newtheorem{proposition}[theorem]{Proposition}
\newtheorem{lemma}[theorem]{Lemma}
\newtheorem{corollary}[theorem]{Corollary}
\theoremstyle{definition}
\newtheorem{definition}[theorem]{Definition}
\newtheorem{invariant}[theorem]{Invariant}
\theoremstyle{remark}
\newtheorem{remark}[theorem]{Remark}

% ===== Shared Notation (guarded; safe if P_clean already defines these) =====
% Causal set / order
\providecommand{\C}{\mathcal{C}}
\providecommand{\E}{\mathcal{E}}
\providecommand{\Vset}{\mathcal{V}}
\providecommand{\preceq}{\leq}

% Universe / tensors / operators
\providecommand{\U}{\mathbf{U}}
\providecommand{\T}{\mathbf{T}}
\providecommand{\Ek}{\mathbf{E}_k}
\providecommand{\grad}{\nabla}

% State/sample space and counting
\providecommand{\OmegaSet}{\Omega}
\DeclareMathOperator{\Dist}{Dist}

% Entropy notation (avoid \S which is section symbol)
\providecommand{\Entropy}{\mathrm{S}}
\providecommand{\kB}{k_{\mathrm{B}}}

% Monotonicity shorthand and visual tag
\providecommand{\Monotone}{\Delta\,\Entropy \ge 0}
\providecommand{\FixedPoint}{\mathsf{FP}}

% ===== Gold Check harness (generic) =====
\newcounter{goldcheck}
\newenvironment{goldcheck}[1][]{%
  \refstepcounter{goldcheck}%
  \par\medskip\noindent\textbf{Gold Check G\thegoldcheck. #1}\par\smallskip\noindent
  \begin{enumerate}[leftmargin=2.2em,label=\arabic*.]
}{%
  \end{enumerate}\medskip
}

% Pass/Fail glyphs
\newcommand{\gpass}{\textbf{[PASS]}}
\newcommand{\gfail}{\textbf{[FAIL]}}

% Specialization for G1 (ΔS ≥ 0)
\newenvironment{GOneSpec}{%
  \begin{goldcheck}[Monotone Entropy (\(\Delta\Entropy \ge 0\))]
}{%
  \end{goldcheck}
}

% ===== Title =====
\title{V — The Verification Tensor\\\large Module Contract and Test Harness for G1S (\(\Delta\Entropy\ge 0\) under Causal Refinements)}
\author{Verification Layer (V)\\\small Non-invasive interface for G1S; P\_clean remains authoritative}
\date{\today}

\begin{document}
\maketitle

\begin{abstract}
This document specifies the \emph{verification layer} (V) for the G1S module. Its goals are: (i) to make the target claim of G1S explicit and machine-checkable; (ii) to supply a guarded notation interface so that G1S compiles and reads cleanly; and (iii) to render all checks \emph{non-invasive} so that \texttt{P\_clean.tex} remains unchanged and authoritative. The core object is the monotonicity of entropy (distinguishability) under Martin-consistent causal refinements with a fixed observational map~\(\pi\): \(\Delta\Entropy \ge 0\).
\end{abstract}

\tableofcontents

\section{Purpose and Scope}
The V layer defines the \emph{module contract} for G1S and a minimal test harness (\emph{Gold Checks}). The contract pins down:
\begin{itemize}[leftmargin=2.2em]
  \item the refinement relation on causal structures;
  \item the entropy functional used by G1S;
  \item the invariants G1S must respect; and
  \item a compact checklist for pass/fail marking without touching global proof macros.
\end{itemize}
All shared macros in this file are wrapped in \verb|\providecommand| so that if they already exist in \texttt{P\_clean.tex}, they are left untouched.

\section{Module Contract for G1S}
\subsection{Causal Refinements}
\begin{definition}[Causal Refinement]\label{def:refinement}
Let \((\C,\preceq)\) be a locally finite partial order. A \emph{refinement} is an order-embedding
\(\,\phi: \C \to \C'\,\) into another locally finite poset \((\C',\preceq')\) that preserves and reflects order and is \emph{Martin-consistent} with local finiteness. We write
\[
  (\C,\preceq)\xrightarrow{\;\mathrm{ref}\;}(\C',\preceq').
\]
\end{definition}

\begin{remark}[Local Finiteness and Martin-Consistency]
Local finiteness ensures finite causal intervals. Martin-consistency here is the meta-constraint that admissible refinements do not introduce pathologies that break countability or local finiteness assumptions used throughout the proof corpus.
\end{remark}

\subsection{Entropy as Distinguishable Count}
\begin{definition}[Entropy under a Fixed Observable]\label{def:entropy}
Let \(\pi: \C \to \OmegaSet\) be an observational coarse-graining (equivalence on events). Define
\[
  \Entropy[\pi](\C) \;=\; \ln\, \Dist\!\big(\pi(\C)\big),
\]
with physical units set by \(\kB\) when required. Intuitively, \(\Entropy\) measures the number of distinguishable outcome classes under \(\pi\).
\end{definition}

\begin{definition}[Monotone Entropy Claim for G1S]\label{def:g1s-claim}
For any refinement \((\C,\preceq)\to(\C',\preceq')\) as in Def.~\ref{def:refinement}, with the \emph{same} observable \(\pi\), the G1S module must establish
\[
  \Delta\Entropy \;=\; \Entropy[\pi](\C')\,-\,\Entropy[\pi](\C) \;\ge\; 0.
\]
The proof must depend only on refinement properties and local finiteness, not on additional physical law.
\end{definition}

\section{Invariants}
\begin{invariant}[Interface Stability (I\textsubscript{1})]\label{inv:I1}
G1S must not introduce or require global macro/environment redefinitions that conflict with the proof corpus. It must compile using only (i) the guarded shared-notation above and (ii) its own local macros.
\end{invariant}

\begin{invariant}[Monotone Distinguishability (I\textsubscript{2})]\label{inv:I2}
Within the G1S scope (Martin-consistent, locally finite refinements with fixed \(\pi\)), distinguishability is non-decreasing:
\(\;\Delta\Entropy\ge 0\;\).
\end{invariant}

\begin{invariant}[Reproducibility (I\textsubscript{3})]\label{inv:I3}
All claims must be checkable by simple combinatorial counting under \(\pi\); no continuum-analytic facts outside local finiteness are required for the G1S proofs.
\end{invariant}

\section{Reference Argument (Minimal Dependencies)}
This section provides a reference proof skeleton that G1S may quote or specialize. It uses only the objects declared above.

\begin{theorem}[Refinement Non-Destruction of Distinctions]\label{thm:non-destruction}
Let \(\phi: \C\to\C'\) be an order-embedding refinement (Def.~\ref{def:refinement}) and \(\pi\) fixed. Then any two equivalence classes distinguished in \(\pi(\C)\) remain distinguished or further split in \(\pi(\C')\). Hence
\(\;\Dist(\pi(\C'))\ge \Dist(\pi(\C))\;\) and therefore \(\Delta\Entropy\ge 0\).
\end{theorem}
\begin{proof}[Proof sketch]
Order-embedding preserves causal relations, so events that were distinguishable due to distinct causal traces or observational labels under \(\pi\) cannot be identified by refinement; at worst they remain distinct, at best refinement introduces additional resolution, splitting a class into several. Since \(\pi\) is held fixed, no previously distinct labels are coalesced by a change in the observable. Local finiteness prevents divergence in a finite step, and Martin-consistency prohibits pathological identifications. Therefore the count of equivalence classes does not decrease.
\end{proof}

\begin{remark}[Where Monotonicity Can Fail]
A decrease in \(\Entropy\) can only arise if (i) \(\pi\) is changed between \(\C\) and \(\C'\), (ii) refinement is not order-embedding, or (iii) local finiteness/Martin-consistency is violated. These cases are outside the G1S contract and must be rejected by V.
\end{remark}

\section{Gold Checks: G1 Harness}
The following checklist operationalizes the G1 requirement. G1S should include an instance of this environment at the end of each example or lemma that claims \(\Monotone\).

\begin{GOneSpec}
  \item \gpass\; \textbf{Interface}: Symbols used appear in the guarded set or are locally defined (Inv.~\ref{inv:I1}).
  \item \gpass\; \textbf{Refinement}: Map \(\phi\) is order-embedding and Martin-consistent (Def.~\ref{def:refinement}).
  \item \gpass\; \textbf{Observable}: \(\pi\) unchanged between \(\C\) and \(\C'\) (Def.~\ref{def:entropy}).
  \item \gpass\; \textbf{Claim}: Exhibit \(\Entropy[\pi](\C') - \Entropy[\pi](\C) \ge 0\) using only the contract dependencies.
  \item \gpass\; \textbf{Edge Cases}: Ties/merges are ruled out or shown to preserve distinguishability under fixed \(\pi\).
\end{GOneSpec}

\section{Verification Loop (Non-Invasive)}
To keep \texttt{P\_clean.tex} unchanged, V enforces the following loop on any candidate G1S revision:
\begin{enumerate}[leftmargin=2.2em]
  \item \textbf{Interface Freeze}: Reject any revision that compiles only by redefining a macro present in the proof corpus; G1S must adhere to the guarded layer (Inv.~\ref{inv:I1}).
  \item \textbf{Refinement Legality}: Verify \(\phi\) satisfies Def.~\ref{def:refinement}, including Martin-consistency.
  \item \textbf{Observable Lock}: Confirm the same \(\pi\) is used on both sides (Def.~\ref{def:entropy}).
  \item \textbf{Monotonicity Check}: Run the G1 harness and verify that all checklist items pass.
  \item \textbf{Minimal Dependencies}: Ensure the argument uses only local finiteness and the fixed observable (Inv.~\ref{inv:I3}).
\end{enumerate}

\section{Worked Micro-Example (Schematic)}
This toy example is intentionally abstract and \emph{non-binding} on physics; it demonstrates how G1S should structure its internal checks.

\paragraph{Setup.} Let \(\C=\{e_1,e_2\}\) with no relation (antichain), \(\pi(e_i)=i\). Then \(\Dist(\pi(\C))=2\), \(\Entropy[\pi](\C)=\ln 2\).

Refine to \(\C'=\{e_1^a,e_1^b,e_2\}\) with order embedding that refines only \(e_1\) into two sub-events preserving incomparability with \(e_2\). Keep \(\pi\) fixed but more resolving on \(e_1\): \(\pi(e_1^a)=1a\), \(\pi(e_1^b)=1b\), \(\pi(e_2)=2\). Then \(\Dist(\pi(\C'))=3\), \(\Entropy[\pi](\C')=\ln 3\), so \(\Delta\Entropy=\ln(3/2)>0\).

\begin{GOneSpec}
  \item \gpass\; Interface uses only guarded/local symbols.
  \item \gpass\; \(\phi\) is order-embedding, locally finite.
  \item \gpass\; Same \(\pi\) applied before/after (labels refined, observable unchanged).
  \item \gpass\; \(\ln 3 - \ln 2 \ge 0\) established.
  \item \gpass\; Edge cases: none; no merges of previously distinct classes.
\end{GOneSpec}

\section{Fixed Point and Reporting}
When the G1 harness passes across all declared examples and lemmas within G1S, V records the module’s status as \(\FixedPoint(\mathrm{G1})=\mathrm{true}\). Any subsequent edit to G1S that would force a global macro redefinition or alter the observable \(\pi\) invalidates the fixed point and must be rejected until the contract is re-satisfied.

\section*{Implementation Notes}
\begin{itemize}[leftmargin=2.2em]
  \item This file intentionally \emph{does not} redefine global proof macros. All symbols here are optional fallbacks via \verb|\providecommand|.
  \item G1S may add locally-scoped macros within its own file; V only checks interface and claims.
  \item If G1S needs additional symbols, prefer local definitions or submit a guarded addition to this file.
\end{itemize}

\end{document}
