\documentclass[12pt]{article}
\usepackage{amsthm,amssymb,amsmath,mathtools,setspace}
\usepackage{enumitem} % Added for custom enumeration environments
\onehalfspacing

% --- Theorem environments and macros (V.tex aligned) ---
\newtheorem{theorem}{Theorem}
\newtheorem{definition}{Definition}
\newtheorem{proposition}{Proposition}
\newcommand{\U}{\mathbf{U}}        % Universe tensor (admissible history)
\newcommand{\V}{\mathbf{V}}        % Test function / variation
\newcommand{\D}{\mathrm{D}}        % Derivative operator (for 1D)
\newcommand{\Lap}{\Delta}          % Laplacian (ND generalization)
\newcommand{\Acal}{\mathcal{A}}    % Informational curvature action (V.tex Def. 1)
\newcommand{\Bform}{\mathsf{B}}    % Symmetric bilinear form (V.tex Def. 1)
\newcommand{\Hspace}{\mathcal{H}}  % Trial space (e.g., H^2)
\newcommand{\Test}{\mathcal{V}}    % Test/variation space
\newcommand{\deltaVar}{\delta}     % First variation
\newcommand{\Res}{\mathsf{R}}      % Residual (source)
\newcommand{\mesh}{\mathsf{h}}     % Mesh scale (discrete)
\newcommand{\Spline}{\mathsf{S}}   % Spline/interpolant

% --- Optional: Gold Check Harness (V.tex aligned) ---
\newcounter{goldcheck}
\newenvironment{goldcheck}[1][]{%
  \refstepcounter{goldcheck}%
  \par\medskip\noindent\textbf{Gold Check G\thegoldcheck. #1}\par\smallskip\noindent
  \begin{enumerate}[leftmargin=2.2em,label=\arabic*.]
}{%
  \end{enumerate}\medskip
}
\newcommand{\gpass}{\textbf{[PASS]}}
\newcommand{\gfail}{\textbf{[FAIL]}}
\newenvironment{GTwoSpec}{\begin{goldcheck}[G2 Variational Reciprocity]}{\end{goldcheck}}
\newenvironment{varobligation}[1][]{%
  \par\medskip\noindent\textbf{G2 Proof Obligation.} #1\;\emph{(Provide stationarity, reciprocity, and refinement consistency.)}%
  \begin{enumerate}[leftmargin=2.2em,label=\alph*.]
}{%
  \end{enumerate}\medskip
}

\title{\textbf{Module 2: Kinematic Closure and Splines}\\
\large V.tex-Compliant Contract for G2S}
\author{—}
\date{\today}

\begin{document}
\maketitle

% --- Working basis (aligns with V.tex half-step) ---
\noindent\textbf{Working Basis.} We assume ZFC, Martin’s Axiom for relevant ccc posets,
and the Axiom of Event Selection (locally finite, realizable histories).

\section{Input Node: Reciprocity and Variational Setup}

We formalize the unique analytic closure that minimally interpolates the discrete events guaranteed by consistency.
Let the trial space be $\Hspace \subset H^2(I)$ on an interval $I$, with boundary or knot
constraints enforcing interpolation/BCs; let the test space $\Test$ be the corresponding
space of admissible variations (compact support or variations compatible with the BCs).

\begin{definition}[Informational Curvature Action $\Acal$ (V.tex Def. 1)]\label{def:action}
For $\U\in\Hspace$, the informational bending energy (or curvature action) is defined by
\[
  \Acal[\U] = \tfrac12 \int_I (\D^2 \U)^2\,dx.
\]
In 1D, the Euler–Lagrange operator for stationarity is $\D^4$ (biharmonic). In general, $\D^2$ represents the Laplacian operator $\Lap$ and $\D^4$ the biharmonic operator $\Lap^2$ (use $\Lap$ and $\Lap^2$ in ND).
\end{definition}

\begin{definition}[Reciprocity Bilinear Form $\Bform$ (V.tex Def. 1)]\label{def:bilinear}
Define the symmetric bilinear form on admissible variations $\Bform:\Test\times\Test\to\mathbb{R}$ by
\[
  \Bform(\phi,\psi) := \int_I (\D^2 \phi)(\D^2 \psi)\,dx.
\]
The Action is recovered as $\Acal[\U] = \tfrac12\,\Bform(\U,\U)$ when identifying $\U$ with its variation in $\Test$. (use $\Lap$ and $\Lap^2$ in ND).
\end{definition}

\begin{definition}[Residual $\Res$]\label{def:residual}
The weak problem is to find $\U\in\Hspace$ such that
\[
  \Bform(\U,\V) = \Res(\V)\qquad\forall\,\V\in\Test.
\]
For the homogeneous problem of finding the minimal informational curvature (no external sources), $\Res\equiv 0$. Hence the weak form is homogeneous.
\end{definition}

\section{Theorem: Kinematic Closure (Validating G2S)}

\begin{theorem}[Kinematic Closure $\,\D^4 \U=0$]\label{thm:kinematic-closure}
The stationary point $\U^\star\in\Hspace$ of $\Acal$ satisfies the weak form $\Bform(\U^\star,\V)=0$ for all $\V\in\Test$, hence (under the stated boundary conditions, BCs) the strong form
\[
  \D^4 \U^\star = 0 \quad \text{in } I,
\]
i.e., the minimal-curvature condition (cubic spline segments between knots) required for Martin-consistent propagation.
\end{theorem}

\paragraph{BCs used for (IBP).}
Either natural spline BCs ($\U''=\U'''=0$ at $I$'s endpoints) or clamped BCs
($\U, \U'$ fixed at endpoints); variations $\V$ respect these BCs.

\begin{proposition}[Affine kernel and coercivity]\label{prop:kernel-coercive}
On $H^2(I)$ with the BCs above (and interpolation knots),
$\ker \Bform = \{ a+bx : a,b\in\mathbb{R} \}$. Consequently, $\Bform$ is coercive
on the constrained subspace (or on the quotient by this kernel), ensuring uniqueness of $\U^\star$.
\end{proposition}

\begin{proof}[\textbf{G2 Proof Obligation Fulfillment}]
\begin{varobligation}
  \item \textbf{Stationarity (Euler–Lagrange).} For $\V\in\Test$, the first variation of the action is:
  \[
    \deltaVar\Acal[\U](\V)
      = \left.\frac{d}{d\epsilon}\Acal[\U+\epsilon\V]\right|_{\epsilon=0}
      = \int_I (\D^2\U)(\D^2\V)\,dx
      = \Bform(\U,\V).
  \]
  Two integrations by parts (BCs eliminate boundary terms) yield the identity
  $\deltaVar\Acal[\U](\V) = \int_I (\D^4\U)\,\V\,dx$. Stationarity (vanishing variation for all $\V$)
  therefore implies the strong form $\D^4\U=0$.
  \item \textbf{Reciprocity.} The bilinear form $\Bform(\phi,\psi)$ is symmetric by definition, reflecting reciprocity.
  It is coercive on the constrained subspace (or on the quotient by the affine kernel)
  induced by the boundary conditions and interpolation knots, guaranteeing the existence and uniqueness of $\U^\star$ (cf. Proposition \ref{prop:kernel-coercive}).
  \item \textbf{Weak Form.} With the residual $\Res\equiv 0$ (Definition \ref{def:residual}), the problem is:
  find $\U\in\Hspace$ such that $\Bform(\U,\V)=0$ for all $\V\in\Test$.
  \item \textbf{Refinement Consistency.} Let $\U_\mesh$ be the discrete informational record and
  $\Spline_\mesh$ the cubic spline interpolant. Then the convergence
  $\|\Spline_\mesh\U_\mesh - \U^\star\|_{H^2(I)}\to 0$ as $\mesh\to 0$ holds (classical spline theory),
  validating $\U^\star$ as the continuum limit of the discrete reciprocal updates.
\end{varobligation}
\end{proof}

\section{Output Node: Gold Check}

\begin{GTwoSpec}
  \item \gpass\; \textbf{Spaces}: $\Hspace\subset H^2(I)$ and $\Test$ specified; BCs/nodes fix the affine kernel.
  \item \gpass\; \textbf{Action}: $\Acal[\U]=\tfrac12\int (\D^2\U)^2 dx$; $\deltaVar\Acal=\Bform(\U,\cdot)$.
  \item \gpass\; \textbf{Reciprocity}: $\Bform$ symmetric; coercive on constrained space / quotient.
  \item \gpass\; \textbf{Weak Form}: $\Bform(\U,\V)=0$ for all $\V\in\Test$.
  \item \gpass\; \textbf{Spline Consistency}: $\Spline_\mesh\U_\mesh\to \U^\star$ as $\mesh\to 0$.
\end{GTwoSpec}

\begin{center}
\textit{Conclusion for Module 2: Smooth motion is the unique analytic form of consistency, established via a reciprocal variational principle.}
\end{center}

\end{document}
