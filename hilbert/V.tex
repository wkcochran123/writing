% ==========================
% V.tex — The Vulgate / View
% A reader-first companion to hilbert.tex (rigorous proof text)
% Minimal deps: standard LaTeX (article)
% ==========================
\documentclass[12pt]{article}

\usepackage[margin=1in]{geometry}
\usepackage{amsmath, amssymb}
\usepackage{amsthm}
\usepackage{hyperref}

% --- Simple theorem styles (no extra pkgs) ---
\newtheorem{theorem}{Theorem}
\newtheorem{definition}{Definition}
\newtheorem{remark}{Remark}

% --- Tiny macros consistent with hilbert ---
\newcommand{\U}{\mathbf{U}}             % Proof/Universe tensor (semantic stand-in)
\newcommand{\Ek}{\mathbf{E}_k}          % Event tensor
\newcommand{\Un}{\mathbf{U}_n}          % Universe tensor (ordered fold)
\newcommand{\grad}{\nabla}
\newcommand{\T}{\mathbf{T}}             % Stress / bookkeeping tensor
\newcommand{\g}{g}                      % Metric (components 

\title{\vspace{-1.0em}\textbf{The Calculus of Measurement and Causal Order}\\
\large A Reader’s Guide to the Constructive Proof}
\author{Companion to \textit{hilbert.tex}}
\date{\today}

\begin{document}
\maketitle
\vspace{-1.5em}
\begin{table}[h]
\centering
\caption{Metric Tensor $g_{\mu\nu}$}
\begin{tabular}{lll}
\hline
\textbf{Component} & \textbf{Physical (Contravariant)} & \textbf{Logical (Covariant)} \\
\hline
$T_{00}$ & Energy density & Informational density \\
$T_{0i}$ & Momentum flux & Communication flux \\
$T_{ij}$ & Stress tensor & Logical coupling \\
\hline
\end{tabular}
\end{table}
$g_{\mu\nu}$ inline)
\newcommand{\dd}{\mathrm{d}}

\section*{Purpose of this Companion}
This short document (\textbf{V.tex}) is a navigable \emph{map} of the full constructive proof developed in \textbf{hilbert.tex}. It makes the resolution explicit: how minimal logical axioms (the \emph{coarse grid}) map covariantly to physical conclusions (the \emph{fine grid}). Use this as a one-sitting overview or as marginal scaffolding while reading the full text.

\section{The Axioms of Informational Necessity}
\textbf{Premise.} Physics is a theorem of \emph{causal consistency}. We begin with the least structure needed for a universe to maintain a non-contradictory record of its own evolution.

\subsection*{A. Foundation: ZFC and Ordinal Time}
\textbf{Abstraction (Set).} The universe’s \emph{record} is the set of distinguishable events $E$.

\textbf{Counting (Order).} Under ZFC with Choice, events can be well-ordered. \emph{Time} is not a scalar background; it is an \emph{ordinal rank} $k\in\mathbb{N}$ that indexes the sequence of distinctions.

\subsection*{B. Causal Constraint: Event Selection (MA-like)}
\textbf{Axiom (Event Selection).} For any countable family of admissible local causal choices, there exists a single, globally consistent extension. Intuitively: the universe can always \emph{choose coherently}. This logical regularity is the source of kinematics and, ultimately, the Second Law in this framework.

\section{Reciprocity and the Proof Tensor}
We collect distinctions into tensors so that “what is measured” and “how it varies” stay in bijection.

\begin{definition}[Event and Universe Tensors]
An \emph{Event Tensor} $\Ek$ encodes the $k$-th measured distinction.
The \emph{Universe (Proof) Tensor} is the ordered fold $\Un = \sum_{k=1}^n \Ek$.
\end{definition}

\begin{definition}[Reciprocity Law of Physics]
Every admissible \emph{variation} corresponds to a \emph{measurement} of distinction, and vice versa. Denote this duality by $\Phi:\text{Variation}\leftrightarrow\text{Measurement}$.
\end{definition}

\subsection*{The Metric as Operational Map}
Let $g_{\mu\nu}$ be the \emph{metric of information} that keeps co-/contravariant components of the Proof Tensor unified as a scalar invariant. Practically, $g_{\mu\nu}$ is the non-arbitrary recipe that preserves meaning when we pass between coarse (logical) and fine (physical) descriptions.

\paragraph{The Metric Table (Coarse $\leftrightarrow$ Fine).}
\begin{center}
\renewcommand{\arraystretch}{1.25}
\begin{tabular}{p{0.36\linewidth} c p{0.36\linewidth}}
\hline
\centering \textbf{Physical (Fine, $U^{\mu}$)} & $g_{\mu\nu}$ & \centering \textbf{Logical (Coarse, $U_{\mu}$)}\\
\hline
\centering Time/Duration $(t)$ & $\leftrightarrow$ & \centering Ordinal Rank $(k\in\mathbb{N})$\\
\centering Kinematics $(\begin{theorem}[Metric Coherence]\label{thm:u4}
Under the Axiom of Event Selection (MA-like), the universe tensor satisfies $U^{(4)}=0$.
\end{theorem}
U^{(4)}=0)$ & $\leftrightarrow$ & \centering Global Consistency (Event Selection / MA-like)\\
\centering Entropy $(S)$ & $\leftrightarrow$ & \centering Count of Distinctions $(\ln N)$\\
\centering Mass/Energy $(T_{\mu\nu})$ & $\leftrightarrow$ & \centering Conserved Bookkeeping $(\grad_\mu T^{\mu\nu}=0)$\\
\hline
\end{tabular}
\end{center}
\noindent\emph{Reading note.} The constructive proof is the process of \emph{resolving this metric} across domains so the same scalar invariants are preserved under representation changes.

\section{The First Two Theorems (Immediate Consequences)}
\begin{theorem}[\begin{theorem}[Second Law of Causal Order]\label{thm:deltaS}
For all causal evolutions of a measurable system, $\Delta S \ge 0$.
\end{theorem}
Second Law of Causal Order]\label{thm:DeltaS}
Under Event Selection (global consistency), any admissible refinement of a finite causal order increases distinguishability. Hence the entropy is monotone:
\[
\Delta S \;\ge\; 0.
\]
\textit{Meaning.} The Arrow of Time is a theorem of order: the universe cannot lose track of distinctions it has coherently recorded.
\end{theorem}

\begin{theorem}[Kinematic Closure / Minimal Curvature]\label{thm:U4}
Minimal informational friction enforced by consistency yields the cubic-spline condition
\[
U^{(4)} \;=\; 0,
\]
the unique smooth closure compatible with observation. This is the kinematic law in this framework (the smooth limit of finite causal measurement).
\end{theorem}

\section{Arc of the Proof (Dependency Chain)}
\[
\boxed{\;\text{Axioms (ZFC + Event Selection)} \Rightarrow \text{Reciprocity} \Rightarrow U^{(4)}=0 \Rightarrow \grad_\mu T^{\mu\nu}=0 \Rightarrow \Delta S \ge 0\;}
\]

\subsection*{Computational Ceiling (Forward Pointer)}
\begin{center}
\renewcommand{\arraystretch}{1.2}
\begin{tabular}{p{0.28\linewidth} p{0.40\linewidth} p{0.26\linewidth}}
\hline
\centering \textbf{Volume Function} & \centering \textbf{Core Mathematical Identity} & \centering \textbf{Physical Conclusion}\\
\hline
\centering II/III Transition & \centering $P{=}NP \;\Rightarrow\;$ Weierstrass (dense, smooth approximation) & \centering Continuity is computationally guaranteed.\\
\centering IV Conclusion & \centering $ZFC \;\Rightarrow\; \mathbb{R}\le \mathbb{C}$ (representation bound) & \centering Limit of Representation is a constant.\\
\hline
\end{tabular}
\end{center}
\noindent\emph{Idea.} Reducing physics to the cost of representation (the “computational ceiling”) explains why the smooth story holds and where it must break.

\section{Glossary (Causal Multigrid, one-liners)}
\begin{description}
\item[Causal Order (poset).] Locally finite precedence relation on $E$; \emph{time} is ordinal rank.
\item[Event Tensor $\Ek$.] The $k$-th recorded distinction mapped into tensor space.
\item[Universe/Proof Tensor $\Un$.] Ordered fold of events; the constructive record.
\item[Event Selection.] MA-like global consistency: all countable local choices can be glued coherently.
\item[Reciprocity $\Phi$.] Bijection between measurable distinctions and admissible variations.
\item[Metric $g_{\mu\nu}$.] Operational map preserving invariants across domains (coarse/fine).
\item[Kinematic Closure $U^{(4)}{=}0$.] Minimal-curvature smooth limit of finite causal measurement.
\item[Conserved Bookkeeping $\grad_\mu T^{\mu\nu}{=}0$.] Noether-style conservation as statistical identity.
\item[Entropy $S$.] $\ln N$ of admissible distinguishable configurations; monotone under consistency.
\item[Final Residual.] The representation/complexity cost left to resolve (Volumes II–IV).
\end{description}

\section*{How to Read with This Guide}
\textbf{1 page:} Metric table + Theorems~\ref{thm:DeltaS}--\ref{thm:U4}. \quad
\textbf{15 pages:} Chapter~1 of \textbf{hilbert.tex} with this glossary. \quad
\textbf{Full arc:} Follow the boxed chain section-by-section.

\vfill
\begin{center}
\small
\textit{This document is a didactic companion. All claims are conditional on the axioms stated.}\\
For proofs and full formalism, see \textbf{hilbert.tex}.
\end{center}

\end{document}

