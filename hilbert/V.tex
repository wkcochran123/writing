\documentclass[12pt]{article}
\usepackage{amsmath, amssymb, amsthm, geometry, booktabs, hyperref}
% === Invariant Environment Definition ===
\newtheoremstyle{invariantstyle}% name
  {\topsep}   % Space above
  {\topsep}   % Space below
  {\itshape}  % Body font
  {}          % Indent
  {\bfseries} % Head font
  {.}         % Punctuation after theorem header
  {.5em}      % Space after theorem header
  {}          % Theorem head spec

\theoremstyle{invariantstyle}
\newtheorem{invariant}{Invariant}[section]

% === Definition Environment Definition ===
\newtheoremstyle{definitionstyle}% name
  {\topsep}   % Space above
  {\topsep}   % Space below
  {\normalfont} % Body font (upright, not italic)
  {}          % Indent amount
  {\bfseries} % Theorem head font
  {.}         % Punctuation after theorem header
  {.5em}      % Space after theorem header
  {}          % Head spec

\theoremstyle{definitionstyle}
\newtheorem{definition}{Definition}[section]


\geometry{margin=1in}
\hypersetup{colorlinks=true, linkcolor=blue, citecolor=blue}

\title{\textbf{V.tex: The Verification Tensor} \\ Maintaining Logical and Variational Consistency of the Constructive Proof}

\author{W.\,K.\,Cochran Jr.}
\date{\today}

\begin{document}

\maketitle

\begin{abstract}
This document defines the \textit{Verification Tensor}~$V$: a persistent, formal specification ensuring that the logical topology established in \texttt{P\_clean.tex} remains invariant under subsequent transformations.  
$V$ provides both qualitative invariants and quantitative metrics for verifying that each future production tensor~$P_t$ preserves the structural, variational, and informational coherence achieved by the baseline manifold~$M_0$.
\end{abstract}

\section{Purpose and Scope}
The original role of~$V$ was diagnostic: to identify and deflate inconsistent eigenmodes in the construction of~$P$.  
Now that \texttt{P\_clean.tex} has achieved structural coherence and finite logical depth, $V$ transitions to a \emph{control tensor} whose task is to verify continuity, not repair deviation.

The Verification Tensor defines:
\begin{itemize}
  \item Structural invariants~$I_i$ that define the manifold~$M_0$ of coherent logical structure.
  \item Variational checks that ensure physical and mathematical arguments remain dual and continuous.
  \item A quantitative measure~$\Delta K_t$ of informational drift (Kolmogorov variation) between successive revisions~$P_t$.
\end{itemize}

\section{Structural Invariants}
Each chapter~$C_i$ in the constructive proof satisfies the following invariants:

\begin{invariant}[Hierarchy Consistency]
Each chapter~$C_i$ contains a unique subsequence $(S_{i,1},\ldots,S_{i,n})$ such that:
\begin{enumerate}
    \item $\mathrm{parent}(S_{i,k}) = C_i$;
    \item $\mathrm{child}(S_{i,k}) \subseteq C_i$;
    \item No cross-parent duplication exists;
    \item Each section terminates in a definable conclusion or reference within $C_i$.
\end{enumerate}
\end{invariant}

\begin{invariant}[Referential Closure]
All cross-references resolve to existing labels; no symbolic pointers remain open.
This defines the topological closure of~$M_0$ under the logical mapping $\phi: \text{Section} \to \text{Reference}$.
\end{invariant}

\section{Verification Metrics}
The \emph{Gold Checks} serve as the minimal set of logical–variational conditions required for global coherence.

\begin{table}[h!]
\centering
\caption{Verification Table (\textbf{V1.0})}
\begin{tabular}{@{}ll@{}}
\toprule
\textbf{ID} & \textbf{Criterion} \\ \midrule
G1 & Formal statement of $\Delta S \ge 0$ present and consistent. \\
G2 & Coherence operator $U^{4} = 0$ explicitly invoked at point of use. \\
G3 & Spline $\Rightarrow$ Euler–Lagrange derivation complete and unique. \\
G4 & Reciprocity functional includes boundary terms and correct sign convention. \\
G5 & Martin's Axiom $\Leftrightarrow$ Continuity boundary labeled and referenced. \\ \bottomrule
\end{tabular}
\end{table}

Each future version~$P_t$ is deemed \emph{verified} if and only if all $G_i$ hold.

\section{Coherence Operator \(U^{4}=0\)}
The coherence condition defines the logical closure of the proof manifold.
\begin{definition}[Coherence Operator]
Let $U^{4}$ denote the fourth-order closure operator on the hierarchy of definitions and propositions in $P_t$.  
Then the system is \emph{coherent} iff $U^{4}=0$.
\end{definition}

This condition must appear explicitly within each revision of the constructive proof wherever closure is asserted or continuity is invoked.

\section{Kolmogorov Drift Measure}
To maintain informational fidelity between revisions, define
\[
\Delta K_t = K(P_t) - K(P_{\mathrm{clean}}).
\]
Here $K(\cdot)$ denotes the estimated Kolmogorov complexity of the encoded document.  
The verification condition is
\[
|\Delta K_t| < \varepsilon,
\]
where $\varepsilon$ is the allowable informational drift (typically one order of magnitude below the average paragraph entropy).

A non-zero $\Delta K_t$ indicates either redundancy (positive drift) or informational loss (negative drift).  
The Verification Tensor enforces $\Delta K_t \approx 0$ as the condition of structural and semantic equilibrium.

\section{Future Iteration Protocol}
For each new revision~$P_{t+1}$:
\begin{enumerate}
    \item Compute $\Delta K_{t+1}$ relative to $P_{\mathrm{clean}}$.
    \item Verify Gold Checks $G1$–$G5$.
    \item Confirm invariants $I_1$–$I_2$.
    \item Only if $\Delta K_{t+1} < \varepsilon$ and all $G_i$ pass, mark $P_{t+1}$ as \emph{consistent}.
\end{enumerate}

\section{Conclusion}
$V$ thus becomes the steady-state guardian of logical integrity.  
It transforms verification from a manual audit into a formal, repeatable process that measures the continuity of reasoning itself.  
As new chapters and tensors are introduced, $V$ defines the metric space in which coherence is tested, ensuring that $\Delta S \ge 0$ and $U^{4}=0$ remain inviolable logical truths.

\end{document}
