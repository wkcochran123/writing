\chapter{Proofs}
\label{app:proofs}
\propproof{universe-tensor}
This argument is standard in category theory; see Mac~Lane~\cite{maclane1971}
for the classical formulation of naturality in monoidal categories.
\begin{proof}[Proof (ZFC).]
\noindent\emph{Conceptually, this is the demonstration of the naturality square
for the embedding $\Phi$ in the monoidal category of tensor algebras, although
we have presented it here entirely within ZFC.}


All objects are sets.  Let $E$ be the event space.  A record of length $n$ is
an $n$--tuple of indices in $\mathbb{N}$:
\[
\mathbf{i} = (i_1,\dots,i_n) \in \mathbb{N}^n.
\]
Each index $i_k$ refers to an event $e_{i_k}\in E$, and we collect these via the
evaluation map
\[
\pi_E : \mathbb{N}^n \to E^n,\qquad
\pi_E(i_1,\dots,i_n) := (e_{i_1},\dots,e_{i_n}).
\]

\paragraph{Restriction.}
A restriction operator is a function
\[
\widehat{R} : \mathbb{N}^n \longrightarrow \mathbb{N}^m ,
\qquad n \ge m,
\]
returning a shorter admissible record.  Its action on events is defined
componentwise through $\pi_E$.

\paragraph{Embedding.}
Let $\Phi : E \to T(V)$ be the event embedding into the tensor algebra.  Lift it
componentwise to records by
\[
\Phi^{(k)} : E^k \to T(V)^k,\qquad
\Phi^{(k)}(x_1,\dots,x_k) := (\Phi(x_1),\dots,\Phi(x_k)).
\]
Define the induced restriction on embedded factors by componentwise selection:
\[
R^{(m)} : T(V)^m \to T(V)^m.
\]

\paragraph{Naturality.}
We claim the square
\[
R^{(m)}\circ\Phi^{(n)}\circ\pi_E
\;=\;
\Phi^{(m)}\circ\pi_E\circ\widehat{R}
\qquad\text{as maps }\mathbb{N}^n\to T(V)^m.
\]
Let $\mathbf{i}=(i_1,\dots,i_n)\in\mathbb{N}^n$.  Then
\[
\Phi^{(n)}\circ\pi_E(\mathbf{i})
  = (\Phi(e_{i_1}),\dots,\Phi(e_{i_n})).
\]
Applying $R^{(m)}$ selects the $m$ refined components of the admissible record.
On the other hand,
\[
\widehat{R}(\mathbf{i}) = (j_1,\dots,j_m),
\]
and so
\[
\Phi^{(m)}\circ\pi_E\circ\widehat{R}(\mathbf{i})
  = (\Phi(e_{j_1}),\dots,\Phi(e_{j_m})).
\]
Both sides produce the same $m$ embedded admissible events.  Thus the square
commutes.

\paragraph{Tensor product update.}
Given a record $\mathbf{i}\in\mathbb{N}^n$, define the cumulative factor
\[
U(\mathbf{i}) := \prod_{k=1}^n \Phi(e_{i_k})
\]
using the fixed associative product in $T(V)$.  If $\widehat{R}$ eliminates or
reorders indices corresponding to incomparable events, the product over the
refined record is
\[
U(\widehat{R}(\mathbf{i})) = \prod_{\ell=1}^m \Phi(e_{j_\ell}).
\]

\paragraph{Independence under commuting factors.}
If $e_{i_p}$ and $e_{i_q}$ are incomparable, their embeddings commute:
$\Phi(e_{i_p})\Phi(e_{i_q})=\Phi(e_{i_q})\Phi(e_{i_p})$.  Any two admissible
refinements differ by permutations of indices of such incomparable elements,
hence
\[
U(\mathbf{i}) = U(\sigma(\mathbf{i}))
\]
for any permutation $\sigma$ generated by such swaps.  If all factors commute,
$U(\mathbf{i})$ depends only on the multiset $\{\Phi(e_{i_1}),\dots,\Phi(e_{i_n})\}$
and is therefore independent of the ordering of the record.

\paragraph{Conclusion.}
In ZFC, the operators $\pi_E$, $\widehat{R}$, $\Phi^{(k)}$, and $R^{(m)}$ are
well-defined; the naturality square commutes; the update operator $U$ is
well-defined; and the invariance properties under commuting admissible factors
hold.  This proves the proposition.

One may view this as a commuting diagram in a monoidal category.
\end{proof}


\propproof{reciprocity-unique}
\propproof{pigeonhole}
\propproof{free-parameter}
\propproof{antisym}
\propproof{transitive}
\propproof{simultaneity}
\propproof{informational-decoherence}

