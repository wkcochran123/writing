\chapter{Notation}

This appendix summarizes the symbols and conventions used throughout the
monograph. The goal is clarity. Every notation corresponds to an operational
procedure: recording events, merging ledgers, composing systems, or evolving
a notebook of admissible distinctions forward in time.

\subsection*{Events and Ledgers}

\begin{itemize}
  \item An \emph{event} is a measurable, irreversible update to a system's
  state. A finite set of observations produces a finite, ordered record of
  events.

  \item A \emph{ledger} is the notebook containing this ordered record.
  Ledgers are denoted by calligraphic symbols $(\mathcal{L},\mathcal{M},\dots)$.

  \item The \emph{Axiom of Order} guarantees that every ledger is a countable,
  totally ordered sequence of events.
\end{itemize}

\subsection*{Tensor Composition}

\begin{itemize}
  \item Independent ledgers compose via the tensor product
  \[
    \mathcal{L} \otimes \mathcal{M},
  \]
  which produces a joint ledger with no implied evolution. The tensor product
  is symmetric up to canonical isomorphism and carries no time direction.

  \item The tensor product \emph{does not} imply interaction. It merely
  constructs a space capable of recording joint events.
\end{itemize}

\subsection*{Merge Operator}

\begin{itemize}
  \item Compatible ledgers merge by addition, written
  \[
    \mathcal{L} + \mathcal{M}.
  \]
  This operation is commutative and introduces no new events. It simply
  coalesces distinctions already present in the two ledgers.

  \item Because $+$ is commutative and order-independent, no commutator is
  defined on $+$.
\end{itemize}

\subsection*{Evolution (Fold) Operators}

\begin{itemize}
  \item A \emph{fold} is an evolution operator that acts on a ledger,
  \[
    F : \mathcal{L} \to \mathcal{L}.
  \]
  A fold updates the ledger forward in time, reconciling the existing record
  with a new admissible distinction.

  \item Successive folds are composed using standard function composition,
  \[
    G \circ F : \mathcal{L} \to \mathcal{L}.
  \]
  Only folds carry a time direction.

  \item When the sequence of folds varies with time,
  \[
    \bigcirc_{i=1}^{n} F_i
    \;\;=\;\;
    F_n \circ F_{n-1} \circ \cdots \circ F_1.
  \]
  This is the \emph{iterated fold}.

  \item When the fold is identical at each step, we write
  \[
    F^{\circ n}
    \;\;=\;\;
    \underbrace{F \circ F \circ \cdots \circ F}_{n\text{ times}}.
  \]
  We avoid the notation $F^n$ to prevent confusion with contravariant tensor
  indices.
\end{itemize}

\subsection*{Commutators}

\begin{itemize}
  \item The commutator measures the failure of two folds to commute:
  \[
    [F,G] = F\circ G - G\circ F.
  \]
  Because the tensor product and addition introduce no time ordering, the
  commutator is defined only for folds.

  \item Nonzero commutators represent informational curvature: different
  orders of reconciliation produce different ledgers.
\end{itemize}

\subsection*{Functionals and Variations}

\begin{itemize}
  \item A functional on a ledger-refined trajectory is written
  \[
    J[x] = \int_a^b f(t,x(t),\dot{x}(t))\, dt.
  \]

  \item An admissible variation is of the form
  \[
    x_\varepsilon(t) = x(t) + \varepsilon \eta(t),
  \qquad \eta(a)=\eta(b)=0.
  \]

  \item The first variation is the directional derivative
  \[
    \delta J[x;\eta]
      = \left. \frac{d}{d\varepsilon} J[x_\varepsilon] \right|_{\varepsilon=0}.
  \]

  \item The Euler--Lagrange equation is the condition that $\delta J[x;\eta]=0$
  for all admissible $\eta$.
\end{itemize}

These conventions are used consistently in later chapters. No symbol is
overloaded, and every operator corresponds to a physical or informational
procedure: composition, merging, or evolution. All results follow from these
definitions and the axioms introduced in Chapter~1.

