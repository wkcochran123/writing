\chapter{The Curvature of Information}
\label{chap:residue}

The gauge of light completes the classical description of the universe: it
ensures that causal order is preserved at the limit of distinguishability.
But the universe we observe is not smooth.  Measurements are discrete,
events occur finitely, and the invariants of the causal gauge fluctuate
around their ideal values.  These fluctuations are not errors—they are the
quantum fields of the theory.

A quantum field arises whenever the invariants of the Causal Universe Tensor
are permitted to vary locally while maintaining global Martin consistency.
Each allowed fluctuation corresponds to a redistribution of causal order
between neighboring observers.  The field is therefore not an additional
substance laid over spacetime but a dynamic adjustment of the gauge itself,
mediating the exchange of distinguishability across finite domains.

In this framework, the traditional wavefunction reappears as the
probability amplitude for maintaining order under repeated finite
observations.  Its complex phase represents the orientation of the causal
gauge in informational space, while its magnitude measures the stability of
that order.  The principle of superposition follows directly from the
linearity of causal combinations: multiple consistent histories can coexist
until observation resolves a single extension of the network.

Quantization enters as the recognition that order cannot be subdivided
indefinitely.  Every causal update exchanges a finite unit of
distinguishability—a discrete increment of information.  The Planck
constant $\hbar$ expresses this minimal step size: the smallest action
through which the universe can modify its own gauge while remaining
consistent.  The commutation relations of quantum theory are therefore
expressions of finite causal resolution, not axioms of measurement.

This chapter develops these ideas systematically.  Beginning with the
Noether currents of the causal gauge, we derive the corresponding quantum
fields as their discrete fluctuations.  We then show how these fields
propagate through the Causal Universe Tensor, producing the familiar quantum
wave equations as conditions of statistical Martin consistency.  Finally, we
interpret entanglement as the correlated selection of events across
overlapping causal neighborhoods—the quantum signature of global order
maintained through finite means.

\begin{remark}
Classical physics ends where the gauge of light closes; quantum physics
begins where it wavers.  Every quantum field is a small deviation from
perfect causal consistency, a harmonic of order itself.  The task of this
chapter is to make that statement precise.
\end{remark}


	\section{The Residue of Inconsistency}
	\label{sec:residue-inconsistency}

	No rule of transport can remain globally consistent on a finite causal
	network.  When one carries a distinction around a closed loop of events,
	the recovered configuration generally differs from the initial one.  This
	difference is not an error but an invariant: the measurable residue of
	inconsistency required to preserve local order within a global whole.  In
	differential form, that residue is called curvature.

	\subsection{Curvature as the Measure of Non-Closure}

	The connection $\Gamma^{\lambda}_{\ \mu\nu}$ prescribes how distinctions are
	transported to preserve scalar invariants locally.  When the same
	distinction is transported successively along different paths that enclose
	a finite region, the final result may depend on the path taken.  The
	difference between the two results defines the Riemann curvature tensor:
	\[
	R^{\rho}_{\ \sigma\mu\nu}
	  = \partial_{\mu}\Gamma^{\rho}_{\ \sigma\nu}
	  - \partial_{\nu}\Gamma^{\rho}_{\ \sigma\mu}
	  + \Gamma^{\rho}_{\ \lambda\mu}\Gamma^{\lambda}_{\ \sigma\nu}
	  - \Gamma^{\rho}_{\ \lambda\nu}\Gamma^{\lambda}_{\ \sigma\mu}.
	\]
	This object measures the infinitesimal failure of causal transport to
	commute.  When $R^{\rho}_{\ \sigma\mu\nu}=0$, all paths yield the same
	result and the causal network is globally flat; when it does not vanish,
	the inconsistency cannot be removed by any gauge transformation.

% Placement: Part V (your §5.1 “The Residue of Inconsistency”), immediately after §5.1.1–§5.1.2
\subsubsection*{Example: Casimir Effect as Measured Residue of Non-Closure}

\textbf{Statement.} Boundary-induced mode restriction yields a measurable scalar from the residue of non-closure: the Casimir pressure.

\textbf{Key relation (ideal plates, separation $a$).}
\[
P \;=\; -\,\frac{\pi^2}{240}\,\frac{\hbar c}{a^4}.
\]
\textbf{Reciprocity framing.} Plates impose selection on admissible causal updates (mode partitions). The contraction of the Universe Tensor over admissible modes produces a nonzero scalar residue—the pressure—interpretable as curvature from informational incompleteness.

\textbf{Operational consequence.} Moving a plate changes the equivalence class (refines the partition), and the derivative of the class invariant yields a force, closing the loop between geometry and matter.

	\subsection{Physical Interpretation}

	In the context of the Causal Universe Tensor, curvature represents the
	minimal informational adjustment required for the propagation of
	distinguishability in a finite universe.  Each nonzero component of
	$R^{\rho}_{\ \sigma\mu\nu}$ quantifies how much the local gauge of
	separation must bend to remain self-consistent when extended around a
	closed causal loop.  Curvature is thus the differential trace of the
	universe correcting itself: the physical manifestation of the fact that
	perfect global order is impossible, even though local order is preserved.

	\subsection{Contractions and Scalar Invariants}

	Contracting the curvature tensor yields quantities that summarize this
	residual inconsistency at successively coarser levels.  The Ricci tensor
	\[
	R_{\mu\nu} = R^{\rho}_{\ \mu\rho\nu}
	\]
	measures the local divergence of geodesic families—the rate at which
	neighboring causal paths converge or spread.  The scalar curvature
	\[
	R = g^{\mu\nu} R_{\mu\nu}
	\]
	compresses all such deviations into a single invariant of the causal gauge.
	These contractions represent higher-order scalar invariants of the Causal
	Universe Tensor, extending the chain of conserved quantities that began with
	the spline and the principle of least action.

	\subsection{The Meaning of Curvature in the Causal Framework}

	Traditional geometry interprets curvature as a property of space.  Here it
	is a property of information: a measure of how the network of
	distinguishable events must deform to reconcile finite observation with
	global consistency.  Flatness corresponds to exact commutativity of causal
	updates; curvature, to their minimal non-commutativity.  The universe’s
	curvature is therefore the bookkeeping of necessary inconsistency—the trace
	left by causal order maintaining itself through finite means.

	\begin{remark}
	Curvature is the residue of inconsistency.  It is what remains when the
	rule of causal transport cannot close perfectly, the irreducible difference
	between local and global consistency.  In the language of the Causal
	Universe Tensor, curvature represents the self-correcting property of the
	universe: the differential response by which causal order preserves itself
	in time.  The next section will show that this residue, when balanced
	against the stress encoded in the tensor $T_{\mu\nu}$, yields the Einstein
	equation—the equilibrium condition of the gauge of light.
	\end{remark}

\begin{example}[Extrapolation: Rotation Curves as Order-Preserving Transport]
\textbf{N.B.} This is a conceptual extrapolation about \emph{measurement and causal bookkeeping}, not an astrophysical claim.  It illustrates how a curvature term may arise as the minimal correction that preserves an informational invariant under geometric dilution.  No inference about galaxies, dark matter, or dynamics is intended \cite{boltzmann1872, planck1901, cantor1895, wheeler1983, misner1973, jaynes1957}.

\emph{Setup.}  Imagine a discrete causal disk partitioned into annuli $\{A_r\}$, each containing a locally finite set of events.  
A predicate $P$ counts the distinguishable crossings of a reference ray within each annulus.  
For two anchors $a \prec b$, define the measurement
\[
M_P[r;a,b] := \#\{\,e \in A_r \mid a \prec e \prec b,\, P(e)=1\,\}.
\]
Order-preserving advection (as in §3.6) requires that counts move without creation or loss: transport is conservative \cite{bombelli1987, finkelstein1988}.

\begin{remark}[Observational Context: Vera Rubin and the Dark Matter Problem]
The near-constant tangential velocities observed by Vera Rubin and W. Kent Ford (1970) in spiral galaxies revealed a striking departure from Newtonian expectations: orbital speed remained approximately flat with radius.  
In this framework, that empirical fact is reinterpreted as an order-preserving transport condition rather than evidence for unseen mass.  
The constancy of $v_\theta(r)$ follows from invariance in the count of distinguishable causal updates, not from an additional gravitational component.
\end{remark}


\emph{Invariant to preserve.}  
Let $\Phi(r)$ be the count per causal cycle,
\[
\Phi(r) = \frac{M_P[r;a,b]}{\text{cycles between } a \text{ and } b},
\]
which serves as a flux of distinguishability.  Perfect bookkeeping demands $\partial_r \Phi(r)=0$.

\emph{Geometric tension.}  
In a flat disk the circumference grows as $2\pi r$; if events are neutral parcels, their surface density falls as $1/r$.  
Without correction, $\Phi(r)$ would decrease outward \cite{misner1973}.

\emph{Minimal fix.}  
Introduce a compensating connection with curvature $K(r)$ such that
\[
\partial_r\!\big(r\,\rho(r)\,v(r)\big)=0,
\]
where $\rho(r)$ is the local count density and $v(r)$ the tangential update rate.  
The least-bias (minimal-curvature) solution satisfying constant flux is
\[
r\,\rho(r)\,v(r)=C \quad\Rightarrow\quad v(r)\approx \text{const.}
\]
Thus a \emph{flat} tangential rate is the order-preserving closure of the bookkeeping rule \cite{jaynes1957, landauer1961}.

\emph{Interpretation.}  
Within this framework, a “flat rotation curve’’ is the informationally minimal configuration that maintains constant flux of distinguishable events as circumference grows.  
The compensating $K(r)$ is simply the residue of non-closure that enforces causal consistency across radii \cite{wheeler1983, bombelli1987}.

\emph{Scope.}  
This extrapolation is purely formal.  It demonstrates how a constant tangential rate can emerge from an invariant-count condition, not how galaxies move.
\end{example}

\section{Empirical Test: Normal Equations for Rotation Invariance}
\label{sec:normal-eq-rotation}

\textbf{N.B.} This section specifies a falsifiable data model for galaxy rotation curves derived from the
order-preserving transport condition.  It implements the single-parameter limit implied by Chapter 2, where the calculus of measurement allows only one free constant to curve-fit any closed causal system.  The model therefore has a unique intercept but no tunable slope parameters.

\paragraph{Invariant.}
Order-preserving transport with constant flux of distinguishability implies
\[
r\,\rho(r)\,v_\theta(r)=\Phi,\qquad \Phi>0\ \text{constant}.
\]
Here $\rho(r)$ represents the local count density (a proxy for informational concentration) and $v_\theta(r)$ the observed tangential speed.

\paragraph{Observables.}
Let $I(r)$ denote surface brightness and $\kappa>0$ a proportionality linking brightness to count density, $\rho(r)\propto \kappa I(r)$.  
Substituting gives the predicted profile
\[
v_\theta(r)=\frac{\Phi}{r\,\kappa\,I(r)}.
\]
Taking logarithms produces a linear empirical relation:
\[
\underbrace{\log v_\theta(r)}_{y(r)}=\underbrace{\theta_0}_{\log\Phi-\log\kappa}-\log r-\log I(r)+\varepsilon(r),
\]
where $\mathbb{E}[\varepsilon(r)]=0$ under the invariant.

\paragraph{Normal equation (strict test).}
Define $z_i:=\log r_i+\log I_i$ and $y_i:=\log v_i$.  
The strict model is $y_i=\theta_0-z_i+\varepsilon_i$ with a fixed slope of $-1$ on both $\log r$ and $\log I$.  
Only one constant, $\theta_0$, remains free in accordance with Chapter 2.  
The ordinary-least-squares normal equation reduces to
\[
\hat\theta_0=\frac{1}{n}\sum_{i=1}^n(y_i+z_i),
\]
and falsification is assessed by testing for systematic structure in residuals
$\hat\varepsilon_i=y_i-\hat\theta_0+z_i$.

\paragraph{Relaxed model with curvature residue.}
To capture permissible second-order deviations $K(r)$ due to geometric residue, extend to
\[
y_i=\theta_0-z_i+\sum_{j=1}^mB_j(r_i)w_j+\varepsilon_i,
\]
where $B_j$ are spline bases with penalty $\lambda\|Dw\|^2$.  
The penalized normal equations are
\[
\begin{bmatrix}
n & \mathbf{1}^\top B\\
B^\top \mathbf{1} & B^\top B+\lambda D^\top D
\end{bmatrix}
\begin{bmatrix}
\theta_0\\ w
\end{bmatrix}
=
\begin{bmatrix}
\sum_i(y_i+z_i)\\ B^\top(y+z)
\end{bmatrix}.
\]
When $w=0$ the invariant holds exactly; significant $w$ indicates structured departure.

\paragraph{Population form.}
Across galaxies $g$, intercepts $\theta_{0,g}$ vary but slopes remain fixed:
\[
y_{ig}=\theta_{0,g}-\log r_{ig}-\log I_{ig}+\varepsilon_{ig}.
\]
A mixed-effects regression with random intercepts but common slopes tests universality of the invariant; a slope differing from $-1$ falsifies it.

\paragraph{Falsifiable predictions.}
\begin{enumerate}
\item \textbf{Fixed slopes:} Coefficients on $\log r$ and $\log I$ are $-1$ within uncertainty; deviation falsifies the invariant.
\item \textbf{White residuals:} Adjusted residuals $\hat\varepsilon_i$ show no systematic radial trend.
\item \textbf{Cross-sample invariance:} Slopes are common to all galaxies; intercepts vary only by normalization.
\item \textbf{Low-brightness scaling:} Lower $I$ implies higher $v$ at fixed $r$; violation falsifies order-preserving transport.
\end{enumerate}

\paragraph{Scope.}
This regression embodies the Chapter 2 principle that a consistent measurement law introduces at most one free constant.  
Its acceptance or rejection is therefore directly falsifiable: systematic deviations in slope or curvature constitute empirical evidence against the order-preserving flux hypothesis.


	\section{Global Constraint as the Einstein Equation}
	\label{sec:global-constraint}

	The final step is to impose global consistency on the causal network.
	Local rules of separation and transport guarantee Martin consistency within
	each neighborhood, but finite observation requires that these neighborhoods
	overlap.  The residual curvature computed in the previous section measures
	the degree to which local order fails to close globally.  The Einstein
	equation expresses the condition under which that failure is exactly
	balanced by the stress encoded in the Causal Universe Tensor.

	\subsection{From Local Residue to Global Balance}

	Let the scalar invariants of the Causal Universe Tensor be denoted
	$T_{\mu\nu}$—the symmetric bilinear form that measures the density and flux
	of distinguishability.  The curvature invariants of the causal gauge are
	summarized by the Einstein tensor,
	\[
	G_{\mu\nu} = R_{\mu\nu} - \tfrac{1}{2} R g_{\mu\nu}.
	\]
	Both tensors share the same divergence-free property,
	$\nabla^{\mu}G_{\mu\nu} = \nabla^{\mu}T_{\mu\nu} = 0$, a differential
	expression of Martin consistency.  The only admissible global solution is
	therefore their proportional equality,
	\[
	G_{\mu\nu} = 8\pi\, T_{\mu\nu}.
	\]
	This is the Einstein field equation, reinterpreted as the global constraint
	that restores balance between the residue of inconsistency (curvature) and
	the finite structure of distinguishability (stress).

	\subsection{Interpretation in the Causal Framework}

	The Einstein equation states that curvature is not an independent source of
	force but the universe’s adjustment to maintain causal coherence.  Energy
	and stress arise from the finiteness of measurement; curvature arises from
	the impossibility of reconciling all such measurements globally.  The
	equation $G_{\mu\nu} = 8\pi T_{\mu\nu}$ enforces that these two forms of
	inconsistency—informational and geometric—cancel exactly.  When they do,
	the propagation of light remains Martin-consistent throughout the entire
	network.

	In this view, gravitation is the manifestation of the universe correcting
	its own bookkeeping of distinctions.  Mass–energy is simply the local
	density of finite observation, and curvature the global compensation that
	restores order.  Spacetime bends not because matter exerts force, but
	because causal consistency demands it.

	\subsection{The Closure of the Gauge of Light}

	The Einstein equation thus completes the gauge of light.  Beginning with
	the metric as the gauge of separation, the connection as the rule of causal
	transport, and curvature as the residue of inconsistency, the global
	constraint closes the system.  All four structures arise from a single
	requirement: that the scalar invariants of the Causal Universe Tensor
	remain self-consistent under extension to the entire causal domain.

	\begin{remark}
	In this formulation, general relativity is not a separate physical theory
	but the closure condition of the causal calculus.  The Einstein tensor is
	the final differential form of Martin consistency; the stress–energy tensor
	is the discrete record of finite distinction.  Their equality marks the
	point at which the universe’s description becomes self-consistent.  Beyond
	this, nothing remains to adjust—the gauge of light is complete.
	\end{remark}

