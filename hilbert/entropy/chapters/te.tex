
\begin{example}[Thought Experiment: The Knot-Tying Puzzle and Cubic Spline Closure]
\textbf{N.B.} This experiment visualizes why $U^{(4)} = 0$ is the natural limit of causal smoothness.

\emph{Setup.}  
Picture threading a shoelace through a lattice of eyelets representing discrete events.  
Each tie fixes a local order; each pull relates adjacent ties under reciprocity.  
If the path bends beyond third order, new tension points appear that violate global closure.

\emph{Demonstration.}  
Tying loops with 1st–3rd-order curvature yields smooth closure; a 4th-order “wiggle” over-constrains the path, leaving residual curvature.  
Hence the minimal global closure corresponds to cubic continuity: $U^{(4)} = 0$.

\emph{Interpretation.}  
The lattice stands for the causal tensor.  
Minimal curvature ensures bijective mapping of distinctions---no “loose ends.”  
The physical limit is the geodesic; the logical limit is consistency under finite observation.
\end{example}

\begin{example}[Thought Experiment: The Dual Transport of Measurement]
\textbf{N.B.} This thought experiment explores the formal coexistence of discrete and continuous transport under finite observation.  
It does not describe quantum dynamics or wave mechanics.  
It serves to illustrate how an informational system can exhibit both particle–like localization (discrete event selection) and wave–like propagation (distributed causal amplitude) within the same logical framework \cite{bohr1928,heisenberg1927,wheeler1983}.

\emph{Setup.}  
Consider an observer maintaining a causal record $\mathcal{R}$ of distinguishable events $\{e_i\}$.  
Each update corresponds to an informational transport operator $U_t$, which may act:
\begin{enumerate}
  \item \emph{Discretely}---selecting a new event and appending it to $\mathcal{R}$ (particle limit), or
  \item \emph{Continuously}---refining the relational amplitudes between existing events (wave limit).
\end{enumerate}
In a regime of finite observation, these two descriptions coexist:  
the discrete record $e_i$ is a sample of a continuous causal amplitude that itself evolves under adiabatic, annealing, or Brownian mechanisms.

\emph{Invariant to Preserve.}  
For each admissible transformation, the informational flux
\[
\Phi(t) = \sum_i \rho_i(t)\,v_i(t)
\]
remains conserved in expectation.  
Here $\rho_i$ is the probability weight of distinguishing $e_i$ at time $t$, and $v_i$ the causal update rate.  
Conservation of $\Phi(t)$ enforces complementarity: localization increases $\rho_i$ at the expense of amplitude spread (wave), while dispersion equalizes $\rho_i$ (particle uncertainty).  
The two limits are dual descriptions of the same informational invariant.

\emph{Formal Analogy.}  
Let $\Psi$ denote the normalized vector of causal amplitudes across events.  
Under reversible (adiabatic) transport,
\[
U_t\,\Psi = e^{iH t}\Psi,
\]
and under irreversible (annealing or Brownian) refinement,
\[
\Psi_{t+1} = (I - \epsilon L)\Psi_t,
\]
where $L$ is the informational Laplacian on the causal network.  
The “wave” view treats $\Psi$ as a continuous amplitude over possible distinctions;  
the “particle” view corresponds to a single index realization of $\Psi$ by the observer.  
Both are faithful encodings of the same causal process at different resolutions.

\emph{Interpretation.}  
Wave–particle duality thus arises as an epistemic phenomenon of finite observation:  
the wave is the uncollapsed ensemble of admissible distinctions,  
the particle the locally recorded choice among them.  
Observation projects $\Psi$ onto one branch of $\mathcal{R}$, satisfying the Event Selection Axiom but leaving the global informational flux invariant.  
In this sense, the collapse of the wave function is not physical, but the act of registering one consistent causal trajectory within the set of possible orderings.

\emph{Scope.}  
This thought experiment is purely formal.  
It demonstrates how the coexistence of discrete and continuous transport mechanisms follows from the structure of causal information, not from any underlying quantum field.  
Wave–particle duality here is the logical complementarity of refinement and selection---the informational shadow of the observer’s bandwidth limit.
\end{example}

%%%%%%%%%%%%%%%%%%%%%%%%%%%%%%%%


\begin{example}[Galileo's Free--Fall as the Flat--Space Limit of Causal Motion]
In Galileo's experiment, two spheres of unequal mass are dropped from the
same height and reach the ground simultaneously.  Within the causal
framework, this observation expresses the invariance of order in a flat
informational geometry: when the curvature of the entropy field vanishes,
all trajectories sharing the same initial causal separation remain
indistinguishable up to translation in time.

Let the causal paths be $\gamma_1(t)$ and $\gamma_2(t)$, each governed by
\[
\frac{d^2 x}{dt^2} = g,
\]
where $g$ is constant.  Because the informational curvature
$\nabla_i \nabla_j S$ is zero, the metric gauge $g_{ij}$ is uniform, and
the Reciprocity Law preserves equality of causal intervals:
\[
\delta^2 x_1 = \delta^2 x_2.
\]
Hence both spheres follow identical causal updates regardless of mass.

In this limit, the observer's partition $\mathcal{P}_n$ resolves all
relevant distinctions—position, time, and acceleration—so the reciprocity
mapping
\[
\Phi : V /{\sim_{\mathcal{P}_n}} \;\longleftrightarrow\;
M /{\sim_{\mathcal{P}_n}}
\]
is exact.  No refinement of the partition changes the outcome: the motion
is deterministic.  Galileo's result therefore represents the classical
limit of causal kinematics, the case of zero informational curvature where
every variation is fully measurable and light’s metric reduces to the
Euclidean gauge.
\end{example}


\begin{example}[Gravitational Lensing as Informational Curvature]
When light passes near a massive object, its trajectory bends—not because
space itself is a physical medium that deforms, but because the mapping that
preserves causal order becomes nonuniform.  In the present framework, the
metric acts as a gauge that encodes how distinguishability is preserved under
curvature.  Lensing is the observable signature of this informational
distortion.

Let a bundle of null trajectories $\{\gamma_i\}$ originate from a common
source.  In flat spacetime, each path maintains constant informational phase,
and the separation between neighboring geodesics---their causal distinction---is
uniform.  Introducing a local entropy gradient $S(x)$ modifies this gauge:
the effective distance between successive events changes by
\[
\delta ds^2 \propto \nabla_i \nabla_j S,
\]
so that the extremal path satisfies
\[
\delta \!\int ds = 0 \quad \Longrightarrow \quad
\frac{d^2 x^i}{d\lambda^2} + \Gamma^i_{jk}\frac{dx^j}{d\lambda}\frac{dx^k}{d\lambda} = 0,
\]
with $\Gamma^i_{jk}$ determined by the informational curvature $\partial_i
\partial_j S$.  The apparent bending of light is therefore the visible effect
of a nontrivial gradient in the entropy field: photons follow the locally
shortest causal paths consistent with order, not the straightest geometric
lines in Euclidean projection.

Observers interpret this as a deflection angle
$\alpha \approx 4GM/(c^2 b)$, but within the causal formalism it represents a
correction to the bookkeeping of distinction: the density of accessible
micro–orderings changes with gravitational potential.  Lensing thus measures
how informational curvature couples to geometry---a macroscopic manifestation
of the same reciprocity that defines the metric itself.
\end{example}

\begin{example}[The Three--Body Problem as Computational Reciprocity]
Consider three point masses $m_1,m_2,m_3$ interacting gravitationally with
positions $r_i(t)\in\mathbb{R}^3$.  Newtonian dynamics gives
\[
m_i\ddot r_i
  \;=\; G\sum_{j\neq i} \frac{m_i m_j}{\|r_j-r_i\|^3}\,(r_j-r_i),
  \qquad i=1,2,3.
\]
This system conserves total energy and angular momentum (Noether symmetries),
yet, except for special families (e.g.\ Euler and Lagrange configurations), it
admits no closed-form solution.  In the present framework, this means the
reciprocity map closes \emph{only computationally}: the admissible update that
preserves order and invariants exists, but it must be realized by an iterative,
order-preserving scheme.


Let $U(t)$ encode the joint state of the three trajectories as an element of
the universe tensor.  Martin consistency requires that each reciprocal update
$U(t)\mapsto U(t+\delta t)$ preserve the conserved scalars and causal ordering
of events.  Analytic spline closure ($U^{(4)}\!=0$) is insufficient here:
interactions couple the segments so that local cubic envelopes do not
globally commute.  The correct closure is \emph{algorithmic}: a reversible,
symplectic, order-preserving integrator (e.g.\ velocity Verlet/leapfrog) that
implements the reciprocity step without violating the invariants,
\[
\Phi_{\delta t}^{\text{symp}}:\; U(t)\longmapsto U(t+\delta t),\qquad
\delta E=0,\;\;\delta L=0 \;\text{(to integrator accuracy)}.
\]
Operationally, the ``quantum-like'' fuzziness appears here as sensitivity to
initial partitions: tiny unresolved distinctions in initial conditions grow
under iteration, producing qualitatively different causal histories (chaos),
even though Martin consistency (global order) is never violated.

Thus the three--body problem exemplifies a domain where physics \emph{requires}
computation: reciprocity and consistency still govern the update, but their
closure cannot be written in elementary functions.  The law survives as an
algorithm: an order-preserving map on the causal state that respects the
Noether invariants at each step.
\end{example}


