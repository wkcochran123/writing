\chapter{The Calculus of Dynamics}
\label{chap:dynamics}

\section{Emergent Dynamics}
\label{sec:emergent-dynamics}

The discrete Causal Universe Tensor defines a finite informational measure
on admissible configurations.  When these configurations are refined, any
replacement of an admissible history by one with additional unrecorded
structure is forbidden by Axiom~\ref{ax:boltzmann}.  As a consequence,
the physical notion of ``dynamics’’ is not an independent postulate: it
arises only as the unique continuous shadow of informational extremality.

\subsection{Weak Formulation on Space--Time}
\label{sec:weak-formulation}

Let $\psi$ denote an admissible configuration consistent with a fixed set of
event anchors, and let $\phi$ be any test configuration sharing those
anchors.  Informational minimality requires that replacing $\psi$ by $\phi$
cannot decrease causal consistency.  In the finite domain, this condition
appears as the weak relation
\begin{equation}
\psi^{\!*}\,\mathcal{L}\,\psi \;\approx\; \psi^{\!*}\,\mathcal{L}\,\phi,
\label{eq:weak-form}
\end{equation}
where $\mathcal{L}$ counts distinguishable causal increments and
$\psi^{\!*}$ is the reciprocity dual.  No derivatives are assumed; smooth
structure enters only as the completion of refinement.

\subsection{Reciprocity and the Adjoint Map}
\label{sec:reciprocity-adjoint}

The weak extremality relation~\eqref{eq:weak-form} compares an admissible
configuration $\psi$ against a test configuration $\phi$ that shares the same
event anchors.  In the discrete domain, replacing $\psi$ by $\phi$ is encoded
by a \emph{selected update} of the event record: only those increments that
alter distinguishable curvature are allowed.  This update is represented by
the Selection Operator $\mathcal{S}$ of
Definition~\ref{def:selection-operator}, which maps admissible configurations
to admissible refinements.

To every such operator there is an associated \emph{reciprocity map}
$\psi^{\!*}$, defined as the adjoint of $\mathcal{S}$ with respect to the
informational measure $\mathcal{L}$:
\[
\psi^{\!*}\,\mathcal{L}\,\phi
\;=\;
\mathcal{L}\,\bigl(\mathcal{S}[\psi],\,\phi\bigr),
\]
for all admissible test configurations $\phi$.  Intuitively, $\psi^{\!*}$
records the ``shadow'' of an update when viewed from the perspective of
informational minimality: any component of $\phi$ that would introduce an
unobserved distinction is suppressed by the adjoint action.

Because $\mathcal{S}$ admits only refinements that do not create hidden
structure, the reciprocity map annihilates variations invisible at the event
anchors.  Formally, if $\phi$ and $\psi$ agree at the anchors and differ only
by an undetectable perturbation, then
\[
\psi^{\!*}\,\mathcal{L}\,\phi
\;=\;
\psi^{\!*}\,\mathcal{L}\,\psi.
\]
This is exactly the weak relation~\eqref{eq:weak-form}.  In this sense,
$\psi^{\!*}$ enforces closure: it guarantees that the extremal configuration
carries no latent curvature that would be revealed under refinement.

The ``variation'' of $\psi$ is therefore not a differential operation but a
selected refinement of the causal record.  The reciprocity map is the dual
constraint that removes any component of that refinement which would violate
informational minimality.  Together, they generate the weak Euler--Lagrange
structure entirely within the discrete domain, without assuming
differentiability or a continuum of states.

In this way, the reciprocity map guarantees that any admissible update of
$\psi$ corresponds to an interpolant $f(\psi)$ that introduces no new
distinguishable structure.
Under refinement, every such interpolant
converges to the same smooth closure $\Psi$.  Because the event tensor
defines a finite labeled partition of the causal domain, $\Psi$ preserves
anchor order and is injective on each partition element.  Its inverse
$\Psi^{-1}$ therefore exists trivially: applying $\Psi$ and then
$\Psi^{-1}$ recovers exactly the original discrete record $\psi$.  The
interpolant and its smooth limit are thus informationally equivalent
representations of the same causal structure.
\begin{equation}
f(\psi) \rightarrow \Psi^{-1}.
\end{equation}


\subsection{Dense Limit and Euler--Lagrange Closure}
\label{sec:euler-lagrange-closure}

In the classical calculus of variations, a weak extremum of a smooth functional implies the Euler--Lagrange equation in strong form; see Courant
\cite{courant1943var} or Ciarlet \cite{ciarlet1978}.  There, differentiability is assumed a priori and the weak form is obtained by integrating by parts.

In the present framework no differentiability is assumed.  The weak relation \eqref{eq:weak-form} is defined entirely in the discrete domain,
where each term counts distinguishable causal increments.  When the causal grid is refined, informational minimality forces cubic continuity at each
event anchor.  In the dense limit, the discrete extremal coincides with the classical Euler--Lagrange closure,
\[
U^{(4)} = 0,
\]
but this appears only as the smooth completion of a countable sequence of refinements, not as an assumed differential equation.

\begin{proposition}[Discrete extremality implies Euler--Lagrange closure]
\label{prop:discrete-to-euler}
Let $U$ be an admissible configuration on a finite causal grid, and assume
it satisfies the weak extremality condition~\eqref{eq:weak-form} against all
test configurations $\phi$ sharing the same event anchors.  Then $U$ is a
piecewise cubic interpolant with global $C^2$ continuity.  In the dense
refinement limit of the causal grid, the discrete extremal satisfies
\[
  U^{(4)} = 0.
  \]
  Thus the classical Euler--Lagrange closure arises as the smooth completion of
  a countable sequence of informational refinements, not as an assumed
  differential equation.
  \end{proposition}

  \begin{proof}[Proof sketch]
  The argument proceeds in four steps.

  \emph{(1) Discrete admissible configurations.}
  On a finite causal grid, $U$ is specified by its values on a countable set of
  event anchors.  Between consecutive anchors, any interpolant with hidden
  curvature would imply additional distinguishable events.  Therefore the
  admissible interpolant on each interval is a polynomial of minimal degree,
  and $U$ is piecewise polynomial.

  \emph{(2) Weak extremality and hidden curvature.}
  The weak relation~\eqref{eq:weak-form} forbids any replacement of $U$ by a
  test configuration $\phi$ that reduces informational consistency.  A
  polynomial of degree greater than three contains latent inflection points
  that would be detected at sufficient refinement.  Thus the extremal is
  piecewise cubic.

  \emph{(3) Matching conditions at anchors.}
  Informational minimality disallows jumps in value, slope, or bending moment
  at an anchor, since each would constitute a new observable event.  Hence the
  piecewise cubic segments glue together with continuous value, first
  derivative, and second derivative.  The third derivative is piecewise
  constant.

  \emph{(4) Dense limit and smooth closure.}
  As the grid is refined, the intervals shrink and the piecewise constant third
  derivative converges to a continuous function.  The only continuous function
  whose integral vanishes on every shrinking interval is zero.  Therefore the
  fourth derivative of the limit vanishes and the discrete extremal satisfies
  $U^{(4)} = 0$.

  This recovers the classical Euler--Lagrange form without assuming
  differentiability: the differential equation is the smooth shadow of finite
  informational constraints.
  \end{proof}

\begin{example}[Repeatability of Invisible Motion~\cite{bacon1620}]
\label{ex:repeatability}
Consider two independent observers, $A$ and $B$, who record the motion of a
particle between the same event anchors $x_i \prec x_{i+1}$.  Each observer
has finite resolution: any acceleration or inflection large enough to be
distinguishable produces a new event.  Both refine their instruments until
no further events are detected on the interval.

If hidden curvature existed between the anchors, further refinement would
create additional distinguishable records.  The absence of such records
forces each observer to recover the same polynomial of minimal degree.  Thus
both obtain a cubic patch on the interval.

Now let $A$ and $B$ exchange data and perform a joint refinement on a finer
grid.  Any disagreement in value, slope, or bending moment at a shared
anchor would itself generate an observable event.  To avoid contradiction,
the cubic patches must glue together with continuous $U$, $U'$, and $U''$.
In the dense refinement limit, the piecewise constant third derivative
converges to a continuous function whose integral vanishes on every
shrinking interval, yielding
\[
U^{(4)} = 0.
\]

Thus repeatability demands the Euler--Lagrange closure: if two observers can
refine their measurements indefinitely without producing new events, their
reconstructions must converge to the same cubic extremal.  Smooth dynamics
are therefore the unique histories that leave no trace.
\end{example}




\subsubsection{Interpretation: No Physics Assumed}
\label{sec:no-physics-assumed}

The quantity $\mathcal{L}$ is not a physical density but an informational
measure of distinguishable curvature.  The ``action'' is a cumulative count
of admissible distinctions, and its extremal is the only configuration that
introduces no unobserved structure.  Classical dynamics therefore arise as a
mathematical consequence of measurement, not as a physical postulate.

\section{Galerkin Solutions to Weak Equations}
\label{sec:galerkin-solutions}

The weak extremality condition~\eqref{eq:weak-form} can be interpreted in the
classical language of Galerkin methods.  In this formulation, admissible
variations are taken from a finite trial space of functions that agree with
the event anchors.  For the curvature functional
\begin{equation}
  \mathcal{J}[U] = \int (U''(x))^2 \, dx,
  \label{eq:curvature-functional}
\end{equation}
the Galerkin weak form is
\begin{equation}
  \int U''(x)\,\phi''(x)\,dx = 0,
  \label{eq:galerkin-weak}
\end{equation}
for all admissible $\phi$.  Integrating~\eqref{eq:galerkin-weak} by parts twice
and using the fact that the variations vanish at the anchors eliminates all
boundary terms and produces the strong Euler--Lagrange equation
\begin{equation}
  \frac{d^2}{dx^2}\bigl(2U''(x)\bigr) = 0,
  \label{eq:strong-euler}
\end{equation}
and therefore
\begin{equation}
  U^{(4)}(x) = 0.
  \label{eq:u4-zero}
\end{equation}
Thus, the Galerkin extremals of the curvature functional are exactly the cubic
polynomials on each interval of the partition.  This establishes the classical
equivalence between weak variational solutions and their strong differential
form.

\subsection{Weierstrass Convergence of Galerkin Extremals}
\label{sec:weierstrass-galerkin}

Let $[a,b]$ be compact and let $\{\mathcal{T}_n\}$ be a sequence of nested
partitions with mesh size $h_n \to 0$.  For each $n$, let $U_n$ be the
Galerkin extremal of the curvature functional
\begin{equation}
  \mathcal{J}[U] \;=\; \int_a^b \bigl(U''(x)\bigr)^2\,dx,
  \label{eq:curvature-functional-weierstrass}
\end{equation}
subject to the anchor constraints induced by the event record on $\mathcal{T}_n$.

\begin{theorem}
\label{thm:weierstrass-galerkin}
There exists a unique smooth closure $\Psi$ satisfying
\begin{equation}
  \Psi^{(4)}(x) \;=\; 0 \quad \text{on each element of every }\mathcal{T}_n,
  \label{eq:psi-four-zero}
\end{equation}
and the sequence of Galerkin extremals $\{U_n\}$ converges to $\Psi$
uniformly on $[a,b]$:
\begin{equation}
  \lim_{n\to\infty}\;\sup_{x\in[a,b]} \bigl|U_n(x) - \Psi(x)\bigr| \;=\; 0.
  \label{eq:uniform-convergence}
\end{equation}
\end{theorem}

\begin{proof}[Proof sketch]
Energy monotonicity and nonnegativity imply $\{\mathcal{J}[U_n]\}$ is bounded,
so $\{U_n''\}$ is bounded in $L^2(a,b)$.  In one dimension, cubic--spline trial
spaces are stable and yield uniform bounds on $U_n$ and $U_n'$.  Hence
$\{U_n\}$ is equicontinuous and uniformly bounded on $[a,b]$.  By
Arzel\`a--Ascoli, there exists a uniformly convergent subsequence.
Any uniform limit $U$ satisfies the weak form on every subinterval, and
integrating by parts recovers the strong Euler--Lagrange condition
\begin{equation}
  U^{(4)}(x) = 0,
  \label{eq:u-four-zero}
\end{equation}
together with the anchor constraints.  Uniqueness of the smooth closure $\Psi$
forces $U = \Psi$, so the full sequence converges uniformly, establishing
\eqref{eq:uniform-convergence}.
\end{proof}

This establishes Weierstrass (uniform) convergence of the Galerkin extremals
to the unique cubic spline closure dictated by informational minimality.

In particular, any admissible interpolant $f(\psi)$ refined through nested
Galerkin spaces converges uniformly to the unique smooth closure $\Psi$, and
the inverse $\Psi^{-1}$ recovers exactly the original anchor data $\psi$.
Thus $f(\psi) \to \Psi$ and $\Psi^{-1} = \psi$.

\begin{corollary}
For any admissible interpolant $f(\psi)$ consistent with the event anchors,
the Galerkin refinements satisfy
\begin{equation}
  \lim_{n\to\infty} f_n(\psi) = \Psi,
\end{equation}
and $\Psi^{-1}$ restricted to the anchors recovers $\psi$ exactly.  Hence
$f(\psi) \to \Psi^{-1}$ in the sense of uniform convergence on $[a,b]$.
\end{corollary}


\section{Point-wise Agreement of the Galerkin Closure}
\label{sec:pointwise-galerkin}

Let $[a,b]$ be compact and let $\{\mathcal{T}_n\}$ be a sequence of nested
partitions with mesh size $h_n \to 0$.  For each $n$, let $U_n$ be the
Galerkin extremal of the curvature functional
\begin{equation}
  \mathcal{J}[U] \;=\; \int_a^b \bigl(U''(x)\bigr)^2\,dx,
  \label{eq:curvature-functional-pwa}
\end{equation}
subject to the anchor constraints induced by the event record on
$\mathcal{T}_n$.

\begin{theorem}[Uniform convergence of Galerkin extremals]
\label{thm:uniform-galerkin}
There exists a unique smooth closure $\Psi$ satisfying
\begin{equation}
  \Psi^{(4)}(x) = 0
  \qquad \text{on each element of every } \mathcal{T}_n,
  \label{eq:psi-four-zero-pwa}
\end{equation}
and the Galerkin sequence $\{U_n\}$ converges to $\Psi$ uniformly:
\begin{equation}
  \lim_{n\to\infty}\;
    \sup_{x\in[a,b]}
      \bigl|U_n(x) - \Psi(x)\bigr|
      \;=\; 0.
  \label{eq:uniform-convergence-pwa}
\end{equation}
\end{theorem}

\begin{proof}[Proof sketch]
Energy monotonicity and nonnegativity imply $\{\mathcal{J}[U_n]\}$ is
bounded, so $\{U_n''\}$ is bounded in $L^2(a,b)$.  One-dimensional spline
spaces provide uniform bounds on $U_n$ and $U_n'$, so $\{U_n\}$ is
equicontinuous and uniformly bounded.  By Arzel\`a--Ascoli, a uniformly
convergent subsequence exists.  Any uniform limit must satisfy the weak form
on every subinterval, and integrating by parts recovers the strong condition
\eqref{eq:psi-four-zero-pwa}.  Uniqueness of the cubic closure forces the
entire sequence to converge uniformly to $\Psi$.
\end{proof}

We now show that agreement at the event anchors forces agreement everywhere.

\begin{proposition}[Point-wise uniqueness]
\label{prop:pointwise-unique}
Let $\Psi$ and $\Phi$ be smooth closures of admissible interpolants of the
same event record $\psi$.  If
\begin{equation}
  \Psi(x_i) = \Phi(x_i)
  \qquad\text{for all event anchors } \{x_i\},
  \label{eq:agreement-anchors-pwa}
\end{equation}
then
\begin{equation}
  \Psi(x) = \Phi(x)
  \qquad\text{for all } x \in [a,b].
  \label{eq:agreement-global-pwa}
\end{equation}
\end{proposition}

\begin{proof}[Proof sketch]
On each interval $[x_i,x_{i+1}]$, both $\Psi$ and $\Phi$ satisfy
\eqref{eq:psi-four-zero-pwa} and are therefore cubic polynomials.  A cubic
is determined by its value, first derivative, and second derivative at a
point.  Informational minimality forbids discontinuities of $U$, $U'$, or
$U''$ at the anchors, so $\Psi$ and $\Phi$ match these quantities at every
$x_i$.  They must therefore coincide on each interval, giving
\eqref{eq:agreement-global-pwa}.
\end{proof}

Finally, since $\Psi$ is cubic on each interval, all derivatives above third
order vanish point-wise.

\begin{corollary}[Point-wise agreement of derivatives]
\label{cor:taylor-pointwise}
For every anchor $x_i$ and for every integer $k \ge 4$,
\begin{equation}
  \Psi^{(k)}(x_i) = 0,
  \label{eq:higher-derivatives-zero}
\end{equation}
and on each interval $[x_i,x_{i+1}]$,
\begin{equation}
  \Psi(x)
  \;=\;
  \Psi(x_i)
  + \Psi'(x_i)(x-x_i)
  + \tfrac{1}{2}\Psi''(x_i)(x-x_i)^2
  + \tfrac{1}{6}\Psi'''(x_i)(x-x_i)^3.
  \label{eq:cubic-taylor}
\end{equation}
Thus $\Psi$ and its Taylor expansion agree to all orders point-wise at each
anchor.
\end{corollary}

Together, \eqref{eq:uniform-convergence-pwa},
\eqref{eq:agreement-global-pwa}, and \eqref{eq:cubic-taylor} show that every
admissible interpolant converges to the same cubic closure $\Psi$, and that
$\Psi^{-1}$ recovers the original anchor data~$\psi$.

\subsection{$C^{2}$ and Piecewise Analytic Solutions}
\label{sec:c2-analytic}

The uniform convergence in
Section~\ref{sec:pointwise-galerkin} shows that the Galerkin sequence
$\{U_n\}$ has a unique smooth limit $\Psi$ satisfying $\Psi^{(4)}(x)=0$ on
each element of the partition.  Since a function with vanishing fourth
derivative is a cubic polynomial on that interval, the closure $\Psi$ is
given by
\begin{equation}
  \Psi(x)
  =
  \Psi(x_i)
  + \Psi'(x_i)(x-x_i)
  + \tfrac{1}{2}\Psi''(x_i)(x-x_i)^2
  + \tfrac{1}{6}\Psi'''(x_i)(x-x_i)^3
  \label{eq:cubic-local}
\end{equation}
on every subinterval $[x_i,x_{i+1}]$.

Because each such polynomial is infinitely differentiable on the open
interval, $\Psi$ is analytic on $(x_i,x_{i+1})$.  Moreover, cubic continuity
at the anchors enforces agreement of value, slope, and curvature across
interval boundaries, so $\Psi$ is globally $C^{2}$:
\begin{equation}
  \Psi \in C^{2}([a,b]).
  \label{eq:c2-global}
\end{equation}
Higher derivatives exist piecewise but need not be continuous across
anchors, and for all $k\ge 4$
\begin{equation}
  \Psi^{(k)}(x) = 0
  \qquad \text{for } x\in(x_i,x_{i+1}).
  \label{eq:higher-derivatives-zero-analytic}
\end{equation}

Thus the Galerkin limit is a classical $C^{2}$ solution of the strong
Euler--Lagrange form, and is analytic on each element of the partition.  No
additional smoothness or structure is present beyond the cubic representation,
and every admissible interpolant converges to this same piecewise analytic
closure.

\subsection{Equivalence of Discrete and Smooth Representations}
\label{sec:equivalence-principle}

Let $\psi$ denote an admissible discrete record supported on event anchors
$\{x_i\}$, and let $f(\psi)$ be any admissible interpolant that introduces no
new distinguishable structure between anchors.  Refining the interpolant over
nested partitions $\{\mathcal{T}_n\}$ produces a Galerkin sequence
$\{U_n\}$.  By Theorem~\ref{thm:uniform-galerkin},
\begin{equation}
  U_n \;\longrightarrow\; \Psi
  \quad\text{uniformly on } [a,b],
  \label{eq:fn-to-psi}
\end{equation}
where $\Psi$ satisfies the classical closure condition
\begin{equation}
  \Psi^{(4)}(x) = 0
  \quad\text{on each element of every } \mathcal{T}_n.
  \label{eq:psi-four-zero-equivalence}
\end{equation}

By Proposition~\ref{prop:pointwise-unique}, any two smooth closures that
agree at the anchors coincide point-wise on $[a,b]$.  Thus the limit
$\Psi$ is uniquely determined by the anchor data of $\psi$:
\begin{equation}
  \Psi(x_i) = \psi(x_i)
  \quad\text{for every anchor } x_i.
  \label{eq:psi-matches-psi}
\end{equation}

Since $\Psi$ is cubic on each interval, all higher-order derivatives vanish on
$(x_i,x_{i+1})$, and $\Psi$ is analytic there.  Cubic continuity across the
anchors implies
\begin{equation}
  \Psi \in C^{2}([a,b]).
  \label{eq:psi-c2-equivalence}
\end{equation}

Finally, $\Psi$ preserves the anchor ordering and is strictly monotone on each
interval.  Therefore its inverse exists on $[a,b]$ and, when restricted to the
anchor set, satisfies
\begin{equation}
  \Psi^{-1}(x_i) = x_i,
  \qquad
  \text{so }
  \Psi^{-1} = \psi
  \text{ on the event record}.
  \label{eq:psi-inverse-recovers-psi}
\end{equation}

Together, \eqref{eq:fn-to-psi}--\eqref{eq:psi-inverse-recovers-psi} show that
any admissible interpolant $f(\psi)$ refined by Galerkin methods converges to
the same $C^{2}$, piecewise analytic closure $\Psi$, and that $\Psi^{-1}$
recovers the original discrete record.  In this sense the discrete and smooth
representations are informationally equivalent: refinement introduces no
additional structure, and the smooth closure contains exactly the information
encoded in $\psi$.

\subsection{Recovery of the Euler--Lagrange Equation}
\label{sec:recovered-euler-lagrange}

The previous sections established that any admissible interpolant of the
event record, when refined over nested partitions, converges uniformly to a
unique $C^{2}$ closure $\Psi$.  On each element of the partition, this
closure satisfies
\begin{equation}
  \Psi^{(4)}(x) = 0,
  \label{eq:psi-u4-reveal}
\end{equation}
and is therefore cubic between anchors.

The corresponding weak extremality condition,
\begin{equation}
  \int \Psi''(x)\,\phi''(x)\,dx = 0
  \qquad\text{for all admissible }\phi,
  \label{eq:weak-var-reveal}
\end{equation}
was obtained directly from refinements of the discrete event record.
Integrating~\eqref{eq:weak-var-reveal} by parts twice eliminates boundary
contributions at the anchors and yields the strong Euler--Lagrange form
\begin{equation}
  \frac{d^2}{dx^2}\bigl(2\Psi''(x)\bigr) = 0
  \quad\Leftrightarrow\quad
  \Psi^{(4)}(x) = 0.
  \label{eq:strong-euler-reveal}
\end{equation}

No differentiability was assumed a priori; smoothness appears only after the
Galerkin limit exists and is unique.  Thus, from a discrete record and the
requirement that refinements introduce no new distinguishable structure, we
can safely infer the Euler--Lagrange equation.  The strong differential form
follows as a consequence of measurement and refinement, rather than as an
assumed property of the underlying system.

\section{Inference Without Assumption}
\label{sec:inference-without-assumption}

The analysis above does not derive the Euler--Lagrange equation from the
axioms of the discrete calculus of measurement.  Rather, it shows that the
Euler--Lagrange form can be inferred without violating those axioms.  From a
finite event record and admissible refinements, the Galerkin sequence
converges uniformly to a unique $C^{2}$ closure $\Psi$ that satisfies
\begin{equation}
  \Psi^{(4)}(x) = 0
  \qquad\text{on each element of the partition,}
  \label{eq:inference-u4}
\end{equation}
and therefore admits the classical weak relation
\begin{equation}
  \int \Psi''(x)\,\phi''(x)\,dx = 0
  \qquad\text{for all admissible }\phi.
  \label{eq:inference-weak}
\end{equation}
Integrating~\eqref{eq:inference-weak} by parts twice produces the strong
Euler--Lagrange form
\begin{equation}
  \frac{d^2}{dx^2}\bigl(2\Psi''(x)\bigr) = 0,
  \label{eq:inference-strong}
\end{equation}
with no additional assumptions.  Smoothness and differentiability arise only
after the Galerkin limit exists and is unique; neither is introduced as an
axiom of the theory.

Thus, the Euler--Lagrange equation is not postulated, but can be recovered
consistently from the refinement of discrete measurements.  The continuum is
compatible with the axioms: if one wishes to represent the discrete record by
a smooth function, the resulting closure obeys the classical variational
condition.  The framework allows the Euler--Lagrange equation to be inferred
from measurement, while remaining agnostic about any underlying differential
structure.

\section{The Free Parameter of the Cubic Spline}

A cubic spline appears, at first glance, to introduce three independent
degrees of freedom at each point: the value of the function, its slope, and
its curvature. In classical analysis these are treated as arbitrary boundary
data. In the present framework they have a different meaning: each arises as
a finite record of measurement.

Let $u(x)$ be the smooth limit of a countable refinement of finite
measurements. The discrete extremality principle established earlier shows
that any interpolant with unrecorded curvature violates informational
minimality. In the dense limit this forces

\begin{equation}
u^{(4)}(x) = 0.
\end{equation}

Every admissible smooth function is therefore locally a cubic polynomial. A
general solution to (2.?) takes the form

\begin{equation}
u(x) = a_0 + a_1 x + a_2 x^2 + a_3 x^3 .
\end{equation}

The coefficients $a_0, a_1, a_2, a_3$ are determined by three local
parameters: the value $u$, the slope $u'$, and the curvature $u''$. The
derivative of curvature,

\begin{equation}
u'''(x) = 6 a_3 ,
\end{equation}

is piecewise constant. This $u'''$ is the only quantity that may vary from
interval to interval without introducing fourth–order structure.

Locally this suggests three degrees of freedom. Globally this is not the
case. Because adjacent spline pieces must meet with no unrecorded
structure, they agree in value, slope, and curvature at every shared
boundary. All but one parameter are fixed by the causal structure of
measurement itself. The apparent local freedom collapses to a single global
scale factor: the magnitude of the consistent curvature field.

Thus the cubic spline does not introduce three physical constants; it
produces a single degree of freedom that expresses how strongly the record
of measurement bends. In this framework the continuum contains only one
free parameter: the global curvature scale that allows a countable record of
distinctions to be represented as a smooth function with no unrecorded
features. All subsequent quantities—second derivatives, wave speeds, stress
tensors, and curvature—are determined by this single parameter.



\section{Coda: Navier--Stokes as a Finite Third Parameter}
\addcontentsline{toc}{section}{Coda: Navier--Stokes as a Finite Third Parameter}

We do not derive the Navier--Stokes equations. Rather, we show how the
measurement calculus constrains any smooth limit of finite records to a
cubic-spline structure and thereby recasts the regularity question as the
finiteness of a single quantity: the third parameter of the spline.

\subsection*{1. Statement of the classical problem}
Let $v(x,t)$ be a velocity field and $p(x,t)$ a pressure satisfying the
incompressible Navier--Stokes system on $\mathbb{R}^3$ (or a smooth domain
with suitable boundary conditions):
\begin{equation}
\label{eq:NSE}
\partial_t v + (v\cdot\nabla)v + \nabla p = \nu \Delta v + f, 
\qquad \nabla\cdot v = 0,
\end{equation}
with smooth initial data $v_0$. The Millennium Problem asks whether smooth
solutions remain smooth for all time or may develop singularities in finite
time.

\subsection*{2. Measurement-to-spline reduction}
Chapter 2 established that admissible smooth limits of finite records obey
a local cubic constraint. Along any coordinate line (and likewise along any
admissible selection chain) each component admits a representation whose
fourth derivative vanishes in the limit:
\begin{equation}
\label{eq:quarticzero}
U^{(4)} = 0 \quad \text{(componentwise along admissible lines)}.
\end{equation}
Hence the only freely varying local quantity is the \emph{third parameter}
(the derivative of curvature). In one dimension this is $U'''$. In three
dimensions we package the idea as the third spatial derivatives of $v$:
\begin{equation}
\label{eq:thirdparam}
\Theta(x,t) := \nabla(\nabla^2 v)(x,t) \quad \text{(a third-derivative tensor)}.
\end{equation}
Informally: $v$, $\nabla v$, and $\nabla^2 v$ are glued continuously by the
spline closure; only $\Theta$ may vary piecewise without introducing
fourth-order structure.

\subsection*{3. Regularity as finiteness of the third parameter}
\begin{quote}
\emph{Principle.} If the third parameter $\Theta$ stays finite at all
scales allowed by measurement, the smooth spline limit persists and no
singularity can occur within the calculus of measurement.
\end{quote}
A practical surrogate is a scale-invariant boundedness criterion on $\Theta$
(or a closely related norm tied to enstrophy growth):
\begin{equation}
\label{eq:criterion}
\sup_{0\le t\le T}\ \|\Theta(\cdot,t)\|_{X} < \infty
\quad \Longrightarrow \quad \text{no blow-up on } [0,T],
\end{equation}
where $X$ is chosen to control the admissible refinements (e.g. an
$L^\infty$-type or Besov/H\"older proxy along selection chains). In words:
the only obstruction to global smoothness is unbounded third-parameter
amplitude.

\subsection*{4. Heuristic link to classical controls}
Energy and enstrophy inequalities control $\|v\|_{L^2}$ and $\|\nabla v\|_{L^2}$.
Vorticity $\omega=\nabla\times v$ monitors the first derivative. Growth of
$\nabla\omega$ involves $\nabla^2 v$; the \emph{onset} of non-smoothness is
therefore detected by $\Theta=\nabla(\nabla^2 v)$, the next rung. Thus the
finite-third-parameter condition \eqref{eq:criterion} plays the same role in
this framework that classical blow-up criteria play in PDE analyses: it is
the minimal spline-compatible guardrail against curvature concentration.

\subsection*{5. Non-classical dependency is not invoked}
No dependency (cause-effect) is asserted. The argument is purely
informational: as long as the admissible record does not force the third
parameter to diverge, the cubic-spline closure remains valid and the smooth
limit inferred earlier continues to apply.

\subsection*{6. The rephrased question}
\begin{quote}
\textbf{Navier--Stokes, reframed.} Given smooth initial data and forcing,
must the third parameter $\Theta$ in \eqref{eq:thirdparam} remain finite
for all time under \eqref{eq:NSE}? Equivalently, can measurement-consistent
refinement generate unbounded third-parameter amplitude in finite time?
\end{quote}
If $\Theta$ stays finite, the spline structure persists, and the calculus
of measurement supports global smoothness. If $\Theta$ diverges, the smooth
continuum description ceases to be representable as a limit of admissible
records, and the measurement calculus no longer licenses Euler--Lagrange
inference on that interval.

\subsection*{7. What we have and have not done}
We have not solved the Millennium Problem. We have shown that within this
program the obstruction to smoothness is concentrated in a single quantity,
the third parameter of the cubic spline representation. The classical
regularity question is thus equivalent, in this calculus, to the finiteness
of $\Theta$.



