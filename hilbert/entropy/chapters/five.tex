\chapter{The Quantum of Information}
\label{chap:kinematics-of-light}

The preceding chapters established that smooth motion appears as the
unique closure of causal order under refinement.  The Law of Spline
Sufficiency showed that any admissible continuous shadow must contain no
unrecorded structure and therefore satisfies the extremal condition
$\Psi^{(4)} = 0$.

In this chapter we examine the opposite extremum: the smallest admissible
refinement of the Causal Universe Tensor.  Such a refinement represents
the maximal rate at which distinguishability can propagate without
violating the Axiom of Planck.  We call this minimal, nonzero update an
\emph{informational quantum}.  It is not a physical particle or field;
it is the atomic refinement permitted by the axioms.

Along an extremal refinement of this form, the scalar invariants of the
Causal Universe Tensor remain stationary.  Any attempt to subdivide the
update would imply the existence of additional events, contradicting the
record.  Thus the geometry of extremal propagation follows entirely from
the consistency of order.

The continuous limit of this extremal behavior admits three structures:

\begin{itemize}
  \item A \emph{metric} $g_{\mu\nu}$, the bilinear gauge of distinguishable
  separation.

  \item A \emph{connection} $\Gamma^{\lambda}{}_{\mu\nu}$, the rule of causal
  transport required to propagate local labels while maintaining Martin
  consistency.

  \item A \emph{curvature tensor}, the residue of inconsistency that remains
  when the transport fails to close around a refinement loop.
\end{itemize}

No geometric or dynamical postulates are introduced.  Each structure
arises only as the smooth extension forced by the requirement that
extremal refinements remain globally coherent.  The kinematics of light
is therefore the kinematics of maximal distinguishability: the gauge
obtained when the rate of informational refinement saturates the causal
limit.

The remainder of this chapter develops these structures systematically.

\section{The Informational Bound $\epsilon$}
\label{sec:epsilon}

\NB{The refinement bound $\epsilon$ is not a physical quantum, particle, or
energy unit.  It is the minimal nonzero increment of distinguishable structure
that survives every admissible refinement of the measurement record.  Its
origin is purely informational: $\epsilon$ is the continuous shadow of the
residual $\mathcal{C}^2$ freedom in spline closure.  No physical ontology is
implied.}

The refinement of an observational record proceeds through countable additions
of distinguishable events.  As established in Chapter~3, the weak form of the
discrete bending functional admits a single free $\mathcal{C}^2$ parameter,
corresponding to the third derivative of the spline interpolant.  This degree
of freedom cannot be removed unless new measurements are recorded.  When no
additional events are observed, the residual freedom is irreducible, and the
continuous shadow must reflect a minimal, nonvanishing bound on further
curvature--level distinction.

We denote this invariant residual by $\epsilon$.  Any admissible refinement of
the continuous shadow must preserve $\epsilon$; to refine below this threshold
would introduce unrecorded structure and contradict the finite measurement
sequence.  Conversely, any refinement that preserves $\epsilon$ remains
consistent with the discrete data.  Thus $\epsilon$ functions as the
kinematic limit of refinement and provides the foundation for the emergent
invariant interval $\tau$, the metric $g_{\mu\nu}$, and the compatible
connection derived later in this chapter.

\subsection{Residual Spline Freedom and the Minimal Refinement Bound}
\label{sec:residual-spline}

The weak formulation developed in Chapter~3 shows that a sequence of discrete
measurements determines a unique cubic spline interpolant, except for a single
free parameter associated with the third derivative.  This residual freedom is
a consequence of the information minimality constraint: in the absence of
additional recorded events, the interpolant must not introduce structure that
was not observed.  A fourth--order correction would imply an unrecorded change
in curvature and is therefore inadmissible.

Let $\Psi$ denote the continuous shadow of the discrete observational chain.
The freedom in $\Psi'''$ represents the smallest degree of curvature that can
be varied without conflicting with the discrete record.  Its magnitude cannot
be reduced by any admissible refinement unless a new event occurs.  The
minimal nonzero shadow of this residual is the refinement bound $\epsilon$.

Formally, $\epsilon$ is the infimum of all curvature--level refinements that do
not violate the existing event sequence.  Any attempt to impose a refinement
of order smaller than $\epsilon$ would create new curvature that must be
supported by new events, contradicting the record.  Thus $\epsilon$ is the
continuous representation of the irreducible $\mathcal{C}^2$ residue.

\subsection{Maximal Informational Propagation}
\label{sec:maximal-informational}

An admissible refinement of the observational record adds distinguishable
structure without contradicting previously recorded events.  A path that
\emph{saturates} the refinement bound $\epsilon$ propagates information at the
maximal admissible rate: it incorporates all allowable distinction while
introducing no unrecorded curvature.

Such paths form the extremal curves of the informational geometry.  They are
defined not by physical principles, but by the logical requirement that
refinement cannot fall below the $\epsilon$ threshold.  Any further reduction
would imply hidden structure and is therefore inadmissible.

In the continuous shadow, these maximally propagated paths serve as the
reference curves for defining the invariant interval $\tau$.  Two observers who
refine the same extremal path must agree on the number of informational units
required to describe it; this count determines the causal interval and anchors
the construction of the metric in Section~5.2.

\begin{example}[The Compact Disc Reader and the Gauge of Separation]
\NB{This thought experiment does \emph{not} describe photons as informational
quanta. It is a finite conceptual model illustrating how a gauge of separation
emerges from the logic of distinguishability alone. No physical ontology is
implied.}

A compact disc stores information as a finite, ordered chain of distinctions.
Each pit or land corresponds to a single admissible event, and the reader
detects a new event only when the reflected signal exceeds its threshold of
discernibility. Everything below this threshold is invisible; it cannot enter
the admissible record. Thus the sequence of detections,
\[
e_1 \prec e_2 \prec e_3 \prec \cdots,
\]
encodes not only what \emph{was} observed, but the binding constraint that no
additional distinguishable structure may be inserted between these events.

From the standpoint of information, the read head defines a \emph{gauge of
minimal separation}: two surface configurations are “far enough apart’’ exactly
when the detector must refine its admissible description to distinguish them.
The metric is not assumed; it is inferred from the rule that only resolvable
differences may appear as refinements in the causal chain.

Now imagine two readers, A and B, scanning the same disc. Reader~A has a coarser
threshold; reader~B resolves finer distinctions. Each produces its own ordered
sequence of admissible events. Where B records additional refinements, A
records none. Yet when their records are merged, global coherence requires a
single history that preserves all recorded distinctions. The finer record forces
a refinement on the coarser: A must treat certain portions of the disc as
informationally extended, for failure to accommodate B’s distinctions would
render the merged history inconsistent.

In the dense limit, this refinement rule induces a continuous connection: the
shadow of the logical requirement that adjacent descriptions remain compatible
under transport. What appears in the smooth theory as a \emph{metric} is
nothing more than this bookkeeping of distinguishability: the minimal rule that
certifies when two states differ in a way that must be reconciled.

In this model, “light’’ corresponds not to a substance but to the maximal rate
at which new distinctions can be admitted without contradiction. Any attempt to
introduce refinements faster than this rate would violate global coherence.
Thus the invariant causal interval of Chapter~5 reflects the same constraint: an
observer may not admit distinctions faster than a globally coherent merge can
support.

The compact disc reader therefore offers a finite, concrete metaphor for the
emergence of the gauge of light, the metric as a rule of separation, and the
transport laws that follow from informational consistency.
\end{example}


\section{The Metric as a Gauge of Informational Separation}
\label{sec:metric}

\NB{The metric is not a physical fabric, field, or medium.  It is the
continuous shadow of the rule by which refinements preserve the invariant
causal interval.  Its function is purely informational: the metric enforces
that all admissible observers agree on the amount of distinguishable structure
between neighboring events.  No geometric ontology is assumed.}

The refinement bound $\epsilon$ defined in Section~\ref{sec:epsilon} limits how
finely an admissible history may be resolved without contradicting the
measurement record.  When this bound is propagated along an extremal path, the
result is a conserved quantity---the informational interval $\tau$---which
represents the number of $\epsilon$-sized refinements required to traverse that
path.  Any two observers refining the same extremal sequence must therefore
agree on the value of $\tau$.  This invariance is the kinematic foundation of
the metric.

Let $dx^\mu$ denote the local labels an observer assigns to successive events
along an extremal refinement.  Another observer, using a different admissible
convention for distinguishing the same events, assigns labels
$dx'^\mu = \Lambda^\mu_{\ \nu} dx^\nu$, where the transformation
$\Lambda^\mu_{\ \nu}$ preserves causal order as required by the Axiom of
Selection.  Although the two observers differ in the coordinates they assign,
they must agree on the number of $\epsilon$-sized refinements separating the
events; otherwise their merged histories would violate global consistency.

This requirement implies the existence of a bilinear form $g_{\mu\nu}$ such
that the scalar quantity
\[
\tau^2 = g_{\mu\nu}\, dx^\mu dx^\nu
\]
is invariant under all admissible changes of local labeling conventions.  The
metric is therefore the gauge of informational separation: it encodes how
distinctions between events are preserved when observers describe them using
different coordinate choices.

Because $\tau$ counts informational increments rather than geometric lengths,
the invariance condition
\[
g'_{\mu\nu}\, dx'^\mu dx'^\nu = g_{\mu\nu}\, dx^\mu dx^\nu
\]
expresses the deeper statement that the number of distinguishable refinements
between neighboring events is independent of the observer.  The metric
formalizes this invariance.  It is the bilinear bookkeeping rule ensuring that
every observer's coordinate description yields the same value of $\tau$ for any
given extremal path.

In this interpretation, the components of $g_{\mu\nu}$ do not prescribe a
geometry; they record the local comparison rule by which distinguishability is
maintained across observational frames.  Section~\ref{sec:law-causal-transport}
elevates this requirement to a formal law, showing that preservation of the
causal interval uniquely determines the compatible affine connection used to
propagate distinguishability under refinement.


\subsection{The Law of Causal Transport}
\label{sec:law-causal-transport}

\NB{The Law of Causal Transport is a kinematic statement.  It asserts only that
informational refinements must preserve the invariant interval $\tau$ defined
in Section~\ref{sec:metric}.  No dynamical interpretation of curvature or
stress is assumed here.  The law specifies how distinguishability must be
propagated under admissible changes of frame; all higher structures of
connection and curvature follow in later sections.}

The refinement bound $\epsilon$ defines the smallest admissible increment of
distinguishable structure.  When propagated along an extremal path, $\epsilon$
induces the invariant interval $\tau$, representing the total number of such
increments required to describe that path.  Because every observer must refine
the same underlying event sequence, the value of $\tau$ must remain unchanged
under all admissible relabelings.

This requirement leads to the following principle.

\begin{law}[Law of Causal Transport]
\label{law:causaltransport}
\emph{[Preservation of Distinguishability]}  
Any admissible refinement of an observational record must preserve the
informational interval $\tau$ between neighboring events.  In the continuous
shadow, this condition determines a unique bilinear form $g_{\mu\nu}$ and a
unique compatible rule of transport $\Gamma^{\lambda}_{\mu\nu}$ satisfying
\[
\nabla_{\lambda} g_{\mu\nu} = 0.
\]
The pair $(g_{\mu\nu}, \Gamma^{\lambda}_{\mu\nu})$ constitutes the metric gauge
of informational separation.
\end{law}

Because observers may assign different coordinates to the same infinitesimal
event displacement, we represent such a relabeling by
$dx'^{\mu} = \Lambda^{\mu}_{\ \nu}\, dx^{\nu}$, where $\Lambda^{\mu}_{\ \nu}$
preserves causal order.  The Law of Causal Transport requires that the
informational interval be invariant under this transformation:
\[
g'_{\mu\nu}\, dx'^{\mu} dx'^{\nu}
=
g_{\mu\nu}\, dx^{\mu} dx^{\nu}.
\]
This invariance elevates $g_{\mu\nu}$ from a mere bookkeeping device to a
constraint: it is the only bilinear form that guarantees all observers agree on
how many $\epsilon$-sized refinements separate neighboring events.

The law further implies that the comparison of nearby refinements must not
depend on the path taken in the space of coordinate labels.  This requirement
determines the connection coefficients $\Gamma^{\lambda}_{\mu\nu}$ as the
unique differential operators that preserve the metric gauge under change of
frame.

In this sense, the Law of Causal Transport encodes the most fundamental rule
of the kinematic structure: that distinguishability is preserved under motion.
The connection is not postulated, but forced by the need to maintain the
interval $\tau$ when an observer’s coordinate conventions vary from point to
point.  Section~\ref{sec:tau-invariance} elaborates the invariance of
$\tau$, and Section~\ref{sec:metric-bilinear} formalizes the role of
$g_{\mu\nu}$ as the bilinear form that preserves the $\epsilon$--refinement
count.


\subsection{Invariance of the Informational Interval $\tau$}
\label{sec:tau-invariance}

\NB{The interval $\tau$ is not a geometric length or a physical duration.  It is
the continuous shadow of an event count: the number of $\epsilon$-sized
refinements required to describe an extremal segment of an observational
record.  Its invariance expresses only that all admissible observers must agree
on the amount of distinguishable structure between neighboring events.}

The refinement bound $\epsilon$ defines the smallest admissible increment of
distinguishability.  When propagated along an extremal path---one that
saturates the refinement bound---each observer records the same number of
$\epsilon$-increments.  This count defines the informational interval $\tau$.
Because $\tau$ represents the number of admissible refinements rather than a
metric distance, its invariance follows from the requirement that no observer
may introduce or remove distinguishable structure that is not supported by the
measurement record.

Let $dx^{\mu}$ and $dx'^{\mu}$ denote the infinitesimal labels assigned by two
admissible observers to the same pair of neighboring events.  Their coordinate
labels differ by a transformation
\[
dx'^{\mu} = \Lambda^{\mu}_{\ \nu}\, dx^{\nu},
\]
where $\Lambda^{\mu}_{\ \nu}$ preserves causal order in the sense of the Axiom
of Selection.  Although the observers assign different coordinates, they must
agree on the number of $\epsilon$-increments between the events; otherwise
their merged histories would violate global consistency.

This agreement is enforced by a bilinear form $g_{\mu\nu}$ satisfying
\[
\tau^2 = g_{\mu\nu}\, dx^{\mu} dx^{\nu}.
\]
Under the coordinate transformation, the metric transforms as
\[
g'_{\mu\nu} = \Lambda^{\alpha}_{\ \mu}\, 
             \Lambda^{\beta}_{\ \nu}\, g_{\alpha\beta}.
\]
Substituting the transformed variables into the definition of $\tau$ yields
\[
\tau'^2 = g'_{\mu\nu}\, dx'^{\mu} dx'^{\nu}
        = g_{\alpha\beta}\, dx^{\alpha} dx^{\beta}
        = \tau^2.
\]

The invariance of $\tau$ thus expresses a simple but fundamental principle:
every admissible observer must assign the same number of distinguishable
increments to an extremal path.  Their coordinate descriptions may vary, but
the informational content of the path does not.

This invariance is the basis of the metric gauge introduced in
Section~\ref{sec:metric}.  It ensures that $\tau$ may serve as the universal
measure of informational separation, independent of the observer’s local
labeling conventions.  Section~\ref{sec:metric-bilinear} develops the metric
$g_{\mu\nu}$ as the bilinear form that enforces this invariance in the
continuous shadow.


\subsection{$g_{\mu\nu}$ as the Bilinear Form Preserving the $\epsilon$--Refinement Count}
\label{sec:metric-bilinear}

\NB{The metric $g_{\mu\nu}$ is not a geometric field on a manifold.  It is the
continuous shadow of the rule ensuring that all admissible observers preserve
the same count of $\epsilon$--sized refinements between neighboring events.
The components of $g_{\mu\nu}$ do not describe a physical medium or curvature;
they encode the invariant comparison rule required by informational
consistency.}

The interval $\tau$ defined in Section~\ref{sec:tau-invariance} expresses
the number of $\epsilon$--refinements along an extremal segment of the
measurement record.  Since this number must remain invariant under all
admissible relabelings of events, there must exist a bilinear form
$g_{\mu\nu}$ such that
\[
\tau^{2} = g_{\mu\nu}\, dx^{\mu} dx^{\nu}
\]
holds for every observer.  This expression is not a postulate but the unique
structure that enforces the preservation of $\tau$ under coordinate changes
that respect causal order.

To see this, consider two observers who assign infinitesimal labels
$dx^{\mu}$ and $dx'^{\mu} = \Lambda^{\mu}_{\ \nu}\, dx^{\nu}$ to the same pair of
neighboring events.  The Law of Causal Transport requires
\[
\tau'^2 = \tau^2,
\]
so we must have
\[
g'_{\mu\nu}\, dx'^{\mu} dx'^{\nu}
=
g_{\mu\nu}\, dx^{\mu} dx^{\nu}.
\]
Substituting $dx'^{\mu}$ and requiring equality for all admissible
transformations $\Lambda^{\mu}_{\ \nu}$ yields the transformation rule
\[
g'_{\mu\nu}
=
\Lambda^{\alpha}_{\ \mu}\,
\Lambda^{\beta}_{\ \nu}\,
g_{\alpha\beta}.
\]
Thus the metric is exactly the object that ensures agreement on the number of
$\epsilon$--sized refinements between neighboring events, regardless of the
coordinates used to describe them.

In this informational framework, $g_{\mu\nu}$ plays a role analogous to that of
a gauge potential: it specifies how infinitesimal refinements are compared
locally so that the global invariant $\tau$ remains unchanged.  The metric does
not specify “distance’’ in any geometric or physical sense; it enforces the
equivalence of all admissible measurement conventions.

Once $g_{\mu\nu}$ is introduced, the need to propagate these comparison rules
from point to point forces a unique notion of compatibility.  This requirement
determines the affine connection in Section~\ref{sec:connection} through the
condition
\[
\nabla_{\lambda} g_{\mu\nu} = 0,
\]
which expresses that the metric gauge is preserved under refinement and
transport.  The next section illustrates this invariance with a concrete
thought experiment.

\begin{example}[Thought Experiment: Michelson--Morley as Gauge Isotropy]
\label{sec:michelson-morley}

\NB{This thought experiment does not appeal to optical physics, wave
interference, or the existence of a medium.  It is a finite informational model
illustrating that the metric gauge must assign the same refinement cost
$\epsilon$ to extremal paths in all admissible directions.  No physical claims
about light or propagation are implied.}

Consider an observer attempting to refine two extremal segments of equal
informational content, but aligned in different coordinate directions.  Let
$dx^{\mu}$ and $dy^{\mu}$ denote the local labels assigned to the two
segments.  Each segment is chosen such that its refinement requires the same
number of $\epsilon$--increments when described in the observer's own frame.

Now suppose the observer rotates their coordinate system.  After rotation, the
new labels are $dx'^{\mu} = \Lambda^{\mu}_{\ \nu} dx^{\nu}$ and
$dy'^{\mu} = \Lambda^{\mu}_{\ \nu} dy^{\nu}$.  The rotation
$\Lambda^{\mu}_{\ \nu}$ preserves causal order, so it is an admissible
transformation.  The question is whether the observer must still assign the
same informational interval $\tau$ to both segments after the rotation.

The Law of Causal Transport requires that the $\epsilon$--refinement counts for
both segments remain invariant:
\[
\tau_x^{2}
=
g_{\mu\nu}\, dx^{\mu} dx^{\nu},
\qquad
\tau_y^{2}
=
g_{\mu\nu}\, dy^{\mu} dy^{\nu}.
\]
After rotation, the transformed intervals are
\[
\tau_x'^{\,2}
=
g'_{\mu\nu}\, dx'^{\mu} dx'^{\nu},
\qquad
\tau_y'^{\,2}
=
g'_{\mu\nu}\, dy'^{\mu} dy'^{\nu}.
\]
Substituting the transformation rules for $dx'^{\mu}$, $dy'^{\mu}$, and
$g'_{\mu\nu}$ gives
\[
\tau_x'^{\,2}
=
g_{\alpha\beta}\, dx^{\alpha} dx^{\beta}
=
\tau_x^{2},
\qquad
\tau_y'^{\,2}
=
g_{\alpha\beta}\, dy^{\alpha} dy^{\beta}
=
\tau_y^{2}.
\]

Thus the observer must continue to assign the same informational interval to
the two extremal segments under any admissible rotation.  There is no freedom
to deform the refinement counts directionally: doing so would imply that
$\epsilon$--sized increments depend on orientation and would violate the
requirement that informational refinement be globally coherent.

This invariance is the informational analogue of isotropy.  It expresses that
the metric gauge $g_{\mu\nu}$ must refine extremal paths uniformly in all
directions: the number of $\epsilon$--increments needed to resolve a segment of
given informational content cannot depend on the coordinate orientation.

The Michelson--Morley experiment is therefore understood here not as a test of
a physical medium, but as a finite illustration of the isotropy of the metric
gauge.  The invariance of $\tau$ under rotations forces $g_{\mu\nu}$ to encode
a direction--independent refinement rule.  Section~\ref{sec:connection} develops
the compatible connection that propagates this rule under changes of frame.
\end{example}

\section{The Connection as Informational Bookkeeping}
\label{sec:connection}

\NB{The affine connection $\Gamma^{\lambda}_{\mu\nu}$ is not a force field or a
physical interaction.  It is the continuous shadow of an informational rule:
the minimal differential adjustment required to preserve the metric gauge
$g_{\mu\nu}$ as an observer moves from one event to its neighbor.  Its role is
purely kinematic.  The connection records how local measurement conventions
must tilt to maintain the invariant interval $\tau$; no dynamical content or
geometric ontology is assumed.}

The metric $g_{\mu\nu}$, introduced in Section~\ref{sec:metric}, guarantees that
all admissible observers assign the same informational interval $\tau$ to an
extremal displacement at a single event.  This invariance is enforced by the
bilinear form $g_{\mu\nu}\, dx^{\mu} dx^{\nu}$, which preserves the
$\epsilon$--refinement count under changes of coordinates.  However, the metric
by itself does not specify how these comparison rules extend from one event to
its neighbors.  To describe how distinguishability is maintained along a path,
we require a differential notion of consistency.

The connection $\Gamma^{\lambda}_{\mu\nu}$ provides this rule.  It specifies how
tensor components must be adjusted when an observer translates a local
measurement convention from one event to an infinitesimally adjacent one.  In
particular, the connection determines the covariant derivative, which measures
change in a way that respects the metric gauge.  Imposing that the metric
remain invariant under such differential updates leads to the condition
$\nabla_{\lambda} g_{\mu\nu} = 0$, known as covariant constancy of the metric.

In the informational picture, this condition is the statement that the act of
refinement may not create or destroy distinguishable structure as an observer
moves through the network of events.  The connection is the unique differential
bookkeeping device that satisfies this constraint.  When the metric is uniform,
the connection vanishes and no adjustment is needed: straight paths remain
informationally straight.  When the metric varies, a nonzero connection encodes
how local gauges must be rotated and rescaled so that scalar quantities built
from $g_{\mu\nu}$ remain unchanged.

The remainder of this section develops the connection as the compatibility
condition implied by covariant constancy of the metric and interprets parallel
transport as the differential expression of Martin consistency.  In this way,
the Law of Causal Transport acquires its full kinematic content: it is the rule
that propagates the gauge of separation through the continuous shadow of the
Causal Universe Tensor.


\subsection{Covariant Constancy and the Compatibility Condition}
\label{ec:covariant-constancy}

\NB{Covariant constancy is not a physical conservation law.  It is the
informational requirement that the metric gauge $g_{\mu\nu}$, which preserves
the $\epsilon$--refinement count at a single event, must continue to preserve
that count as the observer moves to a neighboring event.  The affine connection
$\Gamma^{\lambda}_{\mu\nu}$ is therefore not introduced by assumption; it is
forced by the requirement that informational invariants remain invariant under
differential refinement.}

The metric $g_{\mu\nu}$ ensures that all admissible observers agree on the
informational interval $\tau$ at a point.  But as the observer moves from an
event $x$ to a nearby event $x + dx$, the local coordinate basis changes.
Under such a shift, the numerical components of $g_{\mu\nu}$ may appear to
change due to the alteration in basis, even if the underlying structure of
distinguishability remains the same.  To prevent this apparent change from
contaminating the informational interval, the transformation of $g_{\mu\nu}$
must be corrected by an additional adjustment term.

This correction is encoded by the covariant derivative.  The condition that the
metric gauge remain invariant under differential displacement is expressed as
\[
\nabla_{\lambda} g_{\mu\nu}
=
\partial_{\lambda} g_{\mu\nu}
-
\Gamma^{\sigma}_{\mu\lambda} g_{\sigma\nu}
-
\Gamma^{\sigma}_{\nu\lambda} g_{\mu\sigma}
=
0.
\]
The partial derivative $\partial_{\lambda} g_{\mu\nu}$ captures how the metric
components vary when written in the shifted coordinate system.  The remaining
terms subtract off this apparent variation by compensating for the tilt and
scale of the basis vectors themselves.  The equation $\nabla_{\lambda} g_{\mu\nu} = 0$
thus expresses the requirement that the informational interval $\tau$ remain
unchanged under any infinitesimal update of the observational coordinates.

This compatibility condition uniquely determines the connection when torsion is
absent.  As established in Chapter~3, the spline representation of admissible
histories carries no fourth--order freedom and is therefore torsion-free.  Under
this constraint, the metric compatibility condition fixes the connection to be
the Levi--Civita connection:
\[
\Gamma^{\lambda}_{\mu\nu}
=
\frac{1}{2}\,
g^{\lambda\sigma}
\left(
\partial_{\mu} g_{\nu\sigma}
+
\partial_{\nu} g_{\mu\sigma}
-
\partial_{\sigma} g_{\mu\nu}
\right).
\]
This expression is not a postulate; it is the only operator that ensures the
metric gauge remains intact under transport.  It is the continuous shadow of
the discrete requirement that refinement cannot introduce or eliminate
distinguishable structure beyond the $\epsilon$ bound.

With the connection now fixed by kinematic necessity, we may interpret its role
operationally.  The connection coefficients specify the adjustments required to
compare tensorial quantities at neighboring events, ensuring that the
informational interval $\tau$ and the refinement bound $\epsilon$ remain
consistent throughout the observer’s path.  The next subsection formalizes this
process as parallel transport.


\subsection{Parallel Transport as Differential Martin Consistency}
\label{sec:parallel-transport}

\NB{Parallel transport is not a physical motion of a vector through space.  It
is the informational requirement that the meaning of a direction---a rule for
distinguishing one infinitesimal refinement from another---remain consistent as
an observer updates coordinates from one event to the next.  In this framework,
a “vector’’ is an instruction for refinement, and parallel transport ensures
that such instructions are not distorted by changes in local labeling
conventions.}

The metric compatibility condition $\nabla_{\lambda} g_{\mu\nu} = 0$ determines
how the metric must be preserved under infinitesimal displacement.  Parallel
transport extends this requirement to all tensorial quantities, ensuring that
any object used to encode refinements of the observational record is carried
through the continuous shadow without introducing contradictions.

Let $V^{\mu}$ represent such a refinement direction.  When an observer moves
along a curve $x^{\mu}(s)$ in the event network, the numerical components of
$V^{\mu}$ change because the local coordinate basis changes.  The naive
derivative $d V^{\mu} / ds$ therefore incorporates both the intrinsic change in
the refinement direction and the apparent change induced by the shifting
coordinates.  To isolate the intrinsic change---the part that affects
distinguishability---we must subtract off the bookkeeping contribution provided
by the connection.

The covariant derivative along the path is thus defined as
\[
\frac{D V^{\mu}}{D s}
=
\frac{d V^{\mu}}{d s}
+
\Gamma^{\mu}_{\ \nu\lambda}\,
V^{\nu}\,
\frac{d x^{\lambda}}{d s}.
\]
Parallel transport requires that the intrinsic change vanish:
\[
\frac{D V^{\mu}}{D s}
=
0.
\]
This equation expresses the differential form of Martin consistency.  It states
that the instruction encoded by $V^{\mu}$ must retain its informational meaning
as the observer moves.  All coordinate-induced distortions of $V^{\mu}$ must be
canceled by the corresponding connection terms, ensuring that the refinement
direction does not acquire unrecorded structure.

The geometric interpretation of parallel transport as preserving “straightness’’
is replaced here by a purely informational one: parallel transport guarantees
that refinement instructions remain compatible with the metric gauge
$g_{\mu\nu}$ throughout the observer’s path.  Whenever the metric varies from
event to event, the connection coefficients encode the cost of adjusting the
observer’s basis to ensure that scalar comparisons built from $g_{\mu\nu}$ and
$V^{\mu}$ remain invariant.

In regions where $g_{\mu\nu}$ is uniform, the connection vanishes and the
informational meaning of $V^{\mu}$ is preserved without adjustment.  Where
$g_{\mu\nu}$ varies, nonzero connection coefficients encode the minimal
bookkeeping needed to keep the refinement count consistent.  This adjustment is
the kinematic origin of effects such as frequency shifts between observers in
different informational environments, which we examine in the following
subsection.

\section{Observable Consequence: Refinement--Adjusted Transport}
\label{sec:refinement-transport}

\NB{The frequency shift examined in this section is not a postulated effect.
It is the kinematic consequence of maintaining the invariant informational
interval $\tau$ across regions in which the metric gauge $g_{\mu\nu}$ varies.
No physical ontology is assumed.  The observable change in clock rates reflects
the differential bookkeeping enforced by the connection $\Gamma^{\lambda}_{\mu\nu}$.}

The previous sections established the chain of informational structure:
the refinement bound $\epsilon$ fixes the local increment of distinguishable
structure; the metric $g_{\mu\nu}$ expresses how these increments are compared
between observers; and the connection $\Gamma^{\lambda}_{\mu\nu}$ preserves this
comparison under differential displacement.  When the metric varies from one
location to another, this preservation requires that the local rate of event
counting---the clock frequency---adjusts so that the invariant interval remains
consistent across observers.

This section derives that adjustment and exhibits its observable consequence.

\subsection{The Invariant Causal Tally}
\label{sec:causal-tally}

\NB{An atomic clock does not measure a geometric length or a physical time.
It measures a count of distinguishable events.  The proper interval $\tau$ is
the continuous shadow of this count, expressed in units of the refinement bound
$\epsilon$.}

Consider an observer whose worldline is described by coordinates $(t, x^{i})$.
If the observer is at rest in their coordinate system ($dx^{i} = 0$), the
informational interval between neighboring events satisfies
\[
d\tau^{2} = g_{00}(x)\, dt^{2}.
\]
Thus the locally measured period of the clock is
\[
d\tau = \sqrt{g_{00}(x)}\, dt.
\]
Because $\tau$ counts $\epsilon$--sized refinements, the local clock frequency
$\nu(x)$ is inversely proportional to the size of this interval:
\[
\nu(x) = \frac{1}{d\tau}
        = \frac{1}{\sqrt{g_{00}(x)}}\,\frac{1}{dt}.
\]

Two observers at rest in different metric gauges therefore experience different
informational intervals for the same coordinate increment $dt$.  The
relationship between their locally recorded counts is fixed entirely by the
metric gauge.

\subsection{Derivation of Frequency Adjustment}
\label{sec:frequency-adjustment}

\NB{The global parameter $t$ is not a physical time.  It is the auxiliary
labeling parameter that all admissible observers must agree upon when their
records are merged.  Its increments must match across observers in order for
their $\epsilon$--counts to be reconciled.}

Let observers $A$ and $B$ be stationary in regions with metric components
$g_{00}(A)$ and $g_{00}(B)$.  Over a shared coordinate increment $\Delta t$,
their locally recorded proper intervals are
\[
\Delta \tau_{A} = \sqrt{g_{00}(A)}\, \Delta t,
\qquad
\Delta \tau_{B} = \sqrt{g_{00}(B)}\, \Delta t.
\]
Since a clock’s frequency is the inverse of the proper interval it records,
\[
\nu_{A}
=
\frac{1}{\Delta \tau_{A}}
=
\frac{1}{\sqrt{g_{00}(A)}}\,\frac{1}{\Delta t},
\qquad
\nu_{B}
=
\frac{1}{\sqrt{g_{00}(B)}}\,\frac{1}{\Delta t}.
\]

The ratio of their observed frequencies is therefore
\[
\frac{\nu_{A}}{\nu_{B}}
=
\frac{\sqrt{g_{00}(B)}}{\sqrt{g_{00}(A)}}.
\]
This expression is the kinematic consequence of the Law of Causal Transport.
When $g_{00}$ varies, the connection $\Gamma^{0}_{00}$ compensates by adjusting
the local rate of $\epsilon$--counting so that the merged observational record
remains consistent.  The observed frequency shift is thus the operational
signature of nonzero connection coefficients.

\section{Conclusion: The Kinematic Foundation of Geometry}
\label{sec:kinematic-closure}

\NB{This chapter derived the continuous kinematic structures---the metric
$g_{\mu\nu}$ and the connection $\Gamma^{\lambda}_{\mu\nu}$---from the
informational requirement that refinements remain globally consistent.  No
forces, fields, or dynamical assumptions were introduced.}

The development of this chapter followed the informational chain of emergence:
\[
\epsilon
\;\longrightarrow\;
\tau
\;\longrightarrow\;
g_{\mu\nu}
\;\longrightarrow\;
\Gamma^{\lambda}_{\mu\nu}.
\]
The refinement bound $\epsilon$ fixed the minimal increment of admissible
structure.  The interval $\tau$ encoded the invariant tally of such increments.
The metric $g_{\mu\nu}$ enforced this invariance across observers, and the
connection $\Gamma^{\lambda}_{\mu\nu}$ preserved it under differential
refinement.  The observable consequence of this structure is the redshift
effect, where nonuniformity of the metric gauge requires a corresponding
adjustment of the local $\epsilon$--counting rate.

This completes the kinematic description of informational geometry.  The next
chapter introduces the dynamic concept of curvature, defined as the obstruction
to transporting refinement instructions consistently around a closed loop.  In
this way, the ``Curvature of Information'' becomes the natural extension of the
kinematic structures developed here.




