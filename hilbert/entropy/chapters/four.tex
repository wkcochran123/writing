\chapter{The Kinematics of Light}
\label{chap:kinematics-of-light}

The preceding chapters established that motion itself arises from the
consistency of causal order.  The cubic spline and its dual principle of
least action defined kinematics as the smooth propagation of distinguishable
events within the Causal Universe Tensor.  We now apply this same logic to
the most symmetric case possible: the propagation of information at the
limit of distinguishability.  In that limit, the kinematics of the universe
\emph{is} the kinematics of light.

Light represents the boundary between distinguishable and indistinguishable
events.  Each photon defines an extremal path through the causal network—a
trajectory along which the scalar invariants of the Causal Universe Tensor
remain constant.  Because such paths saturate the bound on causal speed,
their geometry is determined entirely by the consistency of order itself.
Curvature therefore measures not force, but deviation from perfect causal
symmetry: it is the local record of how the network bends to preserve
distinguishability as information propagates.

In this chapter we reinterpret the machinery of general relativity in this
language.  The metric tensor $g_{\mu\nu}$ appears as the continuous shadow of
pairwise event separations; the connection coefficients $\Gamma^\lambda_{\mu\nu}$
encode how those separations adjust to maintain Martin consistency across
overlapping causal neighborhoods.  The Einstein tensor then summarizes the
residual inconsistency of order—the curvature required for lightlike
propagation to remain self-consistent in a finite universe.

Thus general relativity emerges here not as a theory of gravitational force,
but as the \emph{kinematics of light}: the unique geometry in which the
scalar invariants of the Causal Universe Tensor remain stationary along all
null directions.  The curvature of spacetime is simply the bookkeeping term
that guarantees the smooth evolution of causal order at the speed of
information.  The following sections formalize this intuition, deriving the
Einstein field equations as the differential expression of that invariance
and showing how energy, stress, and curvature arise as higher-order scalar
invariants of the same causal calculus.

\subsection{Consequences and Outlook}
\label{subsec:consequences-outlook}

At this point, nothing unfamiliar has been assumed.  Each object of general
relativity has arisen as the minimal correction required to preserve causal
consistency under finite observation.  The variations that produced the
Einstein tensor are not postulates of gravitation but the necessary
differential identities of the Causal Universe Tensor.  Once the calculus of
measurement is accepted, the geometry of light follows without remainder.

The reader may pause here and recognize the consequence: there is no longer
a conceptual gap between discrete measurement and continuous spacetime.
Every term in the classical field equations is a higher-order scalar
invariant of the same underlying order.  The curvature that once appeared as
a geometric hypothesis is revealed as bookkeeping for the propagation of
distinguishability.  There is, so far as the argument stands, no reason it
should not work.

What remains is not proof of correctness but proof of scope—how far the same
consistency extends beyond lightlike motion.  The next chapter therefore
turns from the kinematics of light to the dynamics of matter, asking how the
gauge of causal order constrains systems that deviate from the null limit.

\subsection{Arc of the Proof}
\label{subsec:arc-of-proof}

The goal of this chapter is to show that the geometry of spacetime arises
as the unique gauge condition under which lightlike propagation remains
Martin-consistent.  The proof proceeds in four stages.

\paragraph{1. Construction of the metric as a gauge of separation.}
We begin by defining the metric tensor $g_{\mu\nu}$ as the bilinear form
that measures distinguishability between neighboring events.  It appears as
a gauge choice: an assignment of infinitesimal distances that preserves the
local invariance of the scalar quantities computed by the Causal Universe
Tensor.  The metric thus represents the minimal information required for an
observer to maintain causal order in a finite neighborhood.

\paragraph{2. Connection as the rule of causal transport.}
Next we introduce the connection $\Gamma^{\lambda}_{\mu\nu}$ as the operator
that preserves these scalar invariants under parallel transport of
distinguishable events.  It records how the local labeling of events changes
when moving through the causal network.  The vanishing of the covariant
derivative $\nabla_\mu T^{\mu\nu}=0$ expresses Martin consistency in this
differential form: order is preserved under transport.

\paragraph{3. Curvature as the residue of inconsistency.}
Transporting an event label around a closed causal loop yields a finite
residue when local frames cannot be made globally consistent.  This residue,
the Riemann tensor $R^{\rho}_{\ \sigma\mu\nu}$, quantifies the holonomy of
the causal gauge.  Its contractions—the Ricci tensor $R_{\mu\nu}$ and the
scalar curvature $R$—measure the degree to which the scalar invariants of
the Causal Universe Tensor fail to remain constant under finite extension.

\paragraph{4. Einstein equation as the constraint of global consistency.}
Finally we impose that the total scalar invariant of order, represented by
the Einstein tensor $G_{\mu\nu}=R_{\mu\nu}-\tfrac{1}{2}Rg_{\mu\nu}$, balances
the energy–momentum content encoded in the lower-order invariants of the
Causal Universe Tensor:
\[
G_{\mu\nu} = 8\pi T_{\mu\nu}.
\]
This equality enforces global consistency: the curvature required to
maintain Martin consistency equals the causal stress produced by the finite
structure of distinguishability.  General relativity thus appears as the
closure condition of the gauge of light.

\begin{remark}
In summary, the arc of this proof mirrors the logic of the entire work.
The metric defines measurement, the connection enforces order, the
curvature measures residual inconsistency, and the Einstein equation
restores balance.  Each level is a higher-order invariant of the same causal
calculus.  Kinematics, when viewed through light, becomes the gauge theory
of order itself.
\end{remark}

\subsection{Defining Entropy}

\begin{definition}[Entropy]
Let $\mathcal{C}$ denote the causal set of distinguishable events accessible to an observer, and let $\Omega(\mathcal{C})$ be the set of all admissible micro–orderings of those events consistent with the Reciprocity Law.  
The \emph{entropy} associated with $\mathcal{C}$ is the logarithm of this count:
\[
S[\mathcal{C}] = k_{\mathrm{B}} \ln |\Omega(\mathcal{C})|.
\]
Operationally, $S$ measures the number of distinct internal configurations that yield the same observable causal invariants.  
In the continuum limit, variations in $S$ appear as gradients of informational curvature; coupling this quantity to the stress tensor defines the entropic contribution to spacetime geometry.
\end{definition}

\begin{remark}
Entropy in this framework is not a measure of disorder but of indistinguishability: 
it quantifies how many discrete causal configurations correspond to the same continuous geometry.
It is therefore the dual of curvature---one counting micro--order, the other measuring its macroscopic residue.
\end{remark}



\section{Metric as a Gauge of Separation}
\label{sec:metric-gauge}

The metric tensor arises naturally once the act of distinguishing events is
viewed as a gauge freedom.  Every observer maintains a local convention for
labeling distinguishable events; what we call a \emph{distance} is merely the
scalar quantity that remains invariant when those local conventions are
changed.  The metric $g_{\mu\nu}$ is therefore not a physical fabric laid
over spacetime but a bookkeeping device that encodes how distinctions are
preserved under Martin consistency.

\subsection{From Distinction to Distance}

\subsection{Axiomatic Necessity}

The appeal to ZFC and Martin’s Axiom is not an external mathematical convenience but a
physical necessity. Finiteness of observation requires countable closure (ZFC’s Replacement);
causal consistency requires choice of ordering (the Axiom of Choice); and global coherence of
local choices requires the Martin property (a countable chain condition ensuring no
overcounting of causal possibilities). Thus each axiom corresponds to a measurable physical
principle:
\[
\text{Finiteness} \leftrightarrow \mathrm{ZFC}, \qquad
\text{Consistency} \leftrightarrow \mathrm{Martin}, \qquad
\text{Reversibility} \leftrightarrow \mathrm{Choice}.
\]
Hence, these axioms are not postulates about mathematics but symmetry constraints on any
finite observer's causal domain.

Let each infinitesimal event displacement be represented by the differential
$dx^\mu$, denoting the local coordinates an observer assigns to successive
measurements.  Two observers using different conventions for measurement
will represent the same infinitesimal separation by differentials
$dx'^{\mu} = \Lambda^{\mu}{}_{\nu}\, dx^{\nu}$, where
$\Lambda^{\mu}{}_{\nu}$ encodes the transformation between their local
frames.  To preserve the scalar invariants computed by the Causal Universe
Tensor, we require that the inner product of these displacements remain
unchanged:
\[
ds^2 = g_{\mu\nu}\, dx^{\mu}\, dx^{\nu}
      = g'_{\mu\nu}\, dx'^{\mu}\, dx'^{\nu}.
\]
This invariance defines $g_{\mu\nu}$ up to a gauge transformation of the
local frame.  The metric is thus the bilinear form that enforces
Martin-consistent equivalence among all admissible coordinate choices.

\subsection{The Metric as a Gauge Connection}

Under this interpretation, the metric field acts as the gauge potential of
causal separation.  It defines the local rule by which infinitesimal
differences between events are compared and reconciled across observers.
If $T^{\mu\nu}$ denotes the Causal Universe Tensor, then maintaining the
invariance of its scalar contractions,
\[
\delta(g_{\mu\nu} T^{\mu\nu}) = 0,
\]
requires a covariant definition of the derivative operator that absorbs
changes of frame.  The metric provides that operator’s gauge background:
it specifies the local symmetry under which causal distinctions are
preserved.

\begin{example}[Michelson--Morley as Gauge Isotropy of Causal Separation]

\textbf{Statement.} The null fringe shift in the Michelson--Morley interferometer operationally enforces that the causal interval
\[
ds^2 \;=\; -c^2 dt^2 + dx^2 + dy^2 + dz^2
\]
is invariant under orthogonal transport of the measurement path, i.e.\ the gauge preserving reciprocity is isotropic.

\textbf{Reciprocity framing.} Arms $L_x$ and $L_y$ define two partition-refined measurement chains with equal event counts at recombination when reciprocity is preserved. Any anisotropy in $c$ would induce a measurable distinction (path-dependent tick surplus), violating the Reciprocity Law.

\textbf{Calculation sketch.} The phase difference is
\[
\Delta\phi \;=\; \frac{2\pi}{\lambda}\,\Delta \ell_{\text{opt}}, \qquad 
\Delta \ell_{\text{opt}} \equiv (n\,\ell)_x-(n\,\ell)_y.
\]
Empirically $\Delta\phi \approx 0$ over apparatus rotations, implying $\Delta \ell_{\text{opt}}=0$ and hence invariance of $ds^2$ under rotation of the apparatus frame. In our terms: the metric is the \emph{gauge of separation} that preserves the dual invariants of measurement and variation.
\end{example}

\begin{example}[Thought Experiment: The Shadow Puppet Theater and Gauge Selection]
\textbf{N.B.} This experiment clarifies how the metric acts as a distinction gauge rather than a physical field.

\emph{Setup.}  
Project silhouettes of puppets (events) onto a wall (the Universe Tensor) using a lantern (causal order).  
Tilting the lantern changes how separations appear on the wall—analogous to a gauge transformation.

\emph{Demonstration.}  
When the wall is flat and illumination uniform, puppet overlaps map bijectively to coordinates.  
Bend the wall or tilt the lantern: shadows distort, losing closure.  
Only orientations preserving overlap correspond to admissible $g_{\mu\nu}$ selections.

\emph{Interpretation.}  
The metric is the rule ensuring that co- and contra-variant descriptions yield the same invariant shadow.  
Parallel transport preserves distinguishability; distortion measures curvature—residue of inconsistency.
\end{example}


\subsection{Causal Interpretation}

Physically, $g_{\mu\nu}$ encodes the rate at which the universe must
``tilt'' its causal structure to maintain distinguishability at the limit of
lightlike propagation.  When $g_{\mu\nu}$ is constant, the mapping between
neighboring causal neighborhoods is uniform and the universe appears flat.
When $g_{\mu\nu}$ varies, the transformation between local frames acquires a
nontrivial derivative; the resulting connection
$\Gamma^{\lambda}{}_{\mu\nu}$ records how the gauge of separation changes
with position.

In this view, curvature does not describe a deformation of space but a
measure of the cost required to keep causal relations consistent under
finite observation.  The metric therefore functions as the lowest-order
field in a hierarchy of corrections that preserve the scalar invariants of
the Causal Universe Tensor.  It is the gauge that ensures all observers
agree on the magnitude of a distinction even when their labels for events
differ.

\begin{remark}
To summarize: the metric is the gauge of separation.  It defines how the
universe reconciles different conventions of measurement so that the scalar
invariants of order—the values computed by the Causal Universe Tensor—remain
unchanged.  Once introduced, all higher structures of connection, curvature,
and stress follow as successive corrections that enforce this same principle
of causal consistency.
\end{remark}

\begin{example}[Galileo's Free--Fall as the Flat--Space Limit of Causal Motion]
In Galileo's experiment, two spheres of unequal mass are dropped from the
same height and reach the ground simultaneously.  Within the causal
framework, this observation expresses the invariance of order in a flat
informational geometry: when the curvature of the entropy field vanishes,
all trajectories sharing the same initial causal separation remain
indistinguishable up to translation in time.

Let the causal paths be $\gamma_1(t)$ and $\gamma_2(t)$, each governed by
\[
\frac{d^2 x}{dt^2} = g,
\]
where $g$ is constant.  Because the informational curvature
$\nabla_i \nabla_j S$ is zero, the metric gauge $g_{ij}$ is uniform, and
the Reciprocity Law preserves equality of causal intervals:
\[
\delta^2 x_1 = \delta^2 x_2.
\]
Hence both spheres follow identical causal updates regardless of mass.

In this limit, the observer's partition $\mathcal{P}_n$ resolves all
relevant distinctions—position, time, and acceleration—so the reciprocity
mapping
\[
\Phi : V /{\sim_{\mathcal{P}_n}} \;\longleftrightarrow\;
M /{\sim_{\mathcal{P}_n}}
\]
is exact.  No refinement of the partition changes the outcome: the motion
is deterministic.  Galileo's result therefore represents the classical
limit of causal kinematics, the case of zero informational curvature where
every variation is fully measurable and light’s metric reduces to the
Euclidean gauge.
\end{example}


\begin{example}[Gravitational Lensing as Informational Curvature]
When light passes near a massive object, its trajectory bends—not because
space itself is a physical medium that deforms, but because the mapping that
preserves causal order becomes nonuniform.  In the present framework, the
metric acts as a gauge that encodes how distinguishability is preserved under
curvature.  Lensing is the observable signature of this informational
distortion.

Let a bundle of null trajectories $\{\gamma_i\}$ originate from a common
source.  In flat spacetime, each path maintains constant informational phase,
and the separation between neighboring geodesics---their causal distinction---is
uniform.  Introducing a local entropy gradient $S(x)$ modifies this gauge:
the effective distance between successive events changes by
\[
\delta ds^2 \propto \nabla_i \nabla_j S,
\]
so that the extremal path satisfies
\[
\delta \!\int ds = 0 \quad \Longrightarrow \quad
\frac{d^2 x^i}{d\lambda^2} + \Gamma^i_{jk}\frac{dx^j}{d\lambda}\frac{dx^k}{d\lambda} = 0,
\]
with $\Gamma^i_{jk}$ determined by the informational curvature $\partial_i
\partial_j S$.  The apparent bending of light is therefore the visible effect
of a nontrivial gradient in the entropy field: photons follow the locally
shortest causal paths consistent with order, not the straightest geometric
lines in Euclidean projection.

Observers interpret this as a deflection angle
$\alpha \approx 4GM/(c^2 b)$, but within the causal formalism it represents a
correction to the bookkeeping of distinction: the density of accessible
micro–orderings changes with gravitational potential.  Lensing thus measures
how informational curvature couples to geometry---a macroscopic manifestation
of the same reciprocity that defines the metric itself.
\end{example}

\begin{example}[The Three--Body Problem as Computational Reciprocity]
Consider three point masses $m_1,m_2,m_3$ interacting gravitationally with
positions $r_i(t)\in\mathbb{R}^3$.  Newtonian dynamics gives
\[
m_i\ddot r_i
  \;=\; G\sum_{j\neq i} \frac{m_i m_j}{\|r_j-r_i\|^3}\,(r_j-r_i),
  \qquad i=1,2,3.
\]
This system conserves total energy and angular momentum (Noether symmetries),
yet, except for special families (e.g.\ Euler and Lagrange configurations), it
admits no closed-form solution.  In the present framework, this means the
reciprocity map closes \emph{only computationally}: the admissible update that
preserves order and invariants exists, but it must be realized by an iterative,
order-preserving scheme.

\begin{remark}[Historical Context: Noether Symmetry and Reciprocity]
Emmy Noether’s theorem (1918) formalized the equivalence between invariance under continuous transformation and the conservation of a measurable quantity.  
Within this framework, that equivalence becomes a corollary of the Reciprocity Law: each symmetry of the causal measure corresponds to an invariant distinction preserved under evolution.
\end{remark}


Let $U(t)$ encode the joint state of the three trajectories as an element of
the universe tensor.  Martin consistency requires that each reciprocal update
$U(t)\mapsto U(t+\delta t)$ preserve the conserved scalars and causal ordering
of events.  Analytic spline closure ($U^{(4)}\!=0$) is insufficient here:
interactions couple the segments so that local cubic envelopes do not
globally commute.  The correct closure is \emph{algorithmic}: a reversible,
symplectic, order-preserving integrator (e.g.\ velocity Verlet/leapfrog) that
implements the reciprocity step without violating the invariants,
\[
\Phi_{\delta t}^{\text{symp}}:\; U(t)\longmapsto U(t+\delta t),\qquad
\delta E=0,\;\;\delta L=0 \;\text{(to integrator accuracy)}.
\]
Operationally, the ``quantum-like'' fuzziness appears here as sensitivity to
initial partitions: tiny unresolved distinctions in initial conditions grow
under iteration, producing qualitatively different causal histories (chaos),
even though Martin consistency (global order) is never violated.

Thus the three--body problem exemplifies a domain where physics \emph{requires}
computation: reciprocity and consistency still govern the update, but their
closure cannot be written in elementary functions.  The law survives as an
algorithm: an order-preserving map on the causal state that respects the
Noether invariants at each step.
\end{example}


\section{The Rule of Causal Transport}
\label{sec:rule-causal-transport}

Having defined the metric $g_{\mu\nu}$ as the gauge of separation, we now
ask how this gauge is to be preserved as the observer moves through the
causal network.  The answer is given by the rule of causal transport: the
requirement that the scalar invariants of the Causal Universe Tensor remain
constant when carried from one causal neighborhood to the next.

\subsection{From Gauge Preservation to Connection}
\begin{example}[The Two Observers and the Invariant Count~\cite{lorentz1904,einstein1905,minkowski1908,weyl1929,yang1954}]
\label{ex:two-observers}

\textbf{Concept.}
This experiment reinterprets Lorentz contraction as a gauge effect.
Rather than a physical shortening of rods or slowing of clocks, it is
seen as a mathematical contraction of the gauge components themselves,
required to preserve the invariant scalar---the total count of causal
distinctions---for all observers.

\textbf{Setup.}
Consider a fundamental causal interval: the emission and absorption of a
light pulse.  This interval consists of an invariant number $N$ of
distinguishable ``ticks'' or events.  This $N$ is the true physical
reality upon which all observers agree.  In the continuum limit, the
corresponding invariant is the line element
\[
ds^{2} = g_{\mu\nu}\,dx^{\mu}dx^{\nu}.
\]

\textbf{Observer $\mathcal{O}$ (Rest Frame).}
This observer is at rest relative to the interval.  Her gauge---her
coordinate system---is aligned with the events.  She measures
\[
N_t = N,\qquad N_x = 0,
\]
so that
\[
ds^{2} \propto N_t^{2} - N_x^{2} = N^{2}.
\]

\textbf{Observer $\mathcal{O}'$ (Moving Frame).}
A second observer moves at constant velocity $v$ relative to
$\mathcal{O}$.  Both observers describe the same physical interval but
with different gauge components.  In the traditional interpretation of
special relativity, $\mathcal{O}'$ observes time dilation and length
contraction:
\[
N_t' \neq N_t,\qquad N_x' \neq N_x,
\]
yet the invariant
\[
(N_t')^{2} - (N_x')^{2} = N^{2}
\]
remains unchanged.

\textbf{New View (Gauge Contraction).}
In the informational picture developed here, the observers' clocks and
rods are not physically distorted.  Rather, their \emph{gauges}---their
coordinate mappings of causal distinction---are related by the Lorentz
transformation $\Lambda^{\mu}{}_{\nu}$, which enforces invariance of the
scalar count:
\[
g'_{\mu\nu} = \Lambda^{\rho}{}_{\mu}\Lambda^{\sigma}{}_{\nu} g_{\rho\sigma}.
\]
The Lorentz transformation thus acts as a gauge update preserving the
invariant number of distinctions.  What appears as ``contraction'' is
simply the compensation required to maintain consistency under
relabelling~\cite{einstein1905,minkowski1908,weyl1929}.

\textbf{Conclusion.}
The phenomenon of Lorentz contraction therefore migrates from the
physical measuring apparatus to the mathematical gauge itself.
The metric $g_{\mu\nu}$ is a bookkeeping device ensuring that each
observer's local representation yields the same invariant scalar---the
same informational count $N$.  Spacetime kinematics thus emerge as the
consistency rules of an informational gauge: a structure ensuring global
agreement on what is distinguishable.
\end{example}


Consider the transport of a vector field $V^{\mu}$ representing a direction
in the space of distinguishable events.  To maintain Martin consistency,
the change in $V^{\mu}$ along an infinitesimal displacement $dx^{\nu}$ must
not alter any scalar quantities computed from the tensor
$g_{\mu\nu}V^{\mu}V^{\nu}$.  The differential form of this requirement is
\[
\nabla_{\nu} g_{\mu\sigma} = 0,
\]
which defines the Levi–Civita connection
$\Gamma^{\lambda}_{\ \mu\nu}$.  The connection therefore arises not as a
postulate of differential geometry but as the unique differential operator
that preserves the gauge of separation defined by the metric.  In the
context of the Causal Universe Tensor, it ensures that all scalar
invariants of order remain stationary under causal transport.

\subsection{Operational Meaning}

Each component $\Gamma^{\lambda}_{\ \mu\nu}$ records how the act of
distinction must be adjusted when an observer translates a local rule of
measurement from one event to its neighbor.  It is, in essence, the
differential bookkeeping of consistency.  When the metric is uniform,
$\Gamma^{\lambda}_{\ \mu\nu}=0$, and the mapping of causal neighborhoods is
trivial: straight lines remain straight.  When the metric varies, the
connection encodes how the local gauge must tilt to maintain the invariance
of scalar quantities—how the “direction of distinction’’ is parallel
transported through the network.

\subsection{Parallel Transport and Martin Consistency}

Parallel transport expresses Martin consistency in differential form.  A
vector is said to be parallel transported along a curve $x^{\mu}(s)$ if it
satisfies
\[
\frac{DV^{\lambda}}{Ds} = 
\frac{dV^{\lambda}}{ds} + 
\Gamma^{\lambda}_{\ \mu\nu}\,
V^{\mu}\,\frac{dx^{\nu}}{ds} = 0.
\]
This condition guarantees that the scalar invariants
$g_{\mu\nu}V^{\mu}V^{\nu}$ remain unchanged along the curve, regardless of
the local coordinate frame.  The connection therefore enforces the
\emph{covariant constancy} of the causal gauge: every observer’s
measurements can differ, but the underlying order they describe remains
identical.
\begin{example}[Non-Abelian transport and curvature]
Let $A_\mu(x)$ be a matrix-valued connection (local gauge of distinction). Parallel transport along a path $\gamma$ uses the path-ordered exponential
\[
U[\gamma]=\mathcal{P}\exp\!\left(\int_\gamma A_\mu\,dx^\mu\right).
\]
For an infinitesimal rectangle spanned by $\delta x^\mu,\delta x^\nu$,
\[
U[\square]=I+F_{\mu\nu}\,\delta x^\mu\delta x^\nu+O(\delta^3),\qquad
F_{\mu\nu}=\partial_\mu A_\nu-\partial_\nu A_\mu+[A_\mu,A_\nu].
\]
The $[A_\mu,A_\nu]$ term \emph{is} the non-commutative residue of transporting in different orders around the loop.
Thus, failure of local updates to commute produces a measurable scalar (via contractions of $F_{\mu\nu}$) that records global non-closure—our curvature as informational residue.
This ties the reciprocity-preserving gauge to the field strength that appears in the continuum limit.
\end{example}


	\subsection{Causal Interpretation}

	Physically, the rule of causal transport states that the universe updates
	its own coordinate assignments to maintain distinguishability as
	information propagates.  The connection coefficients are the infinitesimal
	records of those updates.  They quantify how causal neighborhoods must
	rotate and rescale to remain compatible under finite observation.  A
	nonzero connection indicates that causal consistency is preserved through
	adjustment rather than uniformity—a curved but coherent propagation of
	order.

	\begin{remark}
	In summary, the connection $\Gamma^{\lambda}_{\ \mu\nu}$ is the rule of
	causal transport: the unique differential relation that preserves the gauge
	of separation under motion.  It translates the logical demand of Martin
	consistency into a local dynamical law.  Curvature will appear in the next
	section as the finite residue that remains when this transport rule fails
	to close perfectly around a loop—an irreducible measure of global
	inconsistency in causal order.
	\end{remark}
\begin{example}[Invariance of the Causal Interval \(ds^2\)]
Consider two observers, $\mathcal{O}$ and $\mathcal{O}'$, who each assign
coordinates to the same pair of infinitesimally separated events.  Their
local labels differ by a gauge transformation of the form
\[
dx'^{\mu} = \Lambda^{\mu}{}_{\nu}\, dx^{\nu},
\]
where $\Lambda^{\mu}{}_{\nu}$ preserves the ordering of causal relations as
required by Martin’s Axiom.  The scalar quantity
\[
ds^2 = g_{\mu\nu}\, dx^{\mu}\, dx^{\nu}
\]
represents the infinitesimal measure of distinguishability between these
events—the local contraction of the Causal Universe Tensor with the gauge
of separation.

Under the gauge transformation, the differentials and metric transform as
\[
g'_{\mu\nu} = 
\Lambda^{\alpha}{}_{\mu}\, 
\Lambda^{\beta}{}_{\nu}\,
g_{\alpha\beta},
\qquad
dx'^{\mu} = \Lambda^{\mu}{}_{\sigma}\, dx^{\sigma}.
\]
Substituting these into the definition of the interval yields
\[
ds'^2 = g'_{\mu\nu}\, dx'^{\mu}\, dx'^{\nu}
       = g_{\alpha\beta}\, dx^{\alpha}\, dx^{\beta}
       = ds^2.
\]
Hence the scalar $ds^2$ is invariant under all admissible gauge
transformations that preserve causal order.  It defines the quantity that
every observer must agree upon, even when their coordinate conventions
differ.

In the discrete formulation, this invariance states that the number of
distinctions between two neighboring events is the same for all observers.
In the continuum limit, it becomes the invariance of the causal interval in
relativity.  Both express the same principle: the universe may bend,
accelerate, or dilate, but the order of events—the fact that one event can
distinguish another—remains unchanged.
\end{example}

% Placement: Part III §4.2 after “The Rule of Causal Transport” (ideal after §4.2.3 “Parallel Transport and Martin Consistency”)
\subsubsection*{Example: Pound--Rebka Gravitational Redshift as Entropic Transport}

\textbf{Statement.} Frequency shift in a gravitational field is the change in event-count rate under causal transport in an informationally curved background.

\textbf{Key relation (weak field).}
\[
\frac{\Delta \nu}{\nu} \;\approx\; \frac{\Delta \Phi_{\text{grav}}}{c^2} \;=\; \frac{g\,h}{c^2}.
\]
\textbf{Reciprocity framing.} Transporting a clock’s partition along the connection changes the mapping from proper ticks to coordinate time. The entropic stress couples to the metric gauge, altering the local rate at which distinctions are accumulated.

\textbf{Operational consequence.} Redshift is parallel transport of the causal gauge: invariants are preserved, but the local counting density transforms, observed as a shift in $\nu$.


