\chapter{The Kinematics of Matter}
\label{chap:wave}

\section{Introduction: Martin's Condition and the Continuity of Causal Propagation}

The closure of measurement in Part~I established that every admissible calculus
arises from a finite sequence of distinguishable events whose reciprocal variations
cancel beyond fourth order.
The resulting smooth field $U(x)$ represents not an assumption of continuity,
but the unique extension that preserves causal consistency under the
\emph{Axiom of Event Selection}.
Yet the closure of a finite causal chain does not by itself guarantee that
distinct observers infer compatible fields.
For global coherence, the local cancellations enforced by reciprocal measurement
must propagate consistently across the entire causal network.
This propagation is the content of \emph{Martin's Condition}.

\begin{definition}[Martin's Condition (Conceptual)]
A causal network satisfies \emph{Martin's Condition}
if every locally finite subset of events can be extended to a globally
consistent ordering without introducing new distinguishabilities.
Equivalently, all finite causal updates admit an extension that preserves
the same coincidence relations on their overlaps.
\end{definition}

Intuitively, Martin's Condition demands that information created in one region
does not contradict information measured in another.
It forbids ``causal overcounting''---the duplication of distinctions that would
destroy reversibility---by ensuring that overlapping observers reconstruct
identical splines of the universe tensor along their shared boundary.
Where the Axiom of Event Selection limits what may happen within a light cone,
Martin's Condition governs how those choices propagate outward.
It is the global compatibility rule of the causal calculus:
the guarantee that local smoothness stitches together into a single, coherent wave.

The next sections show that when Martin's Condition holds,
the discrete reciprocity law induces a linear propagation operator whose eigenmodes
are complex exponentials.
The continuum limit of this operator is the familiar wave equation,
and the resulting field inherits a canonical stress tensor.
Thus the same closure that produced calculus in Part~I now produces
the continuous propagation of energy and information---%
the universal phenomenon we recognize as a \emph{wave}.

\subsubsection*{Example: Davisson--Germer and the Universality of Causal Waves}

\textbf{Statement.} Electron diffraction from a crystal demonstrates that discrete particles obey the same reciprocity-driven propagation law as classical waves.

\textbf{Key relation.} Bragg condition with de\,Broglie wavelength:
\[
2d\,\sin\theta \;=\; m\,\lambda, \qquad \lambda \;=\; \frac{h}{p}.
\]
\textbf{Reciprocity framing.} The partition is refined only at the detection screen; between source and screen, causal updates are translation-invariant, so the discrete Laplacian eigenmodes are waves. Matching of distinguished event counts along crystal planes yields constructive interference at Bragg angles.

\textbf{Operational consequence.} Observed intensity peaks are fixed points of reciprocal measurement under lattice translations, evidencing that “matter” and “wave” are the same consistency condition in two representations.


\section{Interaction: The Union of Ordered Events}

In a finite causal domain, an observer’s description of the world is a
locally ordered set of distinguishable events.
When two such domains overlap, the question of \emph{interaction} arises:
how are their separate orderings reconciled into a single consistent history?
Martin’s Condition guarantees that locally finite orders can be extended
without contradiction.
Interaction is the constructive realization of that extension.

\begin{definition}[Interaction of Causal Sets]
Let $(E_1,\preceq_1)$ and $(E_2,\preceq_2)$ be locally finite posets of events,
each satisfying Martin’s Condition on its own domain.
Their \emph{interaction} is the smallest poset
\[
(E_{12},\preceq_{12}), \qquad
E_{12} = E_1 \cup E_2,
\]
whose order $\preceq_{12}$ is the transitive closure of
$\preceq_1 \cup \preceq_2$ restricted by the requirement that all overlaps
$E_1 \cap E_2$ remain consistent:
\[
\forall\, e,f \in E_1 \cap E_2,\;
e \preceq_1 f \;\Leftrightarrow\; e \preceq_2 f .
\]
\end{definition}
\begin{example}[Non-commuting measurements as event selection]
Let the event tensor act on a qubit. Take Pauli operators $\sigma_x,\sigma_z$ and projective predicates
\(
P_x^\pm=\tfrac{1}{2}(I\pm \sigma_x),\quad
P_z^\pm=\tfrac{1}{2}(I\pm \sigma_z).
\)
The selection ``measure $x$ then $z$'' corresponds to the ordered update
\(
U_{xz}(\rho)=\sum_{a,b\in\{\pm\}} P_z^b P_x^a\,\rho\, P_x^a P_z^b,
\)
while ``measure $z$ then $x$'' is
\(
U_{zx}(\rho)=\sum_{a,b} P_x^a P_z^b\,\rho\, P_z^b P_x^a.
\)
Because $[\sigma_x,\sigma_z]\neq 0$, one has $U_{xz}\neq U_{zx}$ in general; order changes the post-measurement state and the subsequent event statistics.
Yet the scalar count of admissible outcomes over the full branch set is preserved (the total probability $=1$), reflecting that the measurable bookkeeping across the causal domain remains consistent even when the micro-updates do not commute.
This realizes Event Selection as a non-commutative refinement of the distinguishability chain.
\end{example}


The overlap $E_1 \cap E_2$ represents events recognized by both observers.
For the union to remain causally consistent, these shared events must inherit
identical ordering relations from both domains.
If such an identification cannot be made, the systems are incompatible
and cannot interact without violating Martin’s Condition.

\begin{definition}[Interaction Event]
An event $e \in E_1 \cap E_2$ is called an \emph{interaction event}
if it is maximal in one order and minimal in the other:
\[
e \in \mathrm{Top}(E_1) \cap \mathrm{Min}(E_2)
\quad \text{or} \quad
e \in \mathrm{Top}(E_2) \cap \mathrm{Min}(E_1).
\]
Such an event terminates one causal chain and initiates another.
\end{definition}

Intuitively, an interaction occurs when the future boundary of one local
ordering meets the past boundary of another.
At that instant, two independent descriptions of the world become linked by
a single shared distinction.
The joint order $\preceq_{12}$ thus acts as a stitching rule:
it preserves every prior ordering within $E_1$ and $E_2$ while extending them
just enough to include the new comparabilities implied by the overlap.

\begin{proposition}[Union Consistency]
If $(E_1,\preceq_1)$ and $(E_2,\preceq_2)$ satisfy Martin’s Condition and
agree on all relations within $E_1 \cap E_2$, then their union
$(E_{12},\preceq_{12})$ also satisfies Martin’s Condition.
\end{proposition}

\begin{proof}[Idea of Proof]
Each finite subset $S \subseteq E_{12}$ lies within finitely many overlapping
domains $E_i$ that already satisfy Martin’s Condition.
Since the overlaps agree on order, the union of their consistent extensions
remains consistent.
Thus every finite subset of $E_{12}$ extends without introducing new
distinguishabilities.
\end{proof}

\noindent
\textbf{Interpretation.}
Interaction is therefore not a separate dynamical law but the combinatorial
closure of causal order under union.
Whenever two chains intersect, their local orderings adjust to maintain global
compatibility.
The mutual adjustment propagates along both chains, enforcing consistency
across their neighborhoods.
Viewed iteratively, this propagation behaves as a \emph{wave of ordering}:
a disturbance that travels through the poset whenever new overlaps are formed.
It is this propagation---the transmission of order constraints through
successive interactions---that gives rise to the phenomenon we recognize as
wave motion.

\subsection{Spooky Action as a Dantzig Pivot}

\begin{example}[Mach--Zehnder Interferometer as Causal Superposition]
Consider a photon entering a Mach--Zehnder interferometer.  
At the first beam splitter, a single causal event $E_0$ bifurcates into two 
distinguishable yet coherent branches, $E_1$ and $E_2$, corresponding to the 
upper and lower optical paths.  Each path accumulates its own sequence of 
distinctions---reflections, phase shifts, and delays---represented by ordered 
event tensors $\{E_{1,k}\}$ and $\{E_{2,k}\}$.

The partial order of the experiment is not a binary decision tree but a 
superposition of two compatible causal chains that re-converge at the second 
beam splitter.  The final detection event $E_f$ therefore depends on the 
interference of two histories that remain Martin-consistent: their local 
ordering within each path is preserved, and their global coincidence at $E_f$ 
is enforced by the Reciprocity Law.

Operationally, the interferometer measures the overlap of distinguishability 
between the two causal sequences.  When their accumulated phase difference 
$\Delta\phi$ equals an integer multiple of $2\pi$, the two histories are 
indistinguishable and the universe tensor records them as a single causal 
extension; when $\Delta\phi = \pi$, the histories cancel, producing a node of 
zero probability.  Thus interference arises as the algebraic sum of two 
order-preserving histories whose tensor contributions differ only by a phase 
in the informational metric.

In this framing, the Mach--Zehnder interferometer is the simplest laboratory 
realization of causal superposition: two distinguishable sequences whose 
difference is purely informational, revealing that interference is not a 
mystery of waves but a bookkeeping property of order under the Reciprocity 
Law.
\end{example}


Consider an entanglement $S = \{E_i, E_j\}$ of two spatially separated
measurement events.  By definition, the order of $E_i$ and $E_j$ may be
permuted without changing any invariant scalar of the universe tensor:
\[
E_i + E_j = E_j + E_i. \tag{15}
\]
Each pair of entangled events therefore constitutes a \emph{degenerate
basis} of the global causal structure: multiple local orderings are
consistent with the same global invariants.

\paragraph{Degeneracy and Feasibility.}
Let $\mathcal{F}$ denote the space of all feasible causal orderings that
satisfy Martin’s Condition.  Every element of $\mathcal{F}$ is a
physically admissible extension of the partial order of events.  When
two or more orderings yield the same invariants, the corresponding
configurations form a \emph{degenerate face} of $\mathcal{F}$—analogous
to a flat ridge in a linear‐programming polytope where the objective is
constant.  Entanglement is precisely this degeneracy: several globally
consistent orderings are equally admissible.

\paragraph{Selection as Pivot.}
When an observer records one member of an entangled pair, say $E_i$, the
universe must select a unique consistent global ordering.  This
selection is equivalent to a \emph{pivot operation} in the sense of
Dantzig: a transition from one feasible vertex of $\mathcal{F}$ to
another that preserves all constraints while choosing a particular
basis.  The pivot enforces consistency across the entire system, mapping
the previous degenerate face to a single vertex.  The resulting update
\[
U_{n+1} = \Phi_{\text{sel}}(U_n)
\]
is the causal analog of Dantzig’s step toward optimality: a global
reorganization that leaves all invariants unchanged but redefines which
variables are active.

\paragraph{Nonlocal Consistency.}
Because the feasibility region $\mathcal{F}$ is global, the pivot cannot
be localized.  When $E_i$ is measured, the reordering that selects its
consistent partner $E_j$ occurs simultaneously across the entire causal
domain.  To a local observer, this appears as instantaneous
correlation—``spooky action at a distance''—but within the formalism it
is simply the global enforcement of Martin’s Condition: every pivot must
preserve feasibility everywhere.  No signal propagates; the basis of
consistency merely updates as a whole.

\paragraph{Interpretation.}
Spooky action is therefore not a mysterious nonlocal force but the
\emph{global pivot of consistency} required to maintain a single
feasible ordering of the universe tensor.  Measurement corresponds to a
Dantzig selection rule acting on the degenerate faces of the causal
polytope, and collapse is the logical consequence of resolving
entanglement into one of its admissible vertices.  The Einstein–Podolsky–Rosen
paradox thus reduces to a combinatorial theorem:
\[
\text{Nonlocal correlation} \;=\; \text{Global preservation of feasibility}.
\]

% Placement: Part II §3.2.1 “Spooky Action as a Dantzig Pivot” or immediately after; ties to Martin consistency explicitly
\subsubsection*{Example: Bell--Aspect Tests as Global Martin Consistency}

\textbf{Statement.} Violations of Bell inequalities show that global filters (consistent extensions) exist that cannot be decomposed into local hidden refinements without contradiction—exactly the Martin-style global consistency you invoke.

\textbf{Key relation (CHSH).}
\[
S \;=\; |E(a,b)+E(a,b')+E(a',b)-E(a',b')| \;\le\; 2 \quad (\text{local}) \quad \text{vs}\quad S_{\text{QM}}\le 2\sqrt{2}.
\]
\textbf{Reciprocity framing.} A single global entanglement class $S$ allows reassociation (permutation) before refinement. Local pre-assignments would violate dense-set meeting across settings; the observed $S>2$ indicates that only a \emph{global} selection (filter meeting all dense constraints) is admissible.

\textbf{Operational consequence.} “Nonlocality” is reinterpreted as \emph{global order-preserving selection}: the event filter meets all dense subsets (settings) without a jointly measurable pre-partition.


\subsection{The Qubit as an Example of Event Selection}

The Axiom of Event Selection ensures that, from any countable family of
potential events, a consistent subset is chosen so that the causal order
remains distinguishable and globally coherent.
When two events $E_0$ and $E_1$ are equally admissible until such a
selection is made, the pair forms the minimal unit of causal degeneracy.

\begin{definition}[Qubit as a Causal Doublet]
Let $S=\{E_0,E_1\}$ be an entangled subset of events satisfying
$E_0+E_1 = E_1+E_0$.
Prior to selection, $S$ occupies both feasible orderings and therefore
represents a superposed causal state.
Applying the selection operator $\Phi_{\mathrm{sel}}$ resolves this
degeneracy:
\[
\Phi_{\mathrm{sel}}(S) = E_b , \qquad b\in\{0,1\}.
\]
In the continuum limit this relation corresponds to the quantum state
\[
|\psi\rangle = \alpha|0\rangle + \beta|1\rangle ,
\]
where the coefficients $(\alpha,\beta)$ encode the relative weights of the
feasible orderings prior to event selection.
\end{definition}

Thus a \emph{qubit} is the simplest instance of the Event Selection
process---the minimal pair of distinguishable yet unselected events.
Measurement, represented by the Dantzig pivot $\Phi_{\mathrm{sel}}$,
corresponds to choosing one consistent ordering within this causal doublet,
mirroring the projection of a quantum superposition onto a definite basis
state.


\subsection{Hawking Radiation as the Loss and Restoration of Order}

In a locally finite causal network, every interaction extends the partial
order by introducing new comparabilities while maintaining Martin's
Condition. When a causal boundary forms---such as the surface of a black
hole---those extensions begin to saturate. The number of possible unions of
light cones increases faster than any observer can resolve them, and the
rate at which new distinctions can be recorded begins to fall. What we call
a \emph{horizon} is the surface beyond which the reconciliation of causal
updates exceeds the observer's computational capacity to process them.

\begin{definition}[Causal Horizon]
Let $(E,\preceq)$ be a locally finite poset.  
A subset $H \subset E$ is a \emph{causal horizon} for an observer if there
exist events $e,f \in E$ such that $e \preceq h$ for some $h \in H$ but
$f \not\preceq h$ for any $h \in H$, and no finite extension of the
observer's order can include both $e$ and $f$.  
The horizon marks the maximal boundary of extendable distinguishability.
\end{definition}

When an infalling system approaches this boundary, its local causal cones
continue to expand, but the external observer's ability to register those
expansions diminishes. The total number of events on the infaller's
worldline grows rapidly, while the observer's \emph{distinct} reception
count grows only logarithmically. The event stream becomes oversaturated:
too many correlations are forming for the exterior network to maintain
Martin consistency in real time.

\paragraph{Observer--Side Perception of Order.}
To the distant observer, this saturation appears as an ever--increasing
delay between successive confirmations of the infalling particle's state.
Each emitted distinction must traverse an ever--widening intersection of
causal cones before it can be reconciled with the external order. Because
the unions $E_{\mathrm{obs}} \cup E_{\mathrm{infall}}(t)$ grow
super--linearly in size as the infaller approaches the horizon, the cost of
maintaining order consistency rises faster than the causal network can
propagate it.

Let $N_{\mathrm{ext}}(t)$ be the number of distinguishable updates received
before coordinate time $t$.  If $|E(t)|$ denotes the size of the causal
union at that instant, then
\[
\frac{dN_{\mathrm{ext}}}{dt}
\;\propto\;
\frac{1}{|E(t)|}\,\frac{d|E(t)|}{dt}.
\]
Near the horizon, $|E(t)| \sim (1 - r_s/r)^{-1}$, so as $r \to r_s$,
\[
\lim_{t \to \infty} \frac{dN_{\mathrm{ext}}}{dt} = 0.
\]
The apparent ``freezing'' of the particle in time is therefore not an
illusion of geometry but a property of information flow: the observer's
frame can no longer complete the reconciliation of causal updates as the
interior domain's informational density diverges.

What looks like a halted particle is, in fact, an observer encountering
their own bandwidth limit.  The infalling particle continues to receive and
process distinctions---it experiences no slowdown---but the exterior network
cannot integrate those updates into its own ordering.  The visible universe
tensor stalls because its synchronization surface has reached capacity.

The lag, then, is the \emph{signature of finite computation}: the universe
enforcing the Axiom of Order by denying further updates until all active
causal cones can be reconciled.  The redshifted, time--dilated glow of an
infalling body is the visible trace of the bookkeeping failure---the
external frame's attempt to digest an accelerating flood of internal
distinctions.

\begin{definition}[Order Collapse and Restoration]
Given a Martin bridge $R \subset E_{\mathrm{in}} \times E_{\mathrm{out}}$,
its \emph{collapse} occurs when all $e_{\mathrm{in}} \in E_{\mathrm{in}}$
become causally unreachable.  The induced order on $E_{\mathrm{out}}$ is
restored by introducing surrogate events $E_{\mathrm{rad}}$ and relations
$R' \subset E_{\mathrm{out}} \times E_{\mathrm{rad}}$ such that
$(E_{\mathrm{out}} \cup E_{\mathrm{rad}},\preceq')$ again satisfies
Martin's Condition.
\end{definition}

Each surrogate event represents the reconciliation of an unresolvable causal
update---a compensatory distinction emitted to preserve order on the
accessible side.  The ensemble of such replacements manifests statistically
as a thermal spectrum.

\begin{proposition}[Hawking Radiation as Order Completion]
The apparent radiation observed at the horizon corresponds to the
distribution of surrogate events $E_{\mathrm{rad}}$ required to restore
Martin consistency after the collapse of a causal bridge.  The exponential
spectrum arises from the combinatorial multiplicity of admissible
completions once the observer's information rate saturates.
\end{proposition}

\paragraph{Relation to Holographic Consistency.}
The informational asymptote described above is the discrete analogue of the
holographic correspondence between bulk and boundary theories.  The
interior causal domain $E_{\mathrm{in}}$ plays the role of the bulk, its
rapidly expanding unions of light cones encoding the fine--grained local
order.  The exterior domain $E_{\mathrm{out}}$, bounded by the horizon,
functions as the boundary field theory whose finite causal capacity
reconstructs that interior.  The Martin bridge
$R \subset E_{\mathrm{in}} \times E_{\mathrm{out}}$ acts as the holographic
map: a discrete correspondence ensuring that every admissible bulk update
has a representable boundary image.  When the bridge collapses, the boundary
compensates by emitting the surrogate events $E_{\mathrm{rad}}$, analogous
to boundary degrees of freedom restoring consistency in the AdS/CFT duality.
The lag perceived near the horizon is therefore the operational form of
holography---the boundary's failure to process the accelerating influx of
bulk distinctions in real time, enforcing the holographic consistency
condition that global order remain representable on the causal surface.


\section{Wave Amplitude from Interaction Counts}

Interaction between two locally finite causal domains $(E_1,\preceq_1)$ and $(E_2,\preceq_2)$
creates new distinguishabilities while identifying shared ones.
We define the \emph{wave amplitude} as the net number of new, non-overlapping events produced by the union,
i.e. the cardinality of the set difference between union and intersection.

\begin{definition}[Amplitude of Interaction]
Let $E_{12}=E_1\cup E_2$ be the union poset obtained under Martin's Condition,
with overlap $E_{1\cap 2}=E_1\cap E_2$ order-consistent.
The \emph{amplitude} of the interaction is
\[
\mathcal{A}(E_1,E_2)
:= \bigl| (E_1 \cup E_2) \setminus (E_1 \cap E_2) \bigr|
= |E_1| + |E_2| - 2|E_1 \cap E_2|.
\]
Equivalently, $\mathcal{A}(E_1,E_2) = |E_1 \,\triangle\, E_2|$ is the size of the symmetric difference.
\end{definition}

\noindent
\textbf{Interpretation.}
$\mathcal{A}(E_1,E_2)$ counts exactly the distinguishabilities that are \emph{new to the union}:
it removes anything already shared (the intersection) and keeps only the net additions.
Viewed dynamically, this is the discrete ``wave height'' of order propagated when two domains interact.

\subsection*{Basic Properties}

\begin{proposition}[Symmetry and Nonnegativity]
For any locally finite $E_1,E_2$,
\[
\mathcal{A}(E_1,E_2) = \mathcal{A}(E_2,E_1) \ge 0,
\qquad
\mathcal{A}(E_1,E_2)=0 \iff E_1=E_2.
\]
\end{proposition}

\begin{proof}[Proof sketch]
Symmetry follows from the symmetry of union, intersection, and cardinality.
Nonnegativity is immediate from the definition as a set cardinality.
If $E_1=E_2$, the symmetric difference is empty, hence amplitude $0$.
Conversely, if the symmetric difference is empty, the sets coincide.
\end{proof}

\begin{proposition}[Upper and Lower Bounds]
\[
\bigl||E_1|-|E_2|\bigr| \;\le\; \mathcal{A}(E_1,E_2) \;\le\; |E_1|+|E_2|.
\]
\end{proposition}

\begin{proof}[Proof sketch]
Use $|E_1\cap E_2|\le \min\{|E_1|,|E_2|\}$ and
$\mathcal{A}=|E_1|+|E_2|-2|E_1\cap E_2|$ for the upper bound.
For the lower bound, observe $|E_1\cap E_2|\ge \max\{0,\,|E_1|+|E_2|-|E_1\cup E_2|\}$ and $|E_1\cup E_2|\le |E_1|+|E_2|$.
\end{proof}

\begin{proposition}[Additivity on Disjoint Domains]
If $E_1\cap E_2=\varnothing$, then
\[
\mathcal{A}(E_1,E_2)=|E_1|+|E_2|.
\]
\end{proposition}

\begin{proof}[Proof sketch]
With empty intersection, $(E_1\cup E_2)\setminus(E_1\cap E_2)=E_1\cup E_2$, so the amplitude is the size of the disjoint union.
\end{proof}

\begin{proposition}[Triangle-Type Inequality]
For any locally finite $E_1,E_2,E_3$,
\[
\mathcal{A}(E_1,E_3) \;\le\; \mathcal{A}(E_1,E_2) + \mathcal{A}(E_2,E_3).
\]
\end{proposition}

\begin{proof}[Proof sketch]
$\mathcal{A}$ is the cardinality of the symmetric difference, which is the Hamming distance on indicator functions of subsets.
The triangle inequality for Hamming distance yields the claim.
\end{proof}

\subsection*{Order-Sensitive Refinement}

The amplitude defined above counts events.
We now relate it to the number of \emph{new comparabilities} created by the interaction.

\begin{definition}[Frontiers and New Comparabilities]
For a poset $(E,\preceq)$, write $\mathrm{Top}(E)$ for maximal elements and $\mathrm{Min}(E)$ for minimal elements.
Given $(E_1,\preceq_1)$ and $(E_2,\preceq_2)$ with order-consistent overlap and union order $\preceq_{12}$, define
\[
\Delta_{\!\prec}(E_1,E_2)
:= \#\bigl\{(e,f)\in (E_1\setminus E_2)\times (E_2\setminus E_1)\;:\; e \prec_{12} f \text{ or } f \prec_{12} e \bigr\}.
\]
This counts the \emph{newly created} comparabilities across the interface.
\end{definition}

\begin{proposition}[Amplitude Bounds New Comparabilities]
\[
\Delta_{\!\prec}(E_1,E_2) \;\le\; \mathcal{A}(E_1,E_2)\cdot \min\{|E_1\setminus E_2|,\,|E_2\setminus E_1|\}.
\]
Moreover, if the interface is ``thin'' (only frontier elements interact), then
\[
\Delta_{\!\prec}(E_1,E_2) \;\asymp\; |\mathrm{Top}(E_1)\cap(E_1\setminus E_2)| \cdot |\mathrm{Min}(E_2)\cap(E_2\setminus E_1)|
\]
up to a factor determined by Martin-consistent tie-breaking.
\end{proposition}

\begin{proof}[Proof sketch]
Each new comparability pairs one element from the left difference with one from the right difference.
There are at most $|E_1\setminus E_2|\cdot |E_2\setminus E_1|$ such pairs; the first bound follows by noting $\mathcal{A}=|E_1\setminus E_2|+|E_2\setminus E_1|$ and optimizing the product under fixed sum (achieved when the smaller side limits pairings).
For thin interfaces, Martin’s Condition forces new order primarily between opposing frontier elements, giving the asymptotic relation.
\end{proof}

\subsection*{Superposition over Multiple Domains}

\begin{proposition}[First-Order Superposition]
For three domains $E_1,E_2,E_3$ with small triple-overlap,
\[
\bigl|\, \mathcal{A}(E_1\cup E_2, E_3) - \bigl(\mathcal{A}(E_1,E_3)+\mathcal{A}(E_2,E_3)\bigr) \,\bigr|
\;\le\; 2\,|E_1\cap E_2 \cap E_3|.
\]
\end{proposition}

\begin{proof}[Proof sketch]
Use inclusion--exclusion on unions and intersections to expand both sides and cancel terms.
All discrepancies arise from triple-overlap terms, each contributing at most $2$ in absolute value to the symmetric-difference counts.
\end{proof}

\subsection*{Operational Meaning}

The count
\[
\mathcal{A}(E_1,E_2)= |E_1 \triangle E_2|
\]
is the minimal number of event insertions/deletions needed to transform one local history into the other while preserving the common core.
Under Martin’s Condition, this is precisely the amount of order that must \emph{propagate} across the interface to maintain global consistency.
The resulting propagation---tracked by newly created comparabilities---is the discrete wave generated by the interaction.

\section{First Variation of Amplitude}

The amplitude $\mathcal{A}(E_1,E_2)$ measures the net number of new distinctions
created by the interaction of two causal domains.
The \emph{first variation} describes how that amplitude changes when either
domain gains or loses a single event.
This variation quantifies the local sensitivity of the wave of order.

\begin{definition}[Infinitesimal Variation of an Event Set]
Let $(E,\preceq)$ be a locally finite poset.
An \emph{elementary variation} $\delta E$ is the addition or removal of a single
event $e$ together with its admissible relations that preserve Martin’s Condition:
\[
E' = E \cup \{e\}
\quad\text{or}\quad
E' = E \setminus \{e\},
\qquad
(E',\preceq') \text{ satisfies Martin's Condition.}
\]
\end{definition}

\begin{definition}[First Variation of Amplitude]
Given two interacting domains $E_1,E_2$ and a small perturbation
$E_1' = E_1 \cup \delta E_1$ or $E_2' = E_2 \cup \delta E_2$,
the first variation of the amplitude is
\[
\delta\mathcal{A}
= \mathcal{A}(E_1',E_2) - \mathcal{A}(E_1,E_2)
\quad\text{or}\quad
\delta\mathcal{A}
= \mathcal{A}(E_1,E_2') - \mathcal{A}(E_1,E_2).
\]
Expanding from the definition,
\[
\delta\mathcal{A}
= \bigl| (E_1 \cup \delta E_1) \triangle E_2 \bigr| - |E_1 \triangle E_2|.
\]
\end{definition}

\begin{proposition}[Local Variation Formula]
If $\delta E_1 = \{e\}$ adds a single event $e$ not in $E_2$,
then
\[
\delta\mathcal{A} =
\begin{cases}
+1, & e \notin E_1 \cup E_2, \\[4pt]
-1, & e \in E_2 \setminus E_1, \\[4pt]
0, & e \in E_1 \cap E_2.
\end{cases}
\]
\end{proposition}

\begin{proof}[Proof sketch]
Each event contributes $\pm1$ to the symmetric difference depending on whether
it creates or resolves a unique distinction.
If $e$ is entirely new, the amplitude increases by one.
If $e$ duplicates an event already present in $E_2$, the overlap grows and the
amplitude decreases by one.
If $e$ already exists in both, no new distinguishability is created.
\end{proof}


\begin{proposition}[First Variation as Discrete Derivative]
Let $\mathcal{A}$ be viewed as a function on the lattice of finite subsets of a fixed event universe $\Omega$.
Then the mapping
\[
\delta_e \mathcal{A}(E_1,E_2)
:= \mathcal{A}(E_1\cup\{e\},E_2)-\mathcal{A}(E_1,E_2)
\]
is the discrete directional derivative of $\mathcal{A}$ along $e$.
It satisfies the antisymmetry relation
\[
\delta_e \mathcal{A}(E_1,E_2) = -\,\delta_e \mathcal{A}(E_2,E_1).
\]
\end{proposition}

\begin{proof}[Proof sketch]
Direct expansion using $\mathcal{A}=|E_1|+|E_2|-2|E_1\cap E_2|$
shows that the increment in $E_1$ produces the negative of the increment in $E_2$
for the same event.
Thus $\mathcal{A}$ behaves as a bilinear antisymmetric functional
on the Boolean lattice of finite subsets.
\end{proof}

\noindent
\textbf{Interpretation.}
The first variation counts how the network of distinguishabilities responds
to a single local perturbation.
Adding an event outside the shared overlap increases the amplitude:
a ripple of new order propagates.
Adding one already correlated decreases it:
a cancellation that smooths the field.
Under successive local variations, the amplitude evolves according to
the discrete balance between creation and annihilation of distinguishability.
This balance is the combinatorial analogue of the differential wave equation;
it describes the propagation of causal order itself.

\section{Second Variation of Amplitude}

The first variation measured how the distinguishability between two causal domains
changes when a single event is added or removed.
The \emph{second variation} captures how those incremental changes themselves
interact.
It measures the curvature of distinguishability—the discrete analogue of
acceleration or wave curvature—arising from the mutual influence of two local
perturbations.

\begin{definition}[Second Variation]
Let $\delta_{e}$ and $\delta_{f}$ denote first variations with respect to
elementary event insertions $e$ and $f$.
The \emph{second variation of amplitude} is defined as the symmetric difference
of the corresponding first variations:
\[
\delta^2_{e,f}\mathcal{A}(E_1,E_2)
:= \delta_f\bigl(\delta_e\mathcal{A}(E_1,E_2)\bigr)
= \mathcal{A}(E_1\cup\{e,f\},E_2)
- \mathcal{A}(E_1\cup\{e\},E_2)
- \mathcal{A}(E_1\cup\{f\},E_2)
+ \mathcal{A}(E_1,E_2).
\]
\end{definition}

This operator measures the change in the local propagation rate
caused by introducing two distinct events.
When $\delta^2_{e,f}\mathcal{A}=0$, their effects are independent:
the propagation is linear.
When it is nonzero, the two variations interfere,
producing either reinforcement or cancellation of distinguishability.

\begin{proposition}[Symmetry]
\[
\delta^2_{e,f}\mathcal{A}(E_1,E_2)
= \delta^2_{f,e}\mathcal{A}(E_1,E_2),
\qquad
\delta^2_{e,e}\mathcal{A}(E_1,E_2)=0.
\]
\end{proposition}

\begin{proof}[Proof sketch]
Both $\delta_e$ and $\delta_f$ are finite-difference operators on the Boolean lattice of subsets.
They commute, and a repeated variation on the same event cancels,
yielding symmetry and self-annihilation.
\end{proof}

\begin{proposition}[Explicit Form]
If $e\neq f$ are not contained in $E_2$, then
\[
\delta^2_{e,f}\mathcal{A}(E_1,E_2)
=
\begin{cases}
-2, & e,f\in E_2\setminus E_1, \\[4pt]
+2, & e,f\notin E_1\cup E_2, \\[4pt]
0,  & \text{otherwise.}
\end{cases}
\]
\end{proposition}

\begin{proof}[Proof sketch]
Expand the four amplitude terms in the definition using
$\mathcal{A}=|E_1|+|E_2|-2|E_1\cap E_2|$ and compute the finite difference.
Each event contributes $\pm1$ to the first variation depending on overlap.
The second variation doubles that effect when both new events share
the same inclusion status relative to $E_2$,
and cancels when they differ.
\end{proof}

\begin{definition}[Discrete Laplacian on Event Sets]
Let $\nabla^2_E\mathcal{A}$ denote the sum of all pairwise second variations over
neighboring events in a locally finite causal domain:
\[
\nabla^2_E \mathcal{A}(E_1,E_2)
:= \sum_{\substack{e,f\in E_1\\ e\prec f\text{ or }f\prec e}} 
\delta^2_{e,f}\mathcal{A}(E_1,E_2).
\]
\end{definition}

\begin{proposition}[Wave Equation for Order]
Under Martin’s Condition, the amplitude on any locally finite causal domain
satisfies
\[
\nabla^2_E \mathcal{A} = 0
\]
as the condition for global consistency.
\end{proposition}

\begin{proof}[Proof sketch]
Each pairwise second variation measures the net curvature of distinguishability
between causally related events.
Martin’s Condition enforces that all finite subsets extend consistently,
which requires the total curvature over each closed causal neighborhood to vanish.
Summing over all connected pairs yields $\nabla^2_E\mathcal{A}=0$,
the discrete Laplace equation for order propagation.
\end{proof}

\noindent
\textbf{Interpretation.}
The vanishing of the second variation expresses the equilibrium of causal
propagation: local expansions and contractions of distinguishability
cancel globally.
Where the first variation gave the \emph{slope} of causal change,
the second variation fixes the \emph{curvature}—the shape of the wave.
The condition $\nabla^2_E \mathcal{A}=0$ is therefore the causal-set form
of the homogeneous wave equation:
a statement that information, once created, propagates through the network
of events without net amplification or loss.

\section{Advection as Order-Preserving Transport}

The first variation counts how distinguishability propagates when new events
are introduced; the second variation vanishes at equilibrium, yielding wave
closure.  When propagation is \emph{directional}—because Martin bridges
select a consistent orientation of overlaps along a chain—the resulting
closure is \emph{first–order}: advection.

\subsection*{Setup: a Translation–Invariant Causal Strip}

Let $\Lambda = \{(n,i) : n\in\mathbb{Z},\, i\in\mathbb{Z}\}$ index a locally
finite event strip with “time” levels $n$ (ordinals of measurement steps)
and spatial indices $i$ along a chain of overlaps.  Write $E_n=\{(n,i)\}_i$
and suppose overlaps are oriented so that interaction at level $n$ feeds
level $n\!+\!1$ predominantly from the left neighbor:
\[
(n,i-1)\;\rightarrow\;(n+1,i).
\]
Let $A^n_i\in\mathbb{N}$ denote the \emph{amplitude density} (count of new
distinguishabilities) measured on site $i$ at level $n$.

\begin{definition}[Order–Preserving Transport (Upwind Selection)]
A Martin–consistent, order–preserving update on $\Lambda$ with orientation
to the right is a map $T$ such that
\[
A^{n+1}_i \;=\; (1-\lambda)\,A^n_i \;+\; \lambda\,A^n_{i-1},
\qquad 0\le \lambda \le 1,
\]
with $\lambda$ the \emph{bridge fraction}: the proportion of next–step
distinguishability at $(n\!+\!1,i)$ sourced from the left overlap.
\end{definition}

\noindent
\textbf{Interpretation.}
$\lambda=1$ gives pure shift $A^{n+1}_i=A^n_{i-1}$ (deterministic transport
one site per update). $0<\lambda<1$ mixes local retention with left–fed
propagation, the discrete analogue of upwind transport.  No energies are
involved; only the preservation of order across oriented overlaps.

\subsection*{Discrete Continuity and Characteristics}

\begin{proposition}[Discrete Continuity Law]
For any finite index set $I\subset\mathbb{Z}$,
\[
\sum_{i\in I} A^{n+1}_i - \sum_{i\in I} A^{n}_i
= \lambda\!\left(A^{n}_{\min(I)-1}-A^{n}_{\max(I)}\right).
\]
\end{proposition}

\begin{proof}[Proof sketch]
Telescoping sum of the upwind update across $I$ cancels interior fluxes and
leaves only boundary contributions, expressing conservation of distinguishability
modulo oriented boundary flow.
\end{proof}

\begin{proposition}[Order Characteristics]
If $\lambda=1$, then along lines $i-n=\text{const}$ one has
$A^{n+1}_i=A^n_{i-1}$, hence $A^n_i = A^0_{i-n}$.
Thus distinguishability is constant on the discrete characteristics
$i-n=\mathrm{const}$.
\end{proposition}

\begin{proof}[Proof sketch]
Iterate the shift relation $n$ times.
\end{proof}

\subsection*{Continuum Limit: The Advection Equation}

Let spatial mesh be $h>0$ and step size $\Delta t>0$.
Define a smooth interpolant $a(t_n,x_i)=A^n_i$ with
$t_n=n\,\Delta t$, $x_i=i\,h$, and take
\[
\lambda \;=\; \frac{c\,\Delta t}{h}\quad (0\le \lambda \le 1),
\]
where $c$ is the \emph{order speed} fixed by the oriented Martin bridges.

\begin{theorem}[Advection from Upwind Selection]
Assume $a\in C^2$ and the oriented update
\[
A^{n+1}_i = (1-\lambda)A^n_i + \lambda A^n_{i-1}.
\]
Then, under the scaling $\lambda=\frac{c\Delta t}{h}$ with fixed $c$ and
$\Delta t,h\to 0$ satisfying the Courant condition $0\le \lambda \le 1$,
the interpolant $a$ satisfies
\[
\partial_t a + c\,\partial_x a \;=\; 0 \qquad \text{(advection)}
\]
to first order in $(\Delta t,h)$.
\end{theorem}

\begin{proof}[Proof sketch]
Taylor-expand
$a(t+\Delta t,x)=a+\Delta t\,a_t+\mathcal{O}(\Delta t^2)$ and
$a(t,x-h)=a-h\,a_x+\mathcal{O}(h^2)$, then substitute in
\[
a(t+\Delta t,x)=(1-\lambda)a(t,x)+\lambda\,a(t,x-h).
\]
Divide by $\Delta t$ and use $\lambda=\frac{c\Delta t}{h}$:
\[
a_t + \frac{\lambda}{\Delta t}\bigl(a(t,x-h)-a(t,x)\bigr)
= a_t - \frac{c}{h}\bigl(h\,a_x + \mathcal{O}(h^2)\bigr)
= a_t + c\,a_x + \mathcal{O}(h,\Delta t).
\]
Letting $\Delta t,h\to 0$ yields $\partial_t a + c\,\partial_x a = 0$.
\end{proof}

\subsection*{Order–Theoretic Meaning}

\begin{proposition}[Advection as Oriented Martin Flow]
The advection equation expresses invariance of distinguishability along
order–preserving characteristics $x-ct=\mathrm{const}$ induced by a fixed
orientation of Martin bridges.  Equivalently, for any smooth test function
$\varphi$ compactly supported,
\[
\frac{d}{dt}\!\int a(t,x)\,\varphi(x+ct)\,dx = 0.
\]
\end{proposition}

\begin{proof}[Proof sketch]
Use the weak form of $\partial_t a + c\,\partial_x a = 0$ and integrate
by parts along translated test functions; the quantity is conserved because
propagation is a pure shift along characteristics.
\end{proof}

\subsection*{Remarks on Stability and Causality}

\begin{itemize}
\item \textbf{CFL as Martin Bound.}
$0\le \lambda \le 1$ is exactly the requirement that next–step order at site $i$
is determined by current order from \emph{within} its causal neighborhood,
matching Martin’s Condition (no overreach).

\item \textbf{Asymmetry $\Rightarrow$ Advection.}
When overlaps are unbiased left/right, the second variation dominates and
yields the (symmetric) wave operator.  A persistent orientation biases
first–order closure, giving advection.

\item \textbf{No Energetics.}
All statements concern counts and comparabilities.  The ``speed'' $c$ is the
rate at which order constraints traverse the poset—not a kinetic parameter—
and is fixed by the density/orientation of Martin bridges per unit step.
\end{itemize}

\begin{example}[Thought Experiment: The Knot-Tying Puzzle and Cubic Spline Closure]
\textbf{N.B.} This experiment visualizes why $U^{(4)} = 0$ is the natural limit of causal smoothness.

\emph{Setup.}  
Picture threading a shoelace through a lattice of eyelets representing discrete events.  
Each tie fixes a local order; each pull relates adjacent ties under reciprocity.  
If the path bends beyond third order, new tension points appear that violate global closure.

\emph{Demonstration.}  
Tying loops with 1st–3rd-order curvature yields smooth closure; a 4th-order “wiggle” over-constrains the path, leaving residual curvature.  
Hence the minimal global closure corresponds to cubic continuity: $U^{(4)} = 0$.

\emph{Interpretation.}  
The lattice stands for the causal tensor.  
Minimal curvature ensures bijective mapping of distinctions—no “loose ends.”  
The physical limit is the geodesic; the logical limit is consistency under finite observation.
\end{example}

\section{Adiabatic Transport}
\label{sec:adiabatic-transport}

\textbf{N.B.} This section concerns the preservation of distinguishability under slow, order–preserving transformation of causal structure.  
It defines \emph{adiabatic transport} as informational invariance under continuous refinement, not as a thermodynamic process.  
No physical energy, temperature, or momentum is invoked \cite{landauer1961,jaynes1957,wheeler1983}.

\begin{definition}[Adiabatic Transport]
Let $\mathcal{C}=(E,\preceq)$ be a causal set equipped with a local update rule $T_\lambda : E \to E$ parameterized by a continuous scalar $\lambda$.  
We say that $T_\lambda$ defines \emph{adiabatic transport} if, for every finite subdomain $E_p\subset E$, there exists $\varepsilon>0$ such that for all $|\delta\lambda|<\varepsilon$,
\[
\text{dist}\big(T_{\lambda+\delta\lambda}(E_p),\,T_\lambda(E_p)\big)=0,
\]
where $\text{dist}$ measures distinguishability in the logical sense (i.e., the number of predicate values altered).  
Equivalently, no distinctions are created or destroyed under infinitesimal parameter variation:
\[
\frac{d}{d\lambda}\,\Delta S(E_p;\lambda) = 0.
\]
\end{definition}

\begin{law}[Adiabatic Invariance of Information]
For every causal system evolving under order–preserving refinement, the informational entropy $\Entropy(E_p)$ is invariant under adiabatic transport:
\[
\frac{d}{d\lambda}\Entropy(E_p) = 0,
\]
provided that $\Delta S \ge 0$ globally.  
This expresses the condition that reversible evolution preserves the count of distinguishable states while redistributing them across the parameter domain \cite{boltzmann1872,planck1901}.
\end{law}

\begin{proposition}[Adiabatic Transport as Minimal Update]
Let $U_\lambda$ denote the linear update operator acting on the state vector of local predicates.  
Then $U_\lambda$ is adiabatic iff
\[
U_{\lambda+\delta\lambda} = U_\lambda + \mathcal{O}(\delta\lambda^2),
\]
and $U_\lambda$ is norm–preserving:
\[
\langle U_\lambda x, U_\lambda x \rangle = \langle x, x \rangle,
\]
for all $x$ in the local predicate space.  
Hence adiabatic transport is the first–order limit of informationally reversible evolution—a continuous deformation with zero local entropy production.
\end{proposition}

\begin{proof}[Sketch]
By definition, $\frac{d}{d\lambda}\,\Delta S(E_p;\lambda)=0$ for adiabatic flow.  
Expanding $U_{\lambda+\delta\lambda}$ in $\delta\lambda$ and applying the invariance of the inner product gives
\[
\langle x, (U_{\lambda+\delta\lambda}^\ast U_{\lambda+\delta\lambda} - I)x \rangle = \mathcal{O}(\delta\lambda^2),
\]
so $U_\lambda$ is unitary up to second order, corresponding to zero first–order entropy production.  
This identifies adiabatic transport as the minimal (information-neutral) path in operator space. \qedhere
\end{proof}

\begin{example}[Adiabatic Drift of a Boundary Predicate]
Consider a boundary event $e_b(\lambda)$ parameterized by $\lambda$, marking the interface between recorded and unrecorded regions of a causal set.  
If the mapping $\lambda\mapsto e_b(\lambda)$ is adiabatic, then every increment $\delta\lambda$ changes only the labeling of boundary events, not the count of distinguishable outcomes.  
Thus the observer’s entropy measure remains invariant while the locus of observation drifts—formally, a causal analog of quasi-static expansion in thermodynamics, but without energy exchange.
\end{example}

\emph{Interpretation.}  
Adiabatic transport represents the \emph{limit of causal motion that preserves informational order}.  
It connects reversible evolution (zero $\dot{S}$) with the necessary global condition $\Delta S \ge 0$: the system may drift without loss of information but not contract beyond its recorded distinctions.  
This concept bridges the local invariance of measurement with the global monotonicity of the Second Law of Causal Order.

\emph{Scope.}  
All references to “slow” or “continuous” change are formal, referring to the topology of predicate transformations, not to temporal rate or physical smoothness.  
The adiabatic condition ensures that causal curvature may evolve without violating informational invariance.


\section{Annealing}
\label{sec:annealing}

\textbf{N.B.} This section treats \emph{annealing} as an informational process of relaxation toward maximal distinguishability under causal constraint, not as a physical heat exchange.  
It describes how a causal domain progressively eliminates inconsistent configurations while preserving the monotonic law $\Delta S \ge 0$ \cite{landauer1961,jaynes1957,wheeler1983}.

\begin{definition}[Annealing in the Causal Domain]
Let $\mathcal{C}=(E,\preceq)$ be a causal structure with a family of admissible configurations $\mathcal{H}$ (partial histories) and an informational potential function
\[
\Phi : \mathcal{H} \to \mathbb{R}_{\ge 0},
\]
measuring local inconsistency or ``tension'' between distinguishable records.  
A sequence $\{h_t\}_{t \in \mathbb{N}} \subset \mathcal{H}$ is said to undergo \emph{annealing} if
\begin{enumerate}
  \item $\Phi(h_{t+1}) \le \Phi(h_t)$ for all $t$ (monotonic relaxation), and
  \item $\lim_{t\to\infty}\Phi(h_t)=0$ corresponds to a globally consistent configuration $h_\infty$ satisfying the Event Selection Axiom.
\end{enumerate}
\end{definition}

\begin{law}[Informational Annealing Law]
A causal domain subject to finite observation evolves toward a configuration minimizing informational tension while preserving distinguishability:
\[
\frac{d\Phi}{dt} \le 0, \quad \frac{dS}{dt} \ge 0,
\]
with equality only at global consistency ($\Phi=0$).  
Thus annealing is the formal complement of adiabatic transport: instead of preserving information exactly, it monotonically reorganizes distinctions to reduce redundancy without loss of order \cite{boltzmann1872,planck1901}.
\end{law}

\begin{proposition}[Annealing as Iterated Projection]
Let $U_t$ be the local update operator at iteration $t$, and let $\Pi$ denote the projector onto the subspace of order–consistent states.  
If
\[
h_{t+1} = \Pi(U_t h_t),
\]
then the sequence $\{h_t\}$ is annealing iff the composition $\Pi U_t$ is contractive in the informational metric:
\[
\|\Pi U_t h - \Pi U_t k\| \le \|h - k\| \quad \forall h,k \in \mathcal{H}.
\]
Contractivity ensures that successive refinements move closer to a consistent fixed point while never decreasing the total number of distinguishable predicates.  
\end{proposition}

\begin{proof}[Sketch]
Under contractivity, Banach’s fixed–point theorem guarantees convergence to a unique $h_\infty = \Pi U_\infty h_\infty$.  
Because $\Pi$ only removes logical contradictions and never merges distinct equivalence classes, $S(h_{t+1}) \ge S(h_t)$, while $\Phi(h_{t+1}) \le \Phi(h_t)$.  
Hence the dual monotonic conditions above are satisfied. \qedhere
\end{proof}

\begin{example}[Causal Annealing as Refinement of Observation]
Suppose an observer repeatedly reconciles conflicting records of events $e_i$ and $e_j$ under finite resolution.  
Each reconciliation step eliminates one inconsistency in ordering while maintaining all prior distinctions.  
As $t$ increases, the observer’s causal field ``anneals'' toward a self-consistent partial order: the informational potential $\Phi$ decreases, but $\Delta S$ remains nonnegative.  
The limit corresponds to the observer’s stable record—a causal equilibrium in the informational sense.
\end{example}

\emph{Interpretation.}  
Adiabatic transport corresponds to reversible motion along an iso-entropic surface in informational phase space; annealing describes the relaxation onto that surface.  
Both obey $\Delta S \ge 0$: adiabatic evolution keeps $\Delta S = 0$ locally, annealing makes $\Delta S > 0$ globally until consistency is achieved.  
Together, they constitute the two limiting behaviors of informational dynamics: perfect preservation and convergent refinement.

\emph{Scope.}  
The term ``temperature'' is deliberately avoided; no thermodynamic substrate is implied.  
Annealing here refers solely to the controlled reduction of inconsistency within the logical structure of events, ensuring that the system approaches global causal coherence without violating the monotonic growth of distinguishability.

\section{Brownian Motion}
\label{sec:brownian-motion}

\textbf{N.B.} This section presents \emph{Brownian motion} as the limit of informational diffusion within a causal set, not as molecular kinetics.  
It formalizes randomness as the unresolved superposition of admissible distinctions under finite observation.  
The stochastic character arises from incomplete causal specification, not from physical noise \cite{einstein1905,smoluchowski1906,jaynes1957,wheeler1983}.

\begin{definition}[Brownian Motion in the Causal Framework]
Let $(E,\preceq)$ be a locally finite causal domain and $X_t : E \to \mathbb{R}^n$ a random field assigning numerical observables to events at discrete rank $t$.  
We call $\{X_t\}$ a \emph{Brownian process on the causal domain} if, for every finite set of events $F \subset E$,
\[
\mathbb{E}[X_{t+1}(e) - X_t(e)] = 0, 
\quad 
\mathbb{E}[(X_{t+1}(e) - X_t(e))^2] = 2D\,\delta t,
\]
where $D$ is the informational diffusion coefficient and $\delta t$ the discrete causal increment.  
The expectation is taken over the ensemble of admissible event refinements consistent with the observer’s finite resolution.
\end{definition}

\begin{law}[Causal Diffusion Law]
Let $\rho(e,t)$ denote the probability density over distinguishable states at event $e$ and causal rank $t$.  
Then informational conservation under local diffusion implies
\[
\frac{\partial \rho}{\partial t} 
= D\,\nabla^2 \rho,
\]
where the Laplacian $\nabla^2$ acts on the combinatorial adjacency of $E$.  
This is the causal analogue of the classical diffusion equation, expressing the randomization of distinctions subject to $\sum_e \rho(e,t)=1$.
\end{law}

\begin{proposition}[Variance Growth and Informational Pressure]
Under the Causal Diffusion Law,
\[
\mathbb{E}[(X_t - X_0)^2] = 2 D t.
\]
Thus the variance of informational displacement grows linearly with causal depth.  
Equivalently, uncertainty in the record expands at a constant informational ``pressure'' proportional to $D$.  
This represents the minimal stochastic perturbation compatible with $\Delta S \ge 0$: entropy increases strictly due to unresolved refinement of order.
\end{proposition}

\begin{proof}[Sketch]
Applying the law above to the mean and variance gives $\frac{d}{dt}\mathbb{E}[X_t]=0$ and $\frac{d}{dt}\mathbb{E}[X_t^2]=2D$.  
Integrating yields the linear variance relation.  
Because the process is unbiased and conservative, entropy increases monotonically while total distinguishability (probability normalization) remains invariant. \qedhere
\end{proof}

\begin{example}[Informational Brownian Motion]
Consider an observer recording the position of a causal signal with finite temporal resolution $\delta t$.  
Each interval introduces an unresolved set of micro–orderings among sub–events, which appear as random displacements of the observed value $X_t$.  
As the record refines, the unresolved components average to zero drift but nonzero variance, giving the signature of Brownian motion.  
The diffusion constant $D$ measures the observer’s coarse–graining bandwidth: smaller $D$ corresponds to higher informational precision.
\end{example}

\emph{Interpretation.}  
Brownian motion represents the \emph{limit of stochastic refinement}—the regime where information is preserved on average but redistributed unpredictably within causal bounds.  
Where adiabatic transport is reversible ($\dot S=0$) and annealing directed ($\dot S>0$ with convergence), Brownian evolution is \emph{neutral on average} yet entropic in expectation ($\mathbb{E}[\Delta S]>0$).  
It captures the statistical character of finite observation: even without physical noise, the logical uncertainty of incomplete distinction propagates diffusively.

\emph{Scope.}  
All references to ``randomness'' or ``diffusion'' are formal.  
The process described is informational, not mechanical; it models how a finite observer’s record spreads under repeated partial measurement, consistent with the Second Law of Causal Order.


\begin{example}[Thought Experiment: The Dual Transport of Measurement]
\textbf{N.B.} This thought experiment explores the formal coexistence of discrete and continuous transport under finite observation.  
It does not describe quantum dynamics or wave mechanics.  
It serves to illustrate how an informational system can exhibit both particle–like localization (discrete event selection) and wave–like propagation (distributed causal amplitude) within the same logical framework \cite{bohr1928,heisenberg1927,wheeler1983}.

\emph{Setup.}  
Consider an observer maintaining a causal record $\mathcal{R}$ of distinguishable events $\{e_i\}$.  
Each update corresponds to an informational transport operator $U_t$, which may act:
\begin{enumerate}
  \item \emph{Discretely}—selecting a new event and appending it to $\mathcal{R}$ (particle limit), or
  \item \emph{Continuously}—refining the relational amplitudes between existing events (wave limit).
\end{enumerate}
In a regime of finite observation, these two descriptions coexist:  
the discrete record $e_i$ is a sample of a continuous causal amplitude that itself evolves under adiabatic, annealing, or Brownian mechanisms.

\emph{Invariant to Preserve.}  
For each admissible transformation, the informational flux
\[
\Phi(t) = \sum_i \rho_i(t)\,v_i(t)
\]
remains conserved in expectation.  
Here $\rho_i$ is the probability weight of distinguishing $e_i$ at time $t$, and $v_i$ the causal update rate.  
Conservation of $\Phi(t)$ enforces complementarity: localization increases $\rho_i$ at the expense of amplitude spread (wave), while dispersion equalizes $\rho_i$ (particle uncertainty).  
The two limits are dual descriptions of the same informational invariant.

\emph{Formal Analogy.}  
Let $\Psi$ denote the normalized vector of causal amplitudes across events.  
Under reversible (adiabatic) transport,
\[
U_t\,\Psi = e^{iH t}\Psi,
\]
and under irreversible (annealing or Brownian) refinement,
\[
\Psi_{t+1} = (I - \epsilon L)\Psi_t,
\]
where $L$ is the informational Laplacian on the causal network.  
The “wave” view treats $\Psi$ as a continuous amplitude over possible distinctions;  
the “particle” view corresponds to a single index realization of $\Psi$ by the observer.  
Both are faithful encodings of the same causal process at different resolutions.

\emph{Interpretation.}  
Wave–particle duality thus arises as an epistemic phenomenon of finite observation:  
the wave is the uncollapsed ensemble of admissible distinctions,  
the particle the locally recorded choice among them.  
Observation projects $\Psi$ onto one branch of $\mathcal{R}$, satisfying the Event Selection Axiom but leaving the global informational flux invariant.  
In this sense, the collapse of the wave function is not physical, but the act of registering one consistent causal trajectory within the set of possible orderings.

\emph{Scope.}  
This thought experiment is purely formal.  
It demonstrates how the coexistence of discrete and continuous transport mechanisms follows from the structure of causal information, not from any underlying quantum field.  
Wave–particle duality here is the logical complementarity of refinement and selection—the informational shadow of the observer’s bandwidth limit.
\end{example}


\section{On Deriving Motion Without Energy}

The developments up to this point have been intentionally austere.
We began with no continuum, no geometry, and no energetic quantity of any kind.
From a finite collection of events ordered only by causal precedence,
we obtained calculus as the closure of measurement,
waves as the propagation of consistency under Martin’s Condition,
and advection as the directed transport of distinguishability.
At no step was energy invoked.
Nothing in the construction presupposed force, mass, or curvature.
Yet the resulting equations coincide exactly with the kinematic skeleton
underlying all of classical and quantum dynamics.

\subsection*{The Structural Consequence}

The advection equation,
\[
\partial_t a + c\,\partial_x a = 0,
\]
arose not from the motion of particles through a medium,
but from the preservation of order across oriented overlaps of finite event sets.
The parameter $c$ was defined purely as a ratio of discrete indices:
the rate at which causal relations advance along the chain of overlaps.
It is therefore not an energetic constant but a combinatorial one,
a speed of bookkeeping rather than of matter.
This reversal of interpretation is decisive.
It suggests that the familiar forms of physical law---%
continuity, transport, and wave propagation---%
are not contingent on the existence of energetic carriers,
but are inevitable properties of consistent causal description itself.

\subsection*{The Logical Hierarchy of Physics}

The chain of constructions may now be summarized as
\[
\text{Order} \;\Longrightarrow\; \text{Variation} \;\Longrightarrow\; 
\text{Propagation} \;\Longrightarrow\; \text{Energy}.
\]
Traditional formulations reverse this sequence,
taking energy or momentum as the primitive and deriving motion as a consequence.
Here motion appears first, as a necessary regularity of finite order.
Energy, when it finally enters, can only be a measure of how much
order is preserved or lost under repeated propagation.
What physicists call \emph{kinetic} or \emph{potential} energy
must therefore correspond to the count of distinguishabilities that remain invariant
under the oriented application of Martin’s Condition.
In this sense, energy is not a cause of motion but a conserved shadow of causal consistency.

\subsection*{The Epistemic Reversal}

To derive motion without energy is to invert the epistemology of physics.
It means that the universe does not move because it has energy;
it \emph{has} energy because its order moves.
Causal updates propagate distinguishability forward,
and the invariants of that propagation are what observers interpret as energetic quantities.
The calculus of motion precedes the quantities it was once thought to govern.
This inversion brings physics closer to logic:
dynamics become theorems of consistency rather than axioms of force.

\subsection*{Consistency as the Source of Dynamics}

Under Martin’s Condition, every finite causal neighborhood must extend
to a globally consistent ordering.
When overlaps are unbiased, this requirement produces the symmetric
second--order closure $\nabla^2_E\mathcal{A}=0$,
the discrete wave equation.
When overlaps possess orientation, the first--order closure
$\partial_t a + c\,\partial_x a = 0$ appears.
Both are special cases of the same law:
\begin{law}{Law of Consistency}
The universe minimizes the inconsistency of its own order.
\end{law}
The entire machinery of classical dynamics---waves, advection, diffusion,
and, later, curvature and field stress---can therefore be interpreted as successive
approximations to the global enforcement of Martin’s Condition.
Every differential operator is a bookkeeping device for maintaining consistency
in the face of finite, overlapping observations.

\subsection*{Implications}

This interpretation carries several consequences:

\begin{enumerate}
\item \textbf{Causality precedes energy.}
Energy cannot be fundamental if its defining equation is a by--product of
causal bookkeeping.
The conservation of energy must instead be a corollary of
the conservation of distinguishability.

\item \textbf{Geometry is emergent.}
Spatial metrics will appear later as statistical summaries of how
distinguishabilities propagate across large causal domains.
Space is the coarse--grained shadow of consistent order.

\item \textbf{The field concept is derivative.}
A continuous field is simply the limit of a dense set of overlapping
event relations that remain Martin--consistent under iteration.
Field equations are encoded constraints on the propagation of order.

\item \textbf{Information and physics coincide.}
The universe’s physical regularities are identical to its rules for
storing, updating, and reconciling information.
No extra ontology is required.
\end{enumerate}

\subsection*{Outlook}

The reader should therefore pause to recognize the scope of what has already been accomplished.
Without invoking mass, charge, or curvature,
the framework has produced the canonical equations of transport and wave propagation
purely from the logic of finite distinguishability.
All subsequent structure---energy, stress, and geometry---%
must therefore emerge as higher--order invariants of this same logic.
The remainder of this work develops those invariants explicitly,
showing how the metric tensor, stress tensor, and curvature of spacetime
are the continuous shadows of a discrete causal calculus.

\begin{example}[Extrapolation: The Casimir Effect as Boundary-Limited Distinction]
\textbf{N.B.} This extrapolation is formal. It shows how vacuum energy differences emerge as a consequence of boundary-imposed limits on distinguishability, not as a mechanical “force” between conducting plates.

\emph{Setup.}
In quantum field theory, the Casimir effect arises when two parallel plates constrain the allowable electromagnetic modes between them, yielding a measurable pressure proportional to $1/d^4$, where $d$ is the separation.
Conventionally, this is interpreted as the difference between zero-point energies inside and outside the cavity:
\[
F/A = -\frac{\pi^2 \hbar c}{240 d^4}.
\]

\emph{Formal interpretation.}
In the informational framework, each admissible field mode corresponds to a distinguishable causal update.
When boundary conditions restrict the available modes, the count of consistent updates between the plates decreases relative to the unbounded field.
The universe enforces $\Delta S \ge 0$ by compensating this local loss with an outward pressure—an adjustment that restores global distinguishability.

\emph{Analogy.}
The Casimir pressure is thus an informational curvature: the gradient of admissible state density imposed by constraints in the causal manifold.
It represents the same reciprocity principle found in adiabatic transport and annealing—an equilibrium restoring term that keeps the overall measure consistent.

\emph{Scope.}
This extrapolation demonstrates that apparent “vacuum energy” phenomena can be reinterpreted as rebalancing within the global distinction count.
No change to established quantum electrodynamics is implied; the Casimir effect merely exemplifies how informational consistency reproduces the observed form of boundary-induced pressure.
\end{example}

\begin{example}[Extrapolation: Quantum Tunneling as the Repair of a Broken Correlation]
\textbf{N.B.} This extrapolation is formal. It treats tunneling as the restoration of informational continuity across a boundary where the causal correlation appears classically broken. No claim about subatomic mechanism is made.

\emph{Setup.}
In conventional quantum mechanics, a particle encountering a potential barrier of height $V_0$ and width $d$ exhibits a finite transmission probability even when its energy $E < V_0$.  
The standard explanation attributes this to the non-zero overlap of an exponentially decaying wavefunction across the barrier:
\[
T \propto e^{-2\kappa d}, \quad \kappa = \frac{\sqrt{2m(V_0-E)}}{\hbar}.
\]
Classically, the trajectory would terminate; quantum mechanically, correlation persists.

\emph{Formal interpretation.}
In the causal framework, measurement defines correlations—ordered distinctions that must remain globally consistent under $\Delta S \ge 0$.  
When a barrier divides the causal manifold, the mapping between pre- and post-barrier distinctions momentarily fails: a “broken correlation.”  
To preserve informational continuity, the universe extends the correlation through the barrier as a minimal repair—an *extrapolated distinction* that restores bijection.  
The exponential attenuation reflects the cost of maintaining that minimal correlation across a region forbidden to classical propagation.

\emph{Analogy.}
Brownian motion represents spontaneous fluctuation within consistent order; tunneling represents the inverse—the system’s attempt to prevent discontinuity in the causal mapping.  
It is the informational equivalent of a snapped thread reconnecting under tension: continuity is statistically unlikely but globally mandatory.

\emph{Scope.}
This extrapolation is formal. It shows that tunneling can be viewed as the minimal act required to prevent a breach of causal order—an informational necessity that ensures no correlation, once established, terminates inconsistently.  The observed transmission coefficient quantifies the “cost” of that repair.
\end{example}

\begin{example}[Extrapolation: The Hyperfine Transition as Periodic Reconciliation of Causal Order]
\textbf{N.B.} This extrapolation is formal. It interprets hyperfine transitions as periodic acts of informational reconciliation between nearly degenerate causal configurations, not as dynamical spin flips.

\emph{Setup.}
In the hydrogen atom, the ground state comprises two configurations distinguished only by the relative orientation of the proton and electron spins.  
The energy difference,
\[
\Delta E = h\nu_{21} \approx 5.9\times10^{-6}\ \mathrm{eV},
\]
corresponds to the 21\,cm line ($\nu_{21}=1.420\,405\,751\ \mathrm{GHz}$).  
Quantum electrodynamics treats this as the magnetic dipole interaction between two spin moments.

\emph{Formal interpretation.}
Within the causal framework, the two spin configurations represent adjacent but distinct informational states—nearly indistinguishable under ordinary resolution.  
Over time, their causal descriptions drift out of phase due to accumulated measurement asymmetry.  
The spontaneous emission of a 21 cm photon restores alignment: it is a *reconciliation event* that re-synchronizes the dual description and preserves the global bijection required by $\Delta S \ge 0$.

\emph{Analogy.}
Where tunneling repaired a local breach in continuity, the hyperfine transition performs global bookkeeping.  
It is the metronome of causal coherence—a self-correcting tick ensuring that near-identical representations do not diverge beyond the system’s tolerance.  
Each photon marks the closure of an informational loop, a return to consistency.

\emph{Scope.}
This extrapolation is formal.  It demonstrates how spontaneous emission at the hyperfine scale can be viewed as the minimal act by which the universe re-aligns nearly degenerate causal states.  
The 21 cm line thus records the maintenance of informational equilibrium, not merely atomic magnetism.
\end{example}

\begin{example}[Extrapolation: Correlation Drift and the Redshift of Atomic Clocks]
\textbf{N.B.} This extrapolation is formal. It treats redshift as a change in the density of consistent correlations rather than a mechanical stretching of spacetime.

\emph{Setup.}
Atomic clocks mark the periodic reconciliation of nearly degenerate states—the hyperfine transition.  
When two clocks occupy regions with differing correlation densities $\rho_c$, the rate of reconciliation differs:
\[
\frac{\nu}{\nu_0} = \frac{\rho_c}{\rho_{c,0}}.
\]
A deficit of correlation (e.g., under gravitational potential) yields $\nu < \nu_0$, the observed redshift.

\emph{Interpretation.}
In this framework, gravitational redshift and time dilation emerge as manifestations of informational shear: the cost of preserving bijective correspondence between causal regions of unequal correlation density.  
The “slowing” of time is the reduced frequency of global consistency updates.

\emph{Scope.}
This extrapolation links frequency drift to the correlation structure of the causal field.  
It demonstrates that atomic clocks red-shift because their reconciliation rate is modulated by correlation density, not because time itself dilates.
\end{example}


\cleardoublepage
\section*{Coda: The Informational Harmonic Oscillator}
\addcontentsline{toc}{section}{Coda: The Informational Harmonic Oscillator}

\begin{example}[Coda: The Informational Harmonic Oscillator]
\textbf{N.B.} This coda introduces the simplest reversible dynamics permitted by the Second Law of Causal Order.  
It shows that oscillation—in its most abstract sense—arises whenever information alternates between two complementary forms: record and prediction.  
No physical mass, force, or energy is implied.  
The oscillator here is entirely informational: a minimal closed loop of distinguishability \cite{planck1901,landauer1961,wheeler1983}.

\emph{Setup.}  
Let $(x,p)$ denote a conjugate pair of informational coordinates on a two–dimensional causal phase space.  
$x$ represents the observer’s recorded distinctions (the state of knowledge);  
$p$ represents the predictive momentum (the rate at which distinctions are changing).  
Define the informational Hamiltonian
\[
\Entropy(x,p) = \tfrac{1}{2}\,(\alpha x^{2} + \beta p^{2}),
\]
where $\alpha$ and $\beta$ are positive constants measuring informational stiffness and inertia.  
The reversible evolution equations are
\[
\dot{x} =  \phantom{-}\frac{\partial \Entropy}{\partial p} = \beta p,
\qquad
\dot{p} = -\frac{\partial \Entropy}{\partial x} = -\alpha x.
\]
Eliminating $p$ yields
\[
\ddot{x} + \omega^{2} x = 0, \qquad \omega^{2} = \alpha\beta.
\]
Thus the observer’s state executes harmonic motion in informational phase space with constant total measure $\Entropy(x,p)$.

\emph{Interpretation.}  
At each turning point of the oscillation, information is maximally localized: the record $x$ is fixed, the prediction $p$ vanishes.  
At each midpoint, prediction dominates and the record is momentarily indeterminate.  
The system alternately \emph{stores} and \emph{transmits} distinguishability, maintaining constant total informational entropy.  
The cycle expresses the complementarity of knowledge and expectation: every complete measurement must eventually swing back toward uncertainty to preserve $\Delta S \ge 0$.

\emph{Relation to Transport.}  
The four informational transport regimes appear as limiting cases of this oscillator:
\begin{itemize}
  \item \textbf{Adiabatic transport:} small, continuous oscillations with $\dot S = 0$ (reversible exchange of information).
  \item \textbf{Annealing:} inclusion of damping $\gamma > 0$, giving $\ddot{x} + \gamma\dot{x} + \omega^{2}x = 0$, a monotonic relaxation to equilibrium.
  \item \textbf{Brownian motion:} addition of stochastic forcing $\xi(t)$, $\ddot{x} + \omega^{2}x = \xi(t)$, producing diffusive variance growth.
  \item \textbf{Wave/particle duality:} interpretation of $(x,p)$ as the conjugate pair of localization and amplitude—dual views of the same informational invariant.
\end{itemize}
Each regime preserves causal consistency and satisfies $\Delta S \ge 0$; only the mode of information exchange changes.

\emph{Scope.}  
This oscillator is the canonical closed system of the informational universe:  
a bounded transformation in which every increase in record precision is offset by a proportional loss of predictive capacity, and vice versa.  
It represents the minimal rhythm of causal order—the reversible heartbeat of information itself.
\end{example}


\begin{center}
\textit{Motion, in this theory, is not caused by energy.  
It is the preservation of order under Martin’s Condition.}
\end{center}

