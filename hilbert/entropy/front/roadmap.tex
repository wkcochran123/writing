\chapter*{Roadmap}
\addfiletotoc{Roadmap}


This work begins from the simplest possible assumption: every observation
creates information. When something becomes distinguishable, it has been
measured. Each measurement leaves a record, and the universe grows by
accumulating these records one event at a time.

Nothing continuous is assumed. No geometry, no fields, no action
principle, and no differential equations. The goal is to show that the
familiar structures of physics arise automatically when we ask for one
thing: a universe whose measurements never contradict each other.

\section*{From Events to Smooth Motion}

Imagine that only a finite number of measurements are made along a
particle's path. Between those points, many curves are mathematically
possible, but most of them would imply extra structure: hidden bumps,
oscillations, or accelerations that would have produced additional
measurements. Since no such measurements exist, those curves must be
rejected.

The only admissible path is the one with no unobserved structure. In
practice this means the path of least bending: the same rule that defines
cubic splines. As measurements become more precise, the spline becomes
smooth, and its smooth limit satisfies a fourth–order equation

\[
U^{(4)} = 0 .
\]

This is the Euler--Lagrange equation of a free beam. Here it appears
not because an action was postulated, but because any other path would
predict unrecorded motion. Smooth calculus is the shadow of finite
measurements.

\section*{Matter as Information Flow}

If two observers measure the same system and later compare data, their
descriptions must agree in the overlapping region. This is only possible
if information flows continuously from one measurement to the next. In
practice, the density of recorded information obeys a continuity law

\[
\frac{\partial}{\partial t}\rho + \nabla \cdot j = 0 .
\]

This is the kinematics of matter: information flows in a way that does
not lose or invent measurements.

\section*{Light as the Fastest Information}

There is a maximum rate at which new information can appear. In relativity
this is the speed of light. Here it is simply the rule that no observer can
receive new data from outside a causal boundary. When this idea is carried
into the smooth limit, the result is the wave equation

\[
\frac{\partial^2 U}{\partial t^2} = c^2 \frac{\partial^2 U}{\partial x^2} .
\]

Light is what information looks like when it travels at the fastest
possible speed. Waves are the way measurements spread when there is no
hidden structure in between.

\section*{Quantum Information}

When two alternatives cannot be distinguished by any measurement, they
cannot be assigned separate probabilities. Instead, they combine as
amplitudes. Only when a new measurement is made do they separate into
exclusive outcomes. This is the reason amplitudes add and probabilities
come from their square.

In this framework, quantum mechanics is the bookkeeping of situations
where multiple explanations remain consistent with all available data.
Interference is the comparison of indistinguishable histories.

\section*{Gauge Theory}

Different observers may use different conventions to describe the same
measurements. If their predictions always agree, then their descriptions
must be related by a transformation that leaves the measurable results
unchanged. These transformations form a gauge symmetry, and the rules for
comparing neighboring descriptions produce a connection. Its curvature is
what physicists call a field.

Thus gauge theory is not an extra piece of physics. It is the condition
that descriptions remain compatible when information is transported from
place to place.

\section*{Why Entropy Increases}

Every measurement increases the number of possible histories that remain
consistent with the data. None of those histories can be removed, because
that would erase information already recorded. As a result, the number of
distinguishable configurations never goes down. Define entropy as

\[
S = \log(\text{number of consistent configurations}).
\]

Then $S$ can only increase. This is the Second Law of Causal Order:
information grows because measurements cannot be undone.

\section*{Summary}

\begin{center}
\begin{tabular}{lll}
Structure & Why it appears & Result \\
\hline
Smooth paths & No hidden motion & $U^{(4)} = 0$ \\
Matter flow & Shared measurements & Continuity law \\
Light & Fastest information & Wave equation \\
Quantum & Indistinguishable causes & Amplitudes, interference \\
Gauge fields & Consistent comparison & Connections, curvature \\
Entropy & Measurements are permanent & $\Delta S \ge 0$ \\
\end{tabular}
\end{center}

In this view, physics is not assumed. It is what information must look like
when measurements never contradict each other.

