\clearpage
\thispagestyle{empty}

\vspace*{\fill}        % push downward from top

\begin{center}
\fbox{%
\parbox{0.9\linewidth}{
\centering
\textbf{No differential equations were altered, reinterpreted, or otherwise harmed in the production of this proof.}\\[3pt]
This work treats measurement as a discrete, logical process.
Continuum formulations appear only as smooth limits of countable constructions,
never as physical postulates.
}}
\end{center}



%\vspace{25pt}
%\begin{center}
%\fbox{%
%\parbox{0.9\linewidth}{
%\centering
%\textbf{Martin's Axiom is not required to explain any informational phenomenon in this
%manuscript.} 
%%
%None of the propositions, laws, or consequences derived in the
%main text depend on Martin's Axiom, nor on any uncountable construction.  
%All informational behaviors---motion, transport,
%curvature, strain, symmetry, and the increase of admissible configurations---are
%proved entirely within the countable refinement structure guaranteed by the
%Axiom of Boltzmann.
%
%The role of Martin's Axiom in this work is interpretive rather than
%foundational.  It provides a convenient intuition for understanding the
%logical completion of thought experiments: if a refinement can proceed
%locally without contradiction, then it may be extended step by step to a
%globally coherent history.  This is precisely the informal reasoning we use
%when imagining continuous motion, smooth fields, or refined measurements in
%a conceptual model.  The axioms that govern the theory itself require only
%the countable closure of refinements and the absence of unobserved structure.
%Martin's Axiom simply clarifies how such thought experiments can be read as
%dense limits of admissible, finite event selections.  No phenomenon described
%in this book depends on it.}}
%\end{center}


\vspace*{\fill}        % push upward from bottom

\clearpage


