\chapter*{\emph{Nota Bene}}
\addfiletotoc{\emph{Nota Bene}}


The argument is constructive rather than interpretive.  
Each part extends the previous one by a single act of closure that preserves causal consistency:
\[
\text{Measurement} \;\Rightarrow\; \text{Calculus} \;\Rightarrow\; \text{Wave} \;\Rightarrow\; \text{Geometry} \;\Rightarrow\; \text{Field}.
\]

At every stage, a new invariant appears whenever distinction is preserved under refinement.  
The sequence therefore builds the minimal structure required for a universe that records its own evolution without contradiction.

Thus, the proof is read not as a series of analogies but as a chain of logical consequences.  
Starting from finiteness, order, and choice, one obtains measurement, variation, and their 
reciprocity; from reciprocity, one obtains calculus; from calculus, the smooth invariants of 
physics; and from their global consistency, the Second Law of Causal Order.  
In this sense, $\Delta S \ge 0$ is the unique fixed point of mathematics and physics---the 
inequality that any self-consistent universe must obey.

\NB{The proof contains multiple conceptual examples to help explain the mathematical
machinery.  These \emph{thought experiments} are not empirical illustrations
but formal constructions intended to clarify the logical structure of the axioms.
They are finite conceptual models that demonstrate how the mathematical
relations behave under specific constraints of order and measurement.  
No claim is made regarding physical observation; each serves only to illuminate
the internal mechanics of the theory.}


\NB{Throughout what follows, it is essential to distinguish the logical structure of
measurement from any claim about physical phenomena.  
The arguments presented here concern the internal consistency of \emph{records of distinction}---that is,
the admissible transformations among measurable events---rather than the evolution of material
systems themselves.  
Every symbol, tensor, and variation in the proof refers to relations between observations,
not to unobserved substances or causes.  
The framework thus formalizes the mathematics of \emph{measurement}:
how distinctions can be made, counted, and related without contradiction.
No ontological or dynamical claims are implied; the results hold regardless of what,
if anything, the symbols may represent physically.}

\NB{This is a paper about \emph{information}, not about energy, momentum, 
or any other physical quantity.  
At no point is it suggested that such values are produced, derived, or 
generated by the constructions presented here.  
All arguments concern the logical structure of measurement and the internal 
coherence of distinguishability, not the dynamics of physical systems.}

\NB{This is a \emph{conditional} proof.  
All conclusions hold only under the stated axioms and definitions.  
No claim is made regarding the physical truth of those assumptions,  
only their internal consistency and the consequences that follow from them.}

\NB{Certain sections include extrapolations that relate the formal results to observable patterns 
(e.g., galactic rotation, Cepheid-based distance scaling, or cosmic expansion). 
These are illustrations of consistency, not physical hypotheses. 
Their purpose is to show that familiar empirical regularities may follow naturally 
from the informational constraints proved herein. 
No specific mechanism is proposed, and no claim of empirical verification is implied. 
They are included solely to outline how the formal results may frame, rather than predict, observable regularities.}


