\chapter*{Abstract}
\addcontentsline{toc}{section}{Abstract}

We present a set of axioms describing how measurements must correspond in order
for a universe to draw coherent conclusions about itself. Each act of
measurement creates a finite, distinguishable event, and every admissible
extension of that record must preserve global consistency. This requirement is
formalized by a non-standard application of Martin's Condition, which restricts
how observational data may refine without introducing contradictions or
unrecorded structure.

Under these axioms, Martin's Condition admits a unique smooth closure in which
discrete refinements converge to continuous, spline-level dynamics. Classical
differential equations, Hilbert structures, gauge symmetries, curvature, 
transport laws, and geometric tensors are admitted only when the measurement
record allows a smooth completion consistent with these constraints. None of
these structures are postulated; they appear only when the data permit them.

The resulting framework reproduces the familiar behaviors of physics: waves,
matter flow, diffusion, advection, geodesic motion, quantization, and curvature,
without assuming fields, geometry, or dynamical laws in advance. Most
importantly, the axioms imply that entropy can never decrease. The inequality
\[
\Delta S \ge 0
\]
is proved as a theorem of causal measurement: any universe consistent with its
own record of distinguishable events must increase the number of admissible
configurations. No thermodynamic assumptions or mechanisms are required.

Thus, the familiar laws of physics appear wherever the record of measurement
admits a smooth and globally consistent refinement.

