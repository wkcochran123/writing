\chapter*{Abstract}
\addcontentsline{toc}{chapter}{Abstract}

This work shows that measurement itself defines the metric structure of physics: each act of distinction generates an increment of causal order. From this premise—that every measurement refines the universe’s record of what can be distinguished—it follows that information, and thus entropy, can never decrease. Within standard set theory, this principle is proved as a theorem of causal consistency rather than assumed as a thermodynamic postulate.

We present a constructive proof that the entropy of any causally consistent universe is non-decreasing, $\Delta S \ge 0$. Within the axioms of Zermelo–Fraenkel set theory with Choice, we define a finite causal order of distinguishable events whose reciprocal operations—measurement and variation—form a dual pair under the \emph{Reciprocity Law of Physics}. Each measurement counts distinctions; each variation relates them. Their bijection guarantees that information cannot decrease under any admissible extension of order.

Requiring global coherence under Martin’s Axiom enforces the fourth--order cancellation $U^{(4)} = 0$, identifying the cubic spline as the minimal analytic closure of the dual system. This closure produces the continuous calculus of variations as the smooth limit of finite causal measurement, and its algebraic dual defines the discrete logic of event selection. From this structure, the invariants of physics emerge successively: the wave equation as the propagation of reciprocal consistency, the metric as its gauge, and curvature as the residue of its global non-closure. 
Operationally, entropy measures the logarithm of the number of admissible distinguishable configurations consistent with the causal order.  When the causal field is interpreted as this count of distinctions, its curvature encodes the rate at which distinguishability grows.
Coupling the causal field to entropy yields a constant-curvature stress tensor that defines the gravitational scale.

Thus, $\Delta S \ge 0$ is not a thermodynamic postulate but a theorem of causal measurement: the necessary condition that any universe consistent with its own record of distinctions must increase the count of what can be known.


