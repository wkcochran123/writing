\documentclass[12pt]{article}
\usepackage{amsthm,amssymb,amsmath,mathtools,setspace}
\onehalfspacing

% --- Module Dependencies (Must align with main P.tex definitions) ---
\newtheorem{theorem}{Theorem}
\newtheorem{proposition}{Proposition}
\newcommand{\U}{\mathbf{U}}        % Universe tensor
\newcommand{\Uprime}{\mathbf{U}'}
\newcommand{\Udoubleprime}{\mathbf{U}''}
\newcommand{\grad}{\nabla}
\newcommand{\Lag}{\mathcal{L}}

\title{\textbf{Module 1/2 Bridge: Variational Closure} \\
\large Validating G4/G5: Existence and Boundary Conditions}
\author{Independent Logical Section: Validating G4}
\date{\today}

\begin{document}
\maketitle

\section{Input Node: The Reciprocity Functional}

The system seeks minimal informational curvature, defined by the functional $\mathcal{R}[\U]$. The core principle of Reciprocity (Module 1) requires that every variation $\delta \mathcal{R}$ must correspond to a measurable distinction $\Phi$. The canonical closure is achieved when the first variation vanishes, $\delta \mathcal{R} = 0$.

The functional is defined as the integral of the squared informational curvature $\mathcal{R}[\U] = \int \mathcal{L}(\U'') \, dx$, where $\mathcal{L}(\U'') = \frac{1}{2} (\U'')^2$.

\begin{proposition}[License for Variational Principle (Validating $\mathbf{G5}$)]
The use of the continuous integral $\mathcal{R}[\U]$ is licensed by the **Axiom of Event Selection** (Martin-like consistency, $\mathbf{G5}$), which guarantees the non-constructive existence of a globally consistent extension. This existence allows the discrete reciprocal measure to be approximated by a continuous, minimal-curvature functional $\mathcal{R}[\U]$. This axiom is the required **Continuity Boundary**.
\end{proposition}

\section{Formal Variation and Integration by Parts}

To obtain the Euler-Lagrange condition (the input for G2/G3), we compute the first variation $\delta \mathcal{R}$ over an interval $[x_1, x_2]$ and apply integration by parts. This step formally validates the necessary boundary terms and signs for $\mathbf{G4}$.

\begin{proposition}[Closure of the Reciprocity Functional ($\mathbf{G4}$)]
The first variation of the functional $\mathcal{R}[\U]$ is:
\[
\delta \mathcal{R} = \int_{x_1}^{x_2} \left( \frac{\partial \mathcal{L}}{\partial \U} - \frac{d}{dx} \left( \frac{\partial \mathcal{L}}{\partial \U'} \right) + \frac{d^2}{dx^2} \left( \frac{\partial \mathcal{L}}{\partial \U''} \right) \right) \delta \U \, dx + \text{Boundary Terms}.
\]
For $\mathcal{L} = \frac{1}{2} (\U'')^2$, the Euler-Lagrange equation (the integrand) reduces to $\U^{(4)} = 0$ (the input to G2).
\end{proposition}

\section{The Boundary Terms and Physical Constraint}

The **Boundary Terms** resulting from the integration by parts must vanish for $\delta \mathcal{R} = 0$ to hold for arbitrary internal variations ($\delta \U$):

\[
\text{Boundary Terms} = \left[ \frac{\partial \mathcal{L}}{\partial \U'} - \frac{d}{dx} \left( \frac{\partial \mathcal{L}}{\partial \U''} \right) \right]_{x_1}^{x_2} \delta \U + \left[ \frac{\partial \mathcal{L}}{\partial \U''} \right]_{x_1}^{x_2} \delta \U'
\]

Substituting $\mathcal{L} = \frac{1}{2} (\U'')^2$:

\[
\text{Boundary Terms} = \left[ -\U''' \right]_{x_1}^{x_2} \delta \U + \left[ \U'' \right]_{x_1}^{x_2} \delta \U' = 0
\]

\textbf{Interpretation (Validating G4):} The vanishing of these terms enforces the dual system's closure at the measurement anchors ($x_1, x_2$). They confirm that the variation ($\delta \U$) and its slope ($\delta \U'$) at the boundaries must be either **fixed by measurement** (Dirichlet/Neumann conditions) or that the **natural conditions** ($\U''=0$ and $\U'''=0$) hold. This ensures the correct sign convention and closure needed to enforce $\mathbf{U}^{(4)}=0$ (G2) consistently across finite intervals.

\section{Output Node: Input to Kinematic Closure}

The successful closure of $\delta \mathcal{R} = 0$ via the necessary boundary terms (G4) confirms the validity of the **Euler-Lagrange Equation** $\mathbf{\U^{(4)}=0}$, which is the foundational starting point for Module 2 (Chapter 3, G2/G3).

\begin{center}
\textit{Conclusion for G4/G5: The continuity of the system is logically guaranteed and mechanically closed exactly at the boundaries of observation.}
\end{center}

\end{document}
