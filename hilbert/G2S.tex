\documentclass[12pt]{article}
\usepackage{amsthm,amssymb,amsmath,mathtools,setspace}
\onehalfspacing

% --- Module Dependencies ---
\newtheorem{theorem}{Theorem}
\newtheorem{definition}{Definition}
\newtheorem{proposition}{Proposition}

% --- Load Rosetta Layer ---
% V_macros.tex — ultra-minimal Rosetta (Gemini-safe)
% Only simple \providecommand definitions. No packages, no \makeatletter.

% --- Core symbols used across G1S/G2S/G3S/P/V_article ---
\providecommand{\U}{\mathbf{U}}         % Admissible history / universe field
\providecommand{\V}{\mathbf{V}}         % Test / variation (generic handle)
\providecommand{\ph}{\varphi}           % Test function (lowercase alias)
\providecommand{\Test}{\mathcal{V}}     % Test space
\providecommand{\Bform}{\mathsf{B}}     % Bilinear/reciprocity pairing (weak form)
\providecommand{\grad}{\nabla}          % Gradient / abstract differential operator
\providecommand{\C}{\mathcal{C}}        % Causal class / partition carrier

% --- Entropy & closures (title-safe via \ensuremath) ---
\providecommand{\EntropySymbol}{\mathrm{S}}
% \Entropy[sub] -> S_sub (subscript optional), safe in text/headings
\providecommand{\Entropy}[1][]{\ensuremath{\EntropySymbol_{\!#1}}}
% Fourth derivative shorthand (1D strong closure)
\providecommand{\DFour}{\U^{(4)}}

% --- Named statement handles (contracts), safe in text/headings ---
\providecommand{\GOneStatement}{\ensuremath{\Delta\,\Entropy \ge 0}}
\providecommand{\GTwoStatement}{\ensuremath{\DFour = 0}}
\providecommand{\GTwoWeakForm}{\ensuremath{\Bform(\U,\ph)=0\ \forall\,\ph\in\Test}}
\providecommand{\GThreeStatement}{\ensuremath{\text{Discrete–continuum reciprocity via }\Bform}}

% --- Noether bridge (index-free API primitives) ---
\providecommand{\Xi}{\boldsymbol{\xi}}                                      % symmetry generator (abstract)
\providecommand{\NoetherTensor}[3]{\mathsf{N}\!\left[#1,#2;#3\right]}       % N[L,U;Xi]
\providecommand{\StressEnergy}{\mathbf{T}}                                   % stress–energy (abstract)
\providecommand{\Current}{\mathbf{J}}                                        % Noether current (abstract)
\providecommand{\Div}[1]{\ensuremath{\grad\!\cdot\! #1}}                     % divergence via ∇· (uses \grad)


% Local Macro Definitions (if V_macros didn't load them)
\providecommand{\DTwo}{\U''}
\providecommand{\Domain}{\Omega}
\providecommand{\NaturalBC}{\U''(x_0)=\U''(x_n)=0}
\providecommand{\DFour}{\U^{(4)}}
\providecommand{\ph}{\varphi}
\providecommand{\Acal}{\mathcal{A}}

\newenvironment{varobligation}[1][]{%
  \par\medskip\noindent\textbf{G2 Proof Obligation.} #1\;\emph{(Provide stationarity, reciprocity, and refinement consistency.)}%
  \begin{enumerate}[leftmargin=2.2em,label=\alph*.]
}{%
  \end{enumerate}\medskip
}
\providecommand{\Rfunc}{\mathsf{R}}
\providecommand{\Jfunc}{\mathsf{J}}

\title{\textbf{Module 2: Kinematic Closure, Weak Reciprocity, and Newton Linearization} \\
\large Validating G2: The Minimal Curvature Condition \texorpdfstring{\GTwoStatement}{U^{(4)}=0}}
\author{G2S — Module Contract Fulfillment (V.tex Compliant)}
\date{\today}

\begin{document}
\maketitle

\section{Input Node: Reciprocity and Variational Setup}

The unique analytic closure for the admissible history $\U$ minimizes the **Informational Curvature Action** $\Acal[\U]$:
$$\Acal[\U] = \frac{1}{2} \int_{\Domain} (\DTwo)^2\, dx.$$

\section{Theorem: Kinematic Closure and Reciprocity}

\begin{theorem}[Strong Kinematic Closure $\mathbf{(\GTwoStatement)}$]
The unique minimal-curvature solution $\U^\star$ is the stationary point of $\Acal[\U]$, which yields the Euler-Lagrange strong form $\GTwoStatement$.
\end{theorem}

\begin{proof}[\textbf{G2 Proof Obligation Fulfillment}]
The weak form uses the bilinear reciprocity pairing $\Bform(\U, \ph) = \int_{\Domain} (\DTwo) (\ph'')\, dx$.
\begin{enumerate}
    \item \textbf{Weak Form (Reciprocity):} Stationarity $\delta \Acal = 0$ requires the condition:
    $$\text{Find } \U \in \Hspace \text{ such that } \GTwoWeakForm.$$.
    \item \textbf{Strong Form:} Double integration by parts on the weak form yields the strong closure: $\GTwoStatement$.
\end{enumerate}
\end{proof}

\section{G2: Newton Linearization and Noether Bridge}

\paragraph{Residual and Newton step (index-free).}
The analytic closure $\GTwoStatement$ is a non-linear (in coordinates/metric) residual function $\Rfunc(\U)$. The iteration is driven by the Fréchet Jacobian $\Jfunc(\U)$:
\[
\Rfunc(\U) := \DFour,\qquad
\Jfunc(\U) := \frac{\partial \Rfunc}{\partial \U}.
\]
The Newton update $\delta \U$ for an iterate $\U^{(k)}$ satisfies:
\[
\Jfunc(\U^{(k)})\,\delta \U \;=\; -\, \Rfunc(\U^{(k)}).
\]
This defines the computational implementation (the Newton rail) driven directly by the analytic closure $\DFour=0$.

\paragraph{Noether bridge (API-level, index-free).}
The stationary action, $\delta \Acal = 0$, supplies the core hypothesis required to invoke Noether's theorem.
The abstract conservation current $\Current$ (and stress-energy $\StressEnergy$) is derived from a local Lagrangian density $L$ and translational symmetry $\Xi$:
$$
\NoetherTensor{L}{\U}{\Xi} \quad \implies \quad \Div{\Current} = 0.
$$
The kinematic closure $\GTwoStatement$ is therefore the necessary structural input for the derivation of all subsequent conservation laws.

\end{document}