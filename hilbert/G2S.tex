\documentclass[12pt]{article}
\usepackage{amsthm,amssymb,amsmath,mathtools,setspace}
\usepackage{hyperref} % for \texorpdfstring
\usepackage{enumitem} % Required for enumerate formatting
\onehalfspacing

% --- Module Dependencies ---
\newtheorem{theorem}{Theorem}
\newtheorem{definition}{Definition}
\newtheorem{proposition}{Proposition}
\newtheorem{remark}{Remark} % <-- FIX: Added missing environment definition
% 
% --- Load Rosetta Layer (Contracts & Shared Symbols) ---
% V_macros.tex — ultra-minimal Rosetta (Gemini-safe)
% Only simple \providecommand definitions. No packages, no \makeatletter.

% --- Core symbols used across G1S/G2S/G3S/P/V_article ---
\providecommand{\U}{\mathbf{U}}         % Admissible history / universe field
\providecommand{\V}{\mathbf{V}}         % Test / variation (generic handle)
\providecommand{\ph}{\varphi}           % Test function (lowercase alias)
\providecommand{\Test}{\mathcal{V}}     % Test space
\providecommand{\Bform}{\mathsf{B}}     % Bilinear/reciprocity pairing (weak form)
\providecommand{\grad}{\nabla}          % Gradient / abstract differential operator
\providecommand{\C}{\mathcal{C}}        % Causal class / partition carrier

% --- Entropy & closures (title-safe via \ensuremath) ---
\providecommand{\EntropySymbol}{\mathrm{S}}
% \Entropy[sub] -> S_sub (subscript optional), safe in text/headings
\providecommand{\Entropy}[1][]{\ensuremath{\EntropySymbol_{\!#1}}}
% Fourth derivative shorthand (1D strong closure)
\providecommand{\DFour}{\U^{(4)}}

% --- Named statement handles (contracts), safe in text/headings ---
\providecommand{\GOneStatement}{\ensuremath{\Delta\,\Entropy \ge 0}}
\providecommand{\GTwoStatement}{\ensuremath{\DFour = 0}}
\providecommand{\GTwoWeakForm}{\ensuremath{\Bform(\U,\ph)=0\ \forall\,\ph\in\Test}}
\providecommand{\GThreeStatement}{\ensuremath{\text{Discrete–continuum reciprocity via }\Bform}}

% --- Noether bridge (index-free API primitives) ---
\providecommand{\Xi}{\boldsymbol{\xi}}                                      % symmetry generator (abstract)
\providecommand{\NoetherTensor}[3]{\mathsf{N}\!\left[#1,#2;#3\right]}       % N[L,U;Xi]
\providecommand{\StressEnergy}{\mathbf{T}}                                   % stress–energy (abstract)
\providecommand{\Current}{\mathbf{J}}                                        % Noether current (abstract)
\providecommand{\Div}[1]{\ensuremath{\grad\!\cdot\! #1}}                     % divergence via ∇· (uses \grad)


% Local symbols for this module
\providecommand{\DTwo}{\mathbf{U}''}
\providecommand{\Domain}{[x_0, x_n]}
\providecommand{\NaturalBC}{\mathbf{U}''(x_0)=\mathbf{U}''(x_n)=0}
\providecommand{\Acal}{\mathcal{A}}
\providecommand{\Hspace}{H^2(\Domain)} % local: Hilbert space used in this module
\providecommand{\Test}{\mathcal{V}}
\providecommand{\DFour}{\mathbf{U}^{(4)}}
\providecommand{\ph}{\varphi}
\providecommand{\Rfunc}{\mathsf{R}} % optional Newton shims
\providecommand{\Jfunc}{\mathsf{J}}

\newenvironment{varobligation}[1][]{%
  \par\medskip\noindent\textbf{G2 Proof Obligation.} #1\;\emph{(Provide stationarity, reciprocity, and refinement consistency.)}%
  \begin{enumerate}[leftmargin=2.2em,label=\alph*.]
}{%
  \end{enumerate}\medskip
}

\title{\textbf{Module 2: Kinematic Closure, Weak Reciprocity, and Newton Linearization}\\
\large Validating G2: Minimal Curvature Condition \texorpdfstring{\GTwoStatement}{U^{(4)}=0}}
\author{G2S — Module Contract Fulfillment (V.tex Compliant)}
\date{\today}

\begin{document}
\maketitle

\section{Input Node: Reciprocity and Variational Setup}

The unique analytic closure for the admissible history \(\U\) minimizes the \emph{Informational Curvature Action} \(\Acal[\U]\) over the Hilbert space \(\Hspace\):
\[
  \Acal[\U] \;=\; \frac{1}{2}\int_{\Domain} \big(\DTwo\big)^2\,dx.
\]
The minimization is subject to fixed interpolation nodes and the \emph{natural boundary conditions} \(\NaturalBC\).

\begin{definition}[Bilinear Form of Reciprocity]
The symmetric bilinear form \(\Bform: \Hspace \times \Hspace \to \mathbb{R}\) corresponding to the weak form of the variational principle is
\[
  \Bform(\U,\ph) \;=\; \int_{\Domain} \big(\DTwo\big)\,\ph''\,dx.
\]
\end{definition}

\section{Theorem: Kinematic Closure (\GTwoStatement)}

\begin{theorem}[Strong Kinematic Closure \(\mathbf{(\GTwoStatement)}\)]
The unique minimal-curvature solution \(\U^\star\) compatible with Event Selection is the stationary point of \(\Acal[\U]\), which yields the Euler–Lagrange strong form \(\GTwoStatement\).
\end{theorem}

<hr>

\begin{proof}[\textbf{G2 Proof Obligation Fulfillment}]
\begin{varobligation}
  \item \textbf{Reciprocity and coercivity.} The bilinear form \(\Bform(\U,\ph)\) is symmetric. Coercivity holds by a Poincaré–type inequality on \(H^2(\Domain)\).
  \item \textbf{Weak form (stationarity).} The first variation vanishes:
  \[
    \text{Find }\U\in\Hspace\ \text{such that}\ \GTwoWeakForm.
  \]
  \item \textbf{Strong form (Euler–Lagrange).} Integrating by parts twice, boundary terms vanish by \(\NaturalBC\) and the test space definition, yielding
  \[
    0 \;=\; \Bform(\U,\ph) \;=\; \int_\Domain \big(\DFour\big)\,\ph\,dx \quad\Longrightarrow\quad \GTwoStatement.
  \]
  \item \textbf{Refinement consistency (spline).} The stationary solution \(\U^\star\) is the cubic spline interpolant. Convergence of the discrete scheme confirms the continuum limit.
\end{varobligation}
\end{proof}

<hr>

\begin{remark}[\textbf{Kernel elimination}]
The homogeneous strong form \(\GTwoStatement\) admits the four-dimensional kernel \(\mathrm{span}\{1,x,x^2,x^3\}\).
Fixed interpolation nodes together with \(\NaturalBC\) remove the kernel, ensuring the \emph{unique cubic spline solution} \(\U^\star\).
\end{remark}

\section{G2: Operational Interpretation and Noether Bridge}

\paragraph{Physical meaning of \(\GTwoStatement\).}
The closure condition \(\DFour=0\) selects the minimal-information path consistent with global reciprocity.

\begin{example}[Davisson–Germer and kinematic consistency]
Electron diffraction demonstrates that a discrete event chain (particle) must obey the continuous propagation law induced by the reciprocity principle behind \(\GTwoStatement\). The observed intensity peaks are \emph{fixed points} of reciprocal measurement under lattice translations.
\end{example}

\paragraph{Residual and Newton step (index-free).}
The kinematic closure is interpreted as a numerically tractable root-finding problem.
Define the residual \(\Rfunc\) and (Fréchet) Jacobian \(\Jfunc\):
\[
  \Rfunc(\U) \;:=\; \DFour, \qquad
  \Jfunc(\U) \;:=\; \frac{\partial \Rfunc}{\partial \U}.
\]
Given an iterate \(\U^{(k)}\), the update \(\delta\U\) solves the linear system:
\[
  \Jfunc(\U^{(k)})\,\delta\U \;=\; -\,\Rfunc(\U^{(k)}).
\]

\paragraph{Noether bridge (API-level, index-free).}
Stationarity \(\delta\Acal=0\) is the hypothesis for conservation. From a local density \(L\) and symmetry \(\Xi\),
\[
  \StressEnergy \;\leftarrow\; \NoetherTensor{L}{\U}{\Xi}, \qquad
  \Div{\StressEnergy} \;=\; 0.
\]
Thus \(\GTwoStatement\) provides the structural input for the conservation law formalized in G4S.

\end{document}