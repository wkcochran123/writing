\documentclass[12pt]{article}
\usepackage{amsthm,amssymb,amsmath,mathtools,setspace}
\usepackage{hyperref} % for \texorpdfstring
\onehalfspacing

% --- Module Dependencies ---
\newtheorem{theorem}{Theorem}
\newtheorem{definition}{Definition}
\newtheorem{proposition}{Proposition}

% --- Load Rosetta Layer (Contracts & Shared Symbols) ---
% V_macros.tex — ultra-minimal Rosetta (Gemini-safe)
% Only simple \providecommand definitions. No packages, no \makeatletter.

% --- Core symbols used across G1S/G2S/G3S/P/V_article ---
\providecommand{\U}{\mathbf{U}}         % Admissible history / universe field
\providecommand{\V}{\mathbf{V}}         % Test / variation (generic handle)
\providecommand{\ph}{\varphi}           % Test function (lowercase alias)
\providecommand{\Test}{\mathcal{V}}     % Test space
\providecommand{\Bform}{\mathsf{B}}     % Bilinear/reciprocity pairing (weak form)
\providecommand{\grad}{\nabla}          % Gradient / abstract differential operator
\providecommand{\C}{\mathcal{C}}        % Causal class / partition carrier

% --- Entropy & closures (title-safe via \ensuremath) ---
\providecommand{\EntropySymbol}{\mathrm{S}}
% \Entropy[sub] -> S_sub (subscript optional), safe in text/headings
\providecommand{\Entropy}[1][]{\ensuremath{\EntropySymbol_{\!#1}}}
% Fourth derivative shorthand (1D strong closure)
\providecommand{\DFour}{\U^{(4)}}

% --- Named statement handles (contracts), safe in text/headings ---
\providecommand{\GOneStatement}{\ensuremath{\Delta\,\Entropy \ge 0}}
\providecommand{\GTwoStatement}{\ensuremath{\DFour = 0}}
\providecommand{\GTwoWeakForm}{\ensuremath{\Bform(\U,\ph)=0\ \forall\,\ph\in\Test}}
\providecommand{\GThreeStatement}{\ensuremath{\text{Discrete–continuum reciprocity via }\Bform}}

% --- Noether bridge (index-free API primitives) ---
\providecommand{\Xi}{\boldsymbol{\xi}}                                      % symmetry generator (abstract)
\providecommand{\NoetherTensor}[3]{\mathsf{N}\!\left[#1,#2;#3\right]}       % N[L,U;Xi]
\providecommand{\StressEnergy}{\mathbf{T}}                                   % stress–energy (abstract)
\providecommand{\Current}{\mathbf{J}}                                        % Noether current (abstract)
\providecommand{\Div}[1]{\ensuremath{\grad\!\cdot\! #1}}                     % divergence via ∇· (uses \grad)


% Local symbols defined via V_macros for use in the body
\providecommand{\Lcal}{\mathcal{L}}
\providecommand{\Scal}{\mathcal{S}}
\providecommand{\Xi}{\boldsymbol{\xi}}

\title{\textbf{Module 4: The Noether Bridge} \\
\large Validating G4: Derivation of Conserved Currents \texorpdfstring{\Div{\StressEnergy}=0}{div T = 0}}
\author{G4S — Proof of Conservation (V.tex Compliant)}
\date{\today}

\begin{document}
\maketitle

\bigskip\hrule\bigskip

\section{Input Node: Stationarity from Kinematic Closure (G2)}
The conservation law relies on the \emph{action principle} being stationary. The kinematic closure \(\GTwoStatement\) guarantees that the fundamental field \(\U\) is a solution to the Euler–Lagrange equations in the sense needed to invoke stationarity \(\delta \Scal = 0\).

\bigskip\hrule\bigskip

\begin{definition}[Action Functional \(\Scal\)]
The action functional is the integral of the local Lagrangian density \(\Lcal\) that encapsulates informational curvature:
\[
  \Scal[\U] \;=\; \int \Lcal\!\big(\U,\, \grad \U\big)\, d\tau ,
\]
where the measure \(d\tau\) is left abstract (index-free).
\end{definition}

\bigskip\hrule\bigskip

\section{Theorem: Conservation Law (Index-Free Noether)}

\begin{theorem}[Noether Conservation]
For every continuous symmetry \(\Xi\) that leaves the action invariant \((\delta \Scal = 0)\), there exists an associated current \(\Current\) whose divergence vanishes:
\[
  \Div{\Current} \;=\; 0.
\]
\end{theorem}

\bigskip\hrule\bigskip

\begin{proof}[\textbf{G4 Proof Obligation Fulfillment (Translational Case)}]
\begin{enumerate}
  \item \textbf{Hypothesis (stationarity).} The closure \(\GTwoStatement\) positions \(\U\) at a stationary point: \(\delta \Scal = 0\).
  \item \textbf{Symmetry selection.} We select the translational symmetry \(\Xi_{\mathrm{trans}}\).
  \item \textbf{Noether bridge (API).} Invoke the abstract construction
  \[
    \StressEnergy \;\leftarrow\; \NoetherTensor{\Lcal}{\U}{\Xi_{\mathrm{trans}}}.
  \]
  \item \textbf{Conservation identity.} Symmetry implies the resulting current is conserved, establishing the global bookkeeping constraint:
  \[
    \Div{\StressEnergy} \;=\; 0.
  \]
\end{enumerate}
Thus the Noether bridge carries the geometric constraint \(\GTwoStatement\) into the required structural conservation statement.
\end{proof}

\bigskip\hrule\bigskip

\end{document}
