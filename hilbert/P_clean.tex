\documentclass[12pt,oneside]{book}
\usepackage{amsthm,amssymb,amsmath,mathtools,setspace}
\onehalfspacing

% --- Minimal Dependencies for the Proof Structure ---
\newtheorem{theorem}{Theorem}
\newtheorem{axiom}{Axiom}
\newtheorem{definition}{Definition}
\newtheorem{equivalence}{Equivalence}
\newcommand{\U}{\mathbf{U}}        % Universe tensor
\newcommand{\Ek}{\mathbf{E}_k}      % Event tensor
\newcommand{\grad}{\nabla}
\newcommand{\T}{\mathbf{T}}        % Stress tensor

% --- Title: Using the existing document information ---
\title{\textbf{The Limits of Representation}\\ \large Volume I: The Conditions of $\Delta S \ge 0$}
\author{The Constructive Proof}
\date{\today}

\begin{document}
\maketitle
\frontmatter

\chapter*{Abstract}
% [Insert the core abstract text from P.tex, focusing on measurement,
% Event Selection (MA-like), U(4)=0, and Delta S >= 0. This sets the formal tone.]

\chapter*{Overview: The Six Chapters of Consistency}
% [This is the critical linking section. It directly lists the
% V.tex concept and the formal P.tex correlate, showing the flow.]

The constructive proof is divided into six stages, corresponding directly to the hierarchy of necessity in a self-consistent universe. This progression links foundational axioms to physical law:

\begin{enumerate}
    \item \textbf{The Calculus of Measurement:} Defining the physical object (the finite event and the partial order).
    \item \textbf{The Reciprocity Law:} Establishing the duality between observation and variation.
    \item \textbf{Smoothness and Kinematic Closure:} Enforcing minimal informational curvature.
    \item \textbf{Conservation and Coherence:} Deriving conserved laws from translational invariance.
    \item \textbf{The Arrow of Time:} Proving the monotonicity of the set of distinctions ($\Delta S \ge 0$).
    \item \textbf{The Proof Tensor and the Cosmic Ledger:} Unifying the structure as a global fixed point.
\end{enumerate}

\mainmatter

% =================================================================
% The Six Core Chapters based on the V.tex structure
% =================================================================

\chapter{The Calculus of Measurement: Defining Order}
\label{chap:measurement}
\section*{Core Idea: The Universe is a Process of Measurement}

% **V.tex Formal Correlate:** Definition of events, order, and information.
% **Task:** Introduce ZFC, Axiom of Order, Definition of Event Set E, Time as Ordinal Rank (k), Event Tensor E_k, and the Universe Tensor U_n.
% The material comes primarily from P.tex Chapter 1 Introduction and Chapter 2, Section 2.1-2.2.

\begin{axiom}[The Axiom of Order]
Time is an ordinal rank ($k$) on the sequence of distinguishable events.
\end{axiom}
\begin{definition}[Universe Tensor]
The cumulative record of distinctions is the ordered fold $\U_n = \sum_{k=1}^n \Ek$.
\end{definition}


\chapter{The Reciprocity Law: Duality of Observation and Change}
\label{chap:reciprocity}
\section*{Core Idea: Observation and Change are Mirror Operations}

% **V.tex Formal Correlate:** The Reciprocity Condition, linking $\delta U$ and $\delta E$ as inverse transformations.
% **Task:** Introduce Event Selection (MA-like, the selection rule for coherent history) and formally state the Reciprocity Law as the bijection $\Phi: \text{Variation} \leftrightarrow \text{Measurement}$. This establishes the dual system.
% The material comes from P.tex Chapter 2, Section 2.3-2.5.

\begin{axiom}[Axiom of Event Selection (MA-like)]
Every countable family of admissible local causal choices admits a globally consistent extension.
\end{axiom}
\begin{equivalence}[The Reciprocity Law]
Every physically admissible variation corresponds to a measurable distinction, and vice versa.
\end{equivalence}


\chapter{Smoothness and Kinematic Closure}
\label{chap:smoothness}
\section*{Core Idea: Consistency Across Observations Requires Smooth Motion}

% **V.tex Formal Correlate:** The closure condition $U^{(4)} = 0$, the minimal-curvature spline rule.
% **Task:** Prove the necessity of the minimal-curvature closure derived from minimizing the informational curvature functional $R[U] = \int (\Delta^{(2)} U)^2 dx$. This must yield the condition $U^{(4)}=0$.
% The material comes from P.tex Chapter 2, Section 2.5-2.6 (Kinematics and Discrete-to-Continuum Limit).

\begin{theorem}[Kinematic Closure]
Minimal informational friction enforced by consistency yields the cubic-spline condition $\U^{(4)} = 0$.
\end{theorem}


\chapter{Conservation and Coherence}
\label{chap:conservation}
\section*{Core Idea: Smooth Motion Implies Conservation}

% **V.tex Formal Correlate:** $\nabla_\mu T^{\mu\nu} = 0$, the conservation law as the bookkeeping rule of coherence.
% **Task:** Introduce Noether's theorem in the continuum limit. Define the stress-energy tensor $T^{\mu\nu}$ as the conserved flow of distinguishability and show that $\nabla_\mu T^{\mu\nu} = 0$ is the consistency (bookkeeping) equation for a smooth system.
% The material comes from P.tex Chapter 4 (Kinematics of Light) and Chapter 6 (Conservation/Noether).

\begin{theorem}[Bookkeeping of Coherence]
Translation invariance of the Causal Universe Tensor implies conservation of distinguishability, expressed differentially as $\grad_\mu T^{\mu\nu} = 0$.
\end{theorem}


\chapter{The Arrow of Time: The Second Law as a Theorem}
\label{chap:arrow-of-time}
\section*{Core Idea: A Consistent Record Must Grow}

% **V.tex Formal Correlate:** $\Delta S \ge 0$, the theorem of order and distinguishability.
% **Task:** Synthesize the preceding four chapters. Define Entropy S as $k_B \ln |\Omega(\mathcal{C})|$ (the count of admissible orderings) and formally prove the monotonicity theorem $\Delta S \ge 0$.
% The material comes from P.tex Chapter 5 (The Second Law) and Chapter 4.4 (Defining Entropy).

\begin{theorem}[Second Law of Causal Order]
For every Martin-consistent refinement of causal structure, the entropy of the system is non-decreasing: $\Delta S \ge 0$.
\end{theorem}


\chapter{The Proof Tensor and the Cosmic Ledger}
\label{chap:ledger}
\section*{Core Idea: Unifying Structure as a Global Fixed Point}

% **V.tex Formal Correlate:** $U(n)$ as the constructive fixed point linking the logical and physical layers.
% **Task:** Provide a concluding summary, connecting the field equations (like the Einstein Equation, $G_{\mu\nu} \propto T_{\mu\nu}$ derived from the gauge of light/curvature as inconsistency) back to the foundation laid by the Proof Tensor U_n. Emphasize the closure of the argument.
% The material comes from P.tex Chapter 4 and the final Epilogue.

\begin{center}
\textit{Quod erat demonstrandum. Order implies dynamics.}
\end{center}

\backmatter
% [Insert bibliography and any final appendices from P.tex]

\end{document}