\documentclass[12pt,oneside]{book}
\usepackage{amsthm,amssymb,amsmath,mathtools,setspace}
\onehalfspacing

% --- Minimal Dependencies for the Proof Structure ---
\newtheorem{theorem}{Theorem}
\newtheorem{axiom}{Axiom}
\newtheorem{definition}{Definition}
\newtheorem{equivalence}{Equivalence}

% --- Core symbols (guarded to avoid redefinition clashes) ---
\providecommand{\U}{\mathbf{U}}        % Universe tensor 
\providecommand{\Ek}{\mathbf{E}_k}      % Event tensor
\providecommand{\grad}{\nabla}
\providecommand{\T}{\mathbf{T}}         % Stress tensor 
\providecommand{\Acal}{\mathcal{A}}     % Informational Curvature Action (from V)
\providecommand{\Bform}{\mathsf{B}}    % Symmetric Bilinear Form (from V)
\providecommand{\InjectiveMap}{\iota}   % Injective Class Map (from V)
\providecommand{\OmegaSet}{\Omega}      % Set of equivalence classes

% --- V hooks fallbacks (no-ops if V.tex isn't loaded) ---
\makeatletter
\@ifundefined{vnode}{\newcommand{\vnode}[2][]{#2}}{}
\@ifundefined{vnodechange}{\newcommand{\vnodechange}[3][]{}}{}
\makeatother

% --- Title: Using the existing document information ---
\title{\textbf{The Limits of Representation}\\ \large Volume I: The Conditions of $\Delta S \ge 0$} 
\author{Bill Cochran (Minimal Contract Compliance)} 
\date{\today}

\begin{document}
\maketitle
\frontmatter

\vnode[id=abs]{\chapter*{Abstract}}
This work presents a constructive proof that the entropy of any causally consistent universe is non-decreasing, $\Delta S \ge 0$. Measurement itself defines the metric structure of physics: each distinction generates an increment of causal order. Within the axioms of ZFC with Choice, a finite causal order of distinguishable events is defined whose dual operations—measurement and variation—form a bijective pair under the Reciprocity Law of Physics. Requiring global coherence under an axiom of Event Selection enforces the fourth-order cancellation $\mathbf{U}^{(4)} = 0$, which identifies the cubic spline as the minimal analytic closure of the dual system. This closure produces the continuous calculus as the smooth limit of finite causal measurement. The theorem $\Delta S \ge 0$ emerges as the necessary condition that any universe consistent with its own record of distinctions must increase the count of what can be known.

\vnode[id=overview]{\chapter*{Overview: The Logical Chain of Consistency}}
\addcontentsline{toc}{chapter}{Overview: The Logical Chain of Consistency}

The constructive proof proceeds in six stages, mapping axiomatic necessity to physical law.

\begin{enumerate}
    \item \textbf{I. Causal Primitives:} Defining the Ordered Set ($\U$).
    \item \textbf{II. Global Consistency:} The Reciprocity Law (Duality and $\Bform$).
    \item \textbf{III. Kinematic Closure:} Deriving Minimal Curvature ($\mathbf{U}^{(4)}=0$).
    \item \textbf{IV. Dynamics:} Noether's Theorem and Conserved Bookkeeping ($\grad_\mu \mathbf{T}^{\mu\nu}=0$).
    \item \textbf{V. Thermodynamic Closure:} Proving the Arrow of Time ($\Delta S \ge 0$).
    \item \textbf{VI. The Fixed Point:} Unifying the Ledger.
\end{enumerate}

\mainmatter

% =================================================================
% The Six Core Chapters with V-compliant claims (Improved Headings)
% =================================================================

\vnode[id=chap:order]{\chapter{I. The Calculus of Distinction}}
\addcontentsline{toc}{chapter}{I. The Calculus of Distinction}
\section*{Core Idea: Time is the ordinal index of measurable events, defining the Universe Tensor $\U$.}

\begin{axiom}[The Axiom of Order]
Time is an ordinal rank ($k$) on the sequence of distinguishable events.
\end{axiom}
\begin{definition}[Universe Tensor]
The cumulative record of distinctions is the ordered fold $\U_n = \sum_{k=1}^n \Ek$.
\end{definition}

\vnode[id=chap:reciprocity]{\chapter{II. The Reciprocal Duality}}
\addcontentsline{toc}{chapter}{II. The Reciprocal Duality}
\section*{Core Idea: The Axiom of Event Selection enforces a bijective duality ($\Bform$) between observable measurements and admissible variations.}

\begin{axiom}[Axiom of Event Selection (MA-like)]
Every countable family of admissible local causal choices admits a globally consistent extension.
\end{axiom}
\begin{equivalence}[The Reciprocity Law: Bilinear Symmetry]
The duality between observation and variation defines the symmetric \textbf{Bilinear Form $\Bform$}, ensuring the consistency of measurable energy.
\end{equivalence}

\vnode[id=chap:spline]{\chapter{III. Kinematic Closure: The $\mathbf{U}^{(4)}=0$ Fixed Point}}
\addcontentsline{toc}{chapter}{III. Kinematic Closure: The $\mathbf{U}^{(4)}=0$ Fixed Point}
\section*{Core Idea: Enforcing minimal information curvature ($\Acal[\U]$) requires the continuity provided by the unique analytic solution: $\mathbf{U}^{(4)} = 0$.}

\begin{theorem}[Kinematic Closure: Euler Fixed Point]
The minimizer of the \textbf{Informational Curvature Action $\Acal[\U]$} yields the Euler–Lagrange condition $\mathbf{U}^{(4)} = 0$, defining the minimal analytic closure.
\end{theorem}

\vnode[id=chap:conservation]{\chapter{IV. Conservation: Dynamics from Consistency}}
\addcontentsline{toc}{chapter}{IV. Conservation: Dynamics from Consistency}
\section*{Core Idea: Translational symmetry of the $\mathbf{U}^{(4)}=0$ field yields the conserved stress tensor $\mathbf{T}^{\mu\nu}$, enforcing the bookkeeping identity $\grad_\mu \mathbf{T}^{\mu\nu} = 0$.}

\begin{theorem}[Noether Conservation]
The translational symmetry of the $\mathbf{U}^{(4)} = 0$ kinematic field yields the conserved Noether current $\mathbf{T}^{\mu\nu}$, satisfying the bookkeeping identity $\grad_\mu \mathbf{T}^{\mu\nu} = 0$.
\end{theorem}

\vnode[id=chap:entropy]{\chapter{V. Thermodynamic Closure: The $\Delta S \ge 0$ Theorem}}
\addcontentsline{toc}{chapter}{V. Thermodynamic Closure: The $\Delta S \ge 0$ Theorem}
\section*{Core Idea: Causal consistency requires that every admissible refinement increases the set of distinguishable classes, proving the Second Law ($\Delta S \ge 0$).}

\begin{theorem}[Second Law: Injective Count Monotonicity]
For every consistent refinement that maintains observable compatibility, there exists an \textbf{Injective Class Map $\InjectiveMap$} ensuring the cardinality of distinguishable classes is non-decreasing: $\Delta S \ge 0$.
\end{theorem}

\vnode[id=chap:ledger]{\chapter{VI. The Invariant: Unifying the Causal Fixed Point}}
\addcontentsline{toc}{chapter}{VI. The Invariant: Unifying the Causal Fixed Point}
\section*{Core Idea: The entire hierarchy of laws is unified as the unique, self-consistent structure of the Proof Tensor.}

\begin{center}
\textit{Quod erat demonstrandum. Order implies dynamics.}
\end{center}

\backmatter
\end{document}
