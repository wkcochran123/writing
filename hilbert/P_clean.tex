\documentclass[12pt]{article}
\usepackage[T1]{fontenc}
\usepackage[utf8]{inputenc}
\usepackage{lmodern}
\usepackage{microtype}
\usepackage{geometry}
\geometry{margin=1in}
\usepackage{amsmath,amssymb,mathtools}
\usepackage[colorlinks=true,linkcolor=blue,citecolor=blue,urlcolor=blue]{hyperref}
\usepackage{setspace}
\usepackage{pdfpages} % Required for Option A: Including compiled PDFs
\onehalfspacing

% --- Final Unified Macro Set (Source V_macros.tex) ---
% All essential symbols and contracts are loaded here.
% V_macros.tex — ultra-minimal Rosetta (Gemini-safe)
% Only simple \providecommand definitions. No packages, no \makeatletter.

% --- Core symbols used across G1S/G2S/G3S/P/V_article ---
\providecommand{\U}{\mathbf{U}}         % Admissible history / universe field
\providecommand{\V}{\mathbf{V}}         % Test / variation (generic handle)
\providecommand{\ph}{\varphi}           % Test function (lowercase alias)
\providecommand{\Test}{\mathcal{V}}     % Test space
\providecommand{\Bform}{\mathsf{B}}     % Bilinear/reciprocity pairing (weak form)
\providecommand{\grad}{\nabla}          % Gradient / abstract differential operator
\providecommand{\C}{\mathcal{C}}        % Causal class / partition carrier

% --- Entropy & closures (title-safe via \ensuremath) ---
\providecommand{\EntropySymbol}{\mathrm{S}}
% \Entropy[sub] -> S_sub (subscript optional), safe in text/headings
\providecommand{\Entropy}[1][]{\ensuremath{\EntropySymbol_{\!#1}}}
% Fourth derivative shorthand (1D strong closure)
\providecommand{\DFour}{\U^{(4)}}

% --- Named statement handles (contracts), safe in text/headings ---
\providecommand{\GOneStatement}{\ensuremath{\Delta\,\Entropy \ge 0}}
\providecommand{\GTwoStatement}{\ensuremath{\DFour = 0}}
\providecommand{\GTwoWeakForm}{\ensuremath{\Bform(\U,\ph)=0\ \forall\,\ph\in\Test}}
\providecommand{\GThreeStatement}{\ensuremath{\text{Discrete–continuum reciprocity via }\Bform}}

% --- Noether bridge (index-free API primitives) ---
\providecommand{\Xi}{\boldsymbol{\xi}}                                      % symmetry generator (abstract)
\providecommand{\NoetherTensor}[3]{\mathsf{N}\!\left[#1,#2;#3\right]}       % N[L,U;Xi]
\providecommand{\StressEnergy}{\mathbf{T}}                                   % stress–energy (abstract)
\providecommand{\Current}{\mathbf{J}}                                        % Noether current (abstract)
\providecommand{\Div}[1]{\ensuremath{\grad\!\cdot\! #1}}                     % divergence via ∇· (uses \grad)


% Title uses all four contracts for full context
\title{\textbf{P-Clean: The Unified Causal Closure Chain} \\ \quad \\ \large Volume I: The Conditions of \texorpdfstring{\GOneStatement}{Delta S >= 0}}
\author{The Project Synthesis}
\date{\today}

\begin{document}
\maketitle

\begin{abstract}
This document synthesizes the core closure arguments (G1--G4). It shows that the foundational axiom of \textbf{Minimal Informational Curvature} (\texorpdfstring{\StrongForm}{U^{(4)}=0}) supplies the structural hypothesis necessary to derive the \textbf{Conservation of Stress--Energy} (\texorpdfstring{\ConcludeDivT}{div T = 0}), which rigorously enforces the \textbf{Second Law of Causal Order} (\GOneStatement) as a theorem of monotonicity. The full system is constrained by \textbf{Discrete--Continuum Reciprocity} (\GThreeStatement) and the \textbf{Causal Data Processing Inequality} (\GFourDPI), ensuring the integrity of measurement and prediction.
\end{abstract}

% --- TOC Inclusion (Stylistic Polish) ---
\tableofcontents
\bigskip\hrule\bigskip

\section{The Axiomatic Chain: Order Implies Dynamics}\label{sec:arc}

The argument's structure is a single dependency chain, demonstrating that physical laws emerge from consistency:
\[
  \StrongForm \quad \xrightarrow{\text{G4S: Noether Bridge}} \quad \ConcludeDivT \quad \xrightarrow{\text{G1S: Constraint}} \quad \GOneStatement.
\]

\section{Core Contracts and Closure Arguments}

\subsection{Kinematic Closure and Reciprocity (G2)}\label{sec:g2}

The unique analytic closure for the admissible history \(\U\) minimizes the Informational Curvature Action, leading to the \textbf{Kinematic Closure} strong form and its necessary reciprocal constraint.

\paragraph{Strong Closure.} The minimization yields the Euler–Lagrange equation:
\[
  \StrongForm.
\]

\paragraph{Weak Reciprocity.} This closure is equivalent to the \textbf{Weak Reciprocity Form}, which utilizes the symmetric, coercive bilinear form \(\Bform\):
\[
  \WeakForm.
\]

\subsection{Conservation and Discrete Consistency (G3/G4S)}\label{sec:g34}

The stability established in G2 provides the foundation for all conservation laws, verified across the continuous and discrete domains.

\paragraph{Discrete Consistency (G3).} The discrete kinematic problem is proven to be stable and consistent, ensuring that the smooth analytic solution is the unique limit of finite measurement processes, fulfilling the contract:
\[
  \GThreeStatement.
\]

\paragraph{Noether Conservation (G4S).} The stationarity from G2 is the hypothesis for Noether's Theorem. Translational symmetry (\(\Xi\)) generates the conserved \textbf{Stress--Energy Tensor} (\(\StressEnergy\)):
\[
  \ConcludeDivT.
\]

\subsection{Thermodynamic and Statistical Closure (G1/G5)}\label{sec:g15}

The conserved structure provides the constraints necessary for the final laws governing distinguishability and predictability.

\paragraph{Thermodynamic Closure (G1).} The conservation of order enforces that every admissible causal refinement (\(\mathcal{C}_{n} \to \mathcal{C}_{n+1}\)) cannot decrease the set of distinguishable micro-orderings (\(\Omega(\mathcal{C})\)), proving the \textbf{Second Law of Causal Order}:
\[
  \GOneStatement.
\]

\paragraph{Statistical Closure (G5) (one-paragraph seed).}
Admissible coarse-grainings consistent with \(\C\) cannot increase distinguishability and the score is unbiased at truth; together with conservation, these facts pin down which summaries remain stable and how much prediction is possible:
\[
  \GFourDPI \qquad\text{and}\qquad \GFourScoreZero .
\]
Conservation links symmetry to stability,
\[
  \ConcludeDivT \;\Longrightarrow\; \text{stability of } \Invariant{\Xi},
\]
so any remaining structure is captured as constrained noise \(\Res\), and predictability is bounded by the invariant geometry implied by these closures.

% --- Appendix Inclusion (Option A: Including Compiled PDFs with \addtotoc) ---
\appendix
\section*{Appendix: Validation Modules}
\addcontentsline{toc}{section}{Appendix: Validation Modules}
\includepdf[pages=-,pagecommand={\section*{G1S: Order Monotonicity}\addcontentsline{toc}{subsection}{G1S: Order Monotonicity}}]{G1S.pdf}
\includepdf[pages=-,pagecommand={\section*{G2S: Kinematic Closure}\addcontentsline{toc}{subsection}{G2S: Kinematic Closure}}]{G2S.pdf}
\includepdf[pages=-,pagecommand={\section*{G3S: Discrete–Continuum Reciprocity}\addcontentsline{toc}{subsection}{G3S: Discrete–Continuum Reciprocity}}]{G3S.pdf}
\includepdf[pages=-,pagecommand={\section*{G4S: Noether Bridge}\addcontentsline{toc}{subsection}{G4S: Noether Bridge}}]{G4S.pdf}
\includepdf[pages=-,pagecommand={\section*{G5S: Statistical Closure}\addcontentsline{toc}{subsection}{G5S: Statistical Closure}}]{G5S.pdf}

\end{document}