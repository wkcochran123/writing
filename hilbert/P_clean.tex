\documentclass[12pt,oneside]{book}
\usepackage{amsthm,amssymb,amsmath,mathtools,setspace}
\onehalfspacing

% --- Minimal Dependencies for the Proof Structure ---
\newtheorem{theorem}{Theorem}
\newtheorem{axiom}{Axiom}
\newtheorem{definition}{Definition}
\newtheorem{equivalence}{Equivalence}
[cite_start]\newcommand{\U}{\mathbf{U}}        % Universe tensor [cite: 1]
[cite_start]\newcommand{\Ek}{\mathbf{E}_k}      % Event tensor [cite: 1]
\newcommand{\grad}{\nabla}
[cite_start]\newcommand{\T}{\mathbf{T}}        % Stress tensor [cite: 1]

% --- Title: Using the existing document information ---
[cite_start]\title{\textbf{The Limits of Representation}\\ \large Volume I: The Conditions of $\Delta S \ge 0$} [cite: 1]
[cite_start]\author{The Constructive Proof} [cite: 1]
[cite_start]\date{\today} [cite: 1]

\begin{document}
\maketitle
\frontmatter

[cite_start]\chapter*{Abstract} [cite: 1]
% Abstract (content not provided, but chapter title is)

[cite_start]\chapter*{Overview: The Six Chapters of Consistency} [cite: 1]
[cite_start]The constructive proof is divided into six stages [cite: 2][cite_start], corresponding directly to the hierarchy of necessity in a self-consistent universe[cite: 2].

\begin{enumerate}
    [cite_start]\item \textbf{The Calculus of Measurement:} Defining the physical object (the finite event and the partial order)[cite: 3].
    [cite_start]\item \textbf{The Reciprocity Law:} Establishing the duality between observation and variation[cite: 4].
    [cite_start]\item \textbf{Smoothness and Kinematic Closure:} Enforcing minimal informational curvature[cite: 4].
    [cite_start]\item \textbf{Conservation and Coherence:} Deriving conserved laws from translational invariance[cite: 5].
    [cite_start]\item \textbf{The Arrow of Time:} Proving the monotonicity of the set of distinctions ($\Delta S \ge 0$)[cite: 6].
    [cite_start]\item \textbf{The Proof Tensor and the Cosmic Ledger:} Unifying the structure as a global fixed point[cite: 7].
\end{enumerate}

\mainmatter

% =================================================================
% The Six Core Chapters and their core statements
% =================================================================

\chapter{The Calculus of Measurement: Defining Order}
[cite_start]\label{chap:measurement} [cite: 8]
[cite_start]\section*{Core Idea: The Universe is a Process of Measurement} [cite: 8]

[cite_start]\begin{axiom}[The Axiom of Order] [cite: 11]
[cite_start]Time is an ordinal rank ($k$) on the sequence of distinguishable events[cite: 11].
\end{axiom}
[cite_start]\begin{definition}[Universe Tensor] [cite: 12]
[cite_start]The cumulative record of distinctions is the ordered fold $\U_n = \sum_{k=1}^n \Ek$[cite: 12].
\end{definition}


\chapter{The Reciprocity Law: Duality of Observation and Change}
[cite_start]\label{chap:reciprocity} [cite: 13]
[cite_start]\section*{Core Idea: Observation and Change are Mirror Operations} [cite: 13]

[cite_start]\begin{axiom}[Axiom of Event Selection (MA-like)] [cite: 16]
[cite_start]Every countable family of admissible local causal choices admits a globally consistent extension[cite: 16].
\end{axiom}
[cite_start]\begin{equivalence}[The Reciprocity Law] [cite: 17]
[cite_start]Every physically admissible variation corresponds to a measurable distinction, and vice versa[cite: 17].
\end{equivalence}


\chapter{Smoothness and Kinematic Closure}
[cite_start]\label{chap:smoothness} [cite: 18]
[cite_start]\section*{Core Idea: Consistency Across Observations Requires Smooth Motion} [cite: 18]

[cite_start]\begin{theorem}[Kinematic Closure] [cite: 21]
[cite_start]Minimal informational friction enforced by consistency yields the cubic-spline condition $\U^{(4)} = 0$[cite: 21].
\end{theorem}


\chapter{Conservation and Coherence}
[cite_start]\label{chap:conservation} [cite: 23]
[cite_start]\section*{Core Idea: Smooth Motion Implies Conservation} [cite: 23]

[cite_start]\begin{theorem}[Bookkeeping of Coherence] [cite: 25]
[cite_start]Translation invariance of the Causal Universe Tensor implies conservation of distinguishability, expressed differentially as $\grad_\mu \T^{\mu\nu} = 0$[cite: 25].
\end{theorem}


\chapter{The Arrow of Time: The Second Law as a Theorem}
[cite_start]\label{chap:arrow-of-time} [cite: 27]
[cite_start]\section*{Core Idea: A Consistent Record Must Grow} [cite: 27]

[cite_start]\begin{theorem}[Second Law of Causal Order] [cite: 29]
[cite_start]For every Martin-consistent refinement of causal structure, the entropy of the system is non-decreasing: $\Delta S \ge 0$[cite: 29].
\end{theorem}


\chapter{The Proof Tensor and the Cosmic Ledger}
[cite_start]\label{chap:ledger} [cite: 31]
[cite_start]\section*{Core Idea: Unifying Structure as a Global Fixed Point} [cite: 31]

\begin{center}
\textit{Quod erat demonstrandum. [cite_start]Order implies dynamics.} [cite: 33]
\end{center}

\backmatter

\end{document}