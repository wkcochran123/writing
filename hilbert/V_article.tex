\documentclass[12pt]{article}

% Minimal, safe preamble
\usepackage[T1]{fontenc}
\usepackage[utf8]{inputenc}
\usepackage{lmodern}
\usepackage{microtype}
\usepackage{geometry}
\geometry{margin=1in}
\usepackage{amsmath,amssymb,mathtools}
\usepackage[colorlinks=true,linkcolor=blue,citecolor=blue,urlcolor=blue]{hyperref}
\usepackage{setspace}
\usepackage{pdfpages} % for includepdf of supporting modules
\onehalfspacing

% Load Rosetta layer
% V_macros.tex — ultra-minimal Rosetta (Gemini-safe)
% Only simple \providecommand definitions. No packages, no \makeatletter.

% --- Core symbols used across G1S/G2S/G3S/P/V_article ---
\providecommand{\U}{\mathbf{U}}         % Admissible history / universe field
\providecommand{\V}{\mathbf{V}}         % Test / variation (generic handle)
\providecommand{\ph}{\varphi}           % Test function (lowercase alias)
\providecommand{\Test}{\mathcal{V}}     % Test space
\providecommand{\Bform}{\mathsf{B}}     % Bilinear/reciprocity pairing (weak form)
\providecommand{\grad}{\nabla}          % Gradient / abstract differential operator
\providecommand{\C}{\mathcal{C}}        % Causal class / partition carrier

% --- Entropy & closures (title-safe via \ensuremath) ---
\providecommand{\EntropySymbol}{\mathrm{S}}
% \Entropy[sub] -> S_sub (subscript optional), safe in text/headings
\providecommand{\Entropy}[1][]{\ensuremath{\EntropySymbol_{\!#1}}}
% Fourth derivative shorthand (1D strong closure)
\providecommand{\DFour}{\U^{(4)}}

% --- Named statement handles (contracts), safe in text/headings ---
\providecommand{\GOneStatement}{\ensuremath{\Delta\,\Entropy \ge 0}}
\providecommand{\GTwoStatement}{\ensuremath{\DFour = 0}}
\providecommand{\GTwoWeakForm}{\ensuremath{\Bform(\U,\ph)=0\ \forall\,\ph\in\Test}}
\providecommand{\GThreeStatement}{\ensuremath{\text{Discrete–continuum reciprocity via }\Bform}}

% --- Noether bridge (index-free API primitives) ---
\providecommand{\Xi}{\boldsymbol{\xi}}                                      % symmetry generator (abstract)
\providecommand{\NoetherTensor}[3]{\mathsf{N}\!\left[#1,#2;#3\right]}       % N[L,U;Xi]
\providecommand{\StressEnergy}{\mathbf{T}}                                   % stress–energy (abstract)
\providecommand{\Current}{\mathbf{J}}                                        % Noether current (abstract)
\providecommand{\Div}[1]{\ensuremath{\grad\!\cdot\! #1}}                     % divergence via ∇· (uses \grad)


% Version tag
\newcommand{\VVersion}{v1.2}

\title{Interpreting the Causal Closure Chain (\VVersion)\\
\small kinematics \texorpdfstring{\StrongForm}{U^(4)=0}\,;\,
reciprocity \texorpdfstring{\WeakForm}{B(U,phi)=0}\,;\,
conservation \texorpdfstring{\ConcludeDivT}{div T = 0}\,;\,
thermo \texorpdfstring{\GOneStatement}{Delta S >= 0}}
\author{Interpretive Synthesis Module (V.tex Compliant)}
\date{\today}

\begin{document}
\maketitle

\begin{abstract}
This article explains the causal-closure arc using the V handle system:
\emph{kinematics} (\StrongForm) $\Rightarrow$ \emph{conservation} (\ConcludeDivT) $\Rightarrow$ \emph{thermodynamic monotonicity} (\GOneStatement).
It also bundles the validated G1S–G5S supporting modules in an appendix for one-file distribution.
\end{abstract}

\section{The Arc of Closure}
\[
  \StrongForm \quad \longrightarrow \quad \ConcludeDivT \quad \longrightarrow \quad \GOneStatement.
\]

\section{Pillars of Interpretation}

\subsection{Kinematics and Reciprocity (G2)}
\(\StrongForm\) follows from stationarity and is equivalent to the weak reciprocity statement \(\WeakForm\).

\paragraph{Stability from coercivity.}
\noindent\InterpCoercivityToStability

\paragraph{Physical meaning of the kernel.}
\noindent\InterpKernelSignificance

\subsection{Conservation (G4)}
Noether bridge (index-free):
\[
  \StressEnergy \leftarrow \NoetherTensor{L}{\U}{\Xi},
  \qquad
  \ConcludeDivT.
\]

\subsection{Thermodynamics (G1)}
Admissible refinement of \(\C\) yields
\[
  \GOneStatement.
\]
\noindent\InterpDPIThermo

\subsection{Statistical Bounds (G5)}
\[
  \GFourDPI
  \qquad\text{and}\qquad
  \GFourScoreZero.
\]
Residual structure is captured as constrained noise \(\Res\); invariants \(\Invariant{\Xi}\) summarize what persists.

\section{Computation (Index-Free, G3 Alignment)}
Discrete reciprocity mirrors the continuum:
\[
  \frac{\DeltaFourh \Uh}{\mesh^4} \to \DFour \quad (\mesh \to 0),\qquad \Bformh \approx \Bform.
\]
\noindent\InterpDiscToContStability

\paragraph{Newton rail (JFNK + MG).}
\[
  \Rfunc(\U):=\DFour,\quad \NewtonStep,\quad \MGPrecond.
\]

\section*{V-layer Discipline}
\begin{enumerate}
  \item Use handles (\StrongForm, \WeakForm, \ConcludeDivT, \GOneStatement); no indices/stencils in V.
  \item Titles/headings are \texorpdfstring-safe; avoid \texttt{\textbackslash left...\textbackslash right} unless paired.
  \item Prefer interpretation handles for meaning: \InterpCoercivityToStability, \InterpKernelSignificance, \InterpDPIThermo, \InterpDiscToContStability.
\end{enumerate}

% ===============================
% Appendix: Supporting Modules
% ===============================
\appendix
\section*{Appendix: Validation Modules (G1S–G5S)}
\addcontentsline{toc}{section}{Appendix: Validation Modules}

% --- Option A: include compiled PDFs (recommended) ---
\includepdf[pages=-,addtotoc={1,subsection,1,{G1S: Order Monotonicity},sec:g1s}]{G1S.pdf}
\includepdf[pages=-,addtotoc={1,subsection,1,{G2S: Kinematic Closure},sec:g2s}]{G2S.pdf}
\includepdf[pages=-,addtotoc={1,subsection,1,{G3S: Discrete–Continuum Reciprocity},sec:g3s}]{G3S.pdf}
\includepdf[pages=-,addtotoc={1,subsection,1,{G4S: Noether Bridge},sec:g4s}]{G4S.pdf}
\includepdf[pages=-,addtotoc={1,subsection,1,{G5S: Statistical Closure},sec:g5s}]{G5S.pdf}

% --- Option B: (fallback) include source .tex directly ---
% Comment OUT Option A above, then UNcomment the block below if you prefer source inclusion.
% \section*{G1S: Order Monotonicity}\addcontentsline{toc}{subsection}{G1S: Order Monotonicity}
% \documentclass[12pt]{article}
\usepackage{amsthm,amssymb,amsmath,setspace,mathtools}
\usepackage{hyperref} % for \texorpdfstring
\onehalfspacing

% --- Module Dependencies ---
\newtheorem{theorem}{Theorem}
\newtheorem{definition}{Definition}
\newtheorem{axiom}{Axiom}
\newtheorem{example}{Example}

% --- Minimal inline-cite shim (replace later with biblatex/natbib) ---
\newcommand{\pcite}[1]{\textsuperscript{[#1]}}

% --- Load Rosetta Layer (Contracts & Shared Symbols) ---
% V_macros.tex — ultra-minimal Rosetta (Gemini-safe)
% Only simple \providecommand definitions. No packages, no \makeatletter.

% --- Core symbols used across G1S/G2S/G3S/P/V_article ---
\providecommand{\U}{\mathbf{U}}         % Admissible history / universe field
\providecommand{\V}{\mathbf{V}}         % Test / variation (generic handle)
\providecommand{\ph}{\varphi}           % Test function (lowercase alias)
\providecommand{\Test}{\mathcal{V}}     % Test space
\providecommand{\Bform}{\mathsf{B}}     % Bilinear/reciprocity pairing (weak form)
\providecommand{\grad}{\nabla}          % Gradient / abstract differential operator
\providecommand{\C}{\mathcal{C}}        % Causal class / partition carrier

% --- Entropy & closures (title-safe via \ensuremath) ---
\providecommand{\EntropySymbol}{\mathrm{S}}
% \Entropy[sub] -> S_sub (subscript optional), safe in text/headings
\providecommand{\Entropy}[1][]{\ensuremath{\EntropySymbol_{\!#1}}}
% Fourth derivative shorthand (1D strong closure)
\providecommand{\DFour}{\U^{(4)}}

% --- Named statement handles (contracts), safe in text/headings ---
\providecommand{\GOneStatement}{\ensuremath{\Delta\,\Entropy \ge 0}}
\providecommand{\GTwoStatement}{\ensuremath{\DFour = 0}}
\providecommand{\GTwoWeakForm}{\ensuremath{\Bform(\U,\ph)=0\ \forall\,\ph\in\Test}}
\providecommand{\GThreeStatement}{\ensuremath{\text{Discrete–continuum reciprocity via }\Bform}}

% --- Noether bridge (index-free API primitives) ---
\providecommand{\Xi}{\boldsymbol{\xi}}                                      % symmetry generator (abstract)
\providecommand{\NoetherTensor}[3]{\mathsf{N}\!\left[#1,#2;#3\right]}       % N[L,U;Xi]
\providecommand{\StressEnergy}{\mathbf{T}}                                   % stress–energy (abstract)
\providecommand{\Current}{\mathbf{J}}                                        % Noether current (abstract)
\providecommand{\Div}[1]{\ensuremath{\grad\!\cdot\! #1}}                     % divergence via ∇· (uses \grad)


\title{\textbf{Module 1: Order Monotonicity and the Second Law}\\
\large Validating G1: Thermodynamic Closure \texorpdfstring{\GOneStatement}{Delta S >= 0}}
\author{G1S — Module Contract Fulfillment (V.tex Compliant)}
\date{\today}

\begin{document}
\maketitle

\section{Axiomatic Foundation: Causal Order and Entropy}

The proof begins by grounding entropy and distinction within the causal classes \(\C\) of admissible events\pcite{31,32}.

\begin{definition}[Causal Entropy \(\Entropy\)]
The entropy associated with a causal set \(\C\) is defined via the number of admissible micro-orderings \(\Omega(\C)\):
\[
  \Entropy[\C] = k_{\mathrm{B}} \ln |\Omega(\C)|.
\]
Operationally, \(\Entropy\) quantifies the number of distinct internal configurations consistent with the underlying field \(\U\)\pcite{808,809}.
\end{definition}

\section{Theorem: The Second Law of Causal Order (\GOneStatement)}

\begin{theorem}[Monotonicity of Causal Entropy]
In any extension of a finite causal order that remains globally consistent, the count of distinguishable states cannot decrease\pcite{1259}.
\[
  \GOneStatement.
\]
\end{theorem}

\bigskip\hrule\bigskip

\begin{proof}[\textbf{G1 Proof Obligation Fulfillment}]
\begin{enumerate}
  \item \textbf{Causal refinement:} Let \(\C_n\) be a causal set and \(\C_{n+1}\) an admissible extension (refinement). By the Axiom of Event Selection (Martin-like consistency), the refinement preserves the partial order\pcite{29}.
  \item \textbf{Monotonicity of \(\Omega\):} Every micro-ordering admissible in \(\C_n\) remains admissible in \(\C_{n+1}\). Therefore, the set of admissible orderings cannot shrink:
  \[
    \Omega(\C_n) \subseteq \Omega(\C_{n+1}) \, .
  \]\pcite{1233}
  \item \textbf{Closure:} Taking logarithms and using the definition of \(\Entropy\) yields
  \[
    \Entropy[\C_{n+1}] - \Entropy[\C_n] \;=\; \Delta \Entropy \;\ge\; 0 \, .
  \]\pcite{1234}
\end{enumerate}
Thus \(\Entropy\) is nondecreasing under admissible refinement, establishing the thermodynamic closure\pcite{28,59}.
\end{proof}

\bigskip\hrule\bigskip

\section{Interpretation: Maxwell's Demon and Non-Commutativity}

The \GOneStatement{} is maintained even in the presence of an information-processing agent, providing an operational link to generalized information theory.

\begin{example}[Maxwell's Demon as Non-Commutative Selection]
The Demon's attempt to reduce entropy fails because the act of measurement (\(M\)) and the resulting system evolution (\(U\)) do not commute. The necessary entropy production comes directly from the act of distinction itself:
\[
  \text{Entropy production from non-commuting measurement/evolution } [M,U]\neq 0 \, .
\]\pcite{1240,1244}
The Demon exemplifies the theorem that informational refinement in one domain must be offset globally, ensuring the total causal order remains consistent\pcite{1245}.
\end{example}

\end{document}
% \section*{G2S: Kinematic Closure}\addcontentsline{toc}{subsection}{G2S: Kinematic Closure}
% \documentclass[12pt]{article}
\usepackage{amsthm,amssymb,amsmath,mathtools,setspace}
\onehalfspacing

% --- Module Dependencies (Must align with main P.tex definitions) ---
\newtheorem{theorem}{Theorem}
\newtheorem{definition}{Definition}
\newtheorem{proposition}{Proposition} % Added for intermediate proof steps
\newcommand{\U}{\mathbf{U}}        % Universe tensor (Field / Admissible History)
\newcommand{\V}{\mathbf{V}}        % Test Function / Variation
\newcommand{\grad}{\nabla}
\newcommand{\DeltaOp}{\Delta}      % Discrete difference operator

% V.tex Variational Symbols
\newcommand{\Acal}{\mathcal{A}}     % Action-like functional (\mathcal{R} in old G2S)
\newcommand{\Bform}{\mathsf{B}}     % Bilinear form (inner product of second derivatives)
\newcommand{\Hspace}{\mathcal{H}}   % Trial Space (H^2 space of admissible histories)
\newcommand{\Test}{\mathcal{V}}     % Test Space (Compactly supported variations)
\newcommand{\deltaVar}{\delta}      % First variation symbol
\newcommand{\Res}{\mathsf{R}}       % Residual (or source, here = 0)
\newcommand{\mesh}{\mathsf{h}}      % Mesh scale (discrete)
\newcommand{\Spline}{\mathsf{S}}    % Spline/interpolation operator

% --- Gold Check Harness (Copied from V.tex) ---
\newcounter{goldcheck}
\newenvironment{goldcheck}[1][]{%
  \refstepcounter{goldcheck}%
  \par\medskip\noindent\textbf{Gold Check G\thegoldcheck. #1}\par\smallskip\noindent
  \begin{enumerate}[leftmargin=2.2em,label=\arabic*.]
}{%
  \end{enumerate}\medskip
}
\newcommand{\gpass}{\textbf{[PASS]}}
\newcommand{\gfail}{\textbf{[FAIL]}}
\newenvironment{GTwoSpec}{\begin{goldcheck}[G2 Variational Reciprocity]}{\end{goldcheck}}

% --- Proof Obligation environment (Copied from V.tex) ---
\newenvironment{varobligation}[1][]{%
  \par\medskip\noindent\textbf{G2 Proof Obligation.} #1\;\emph{(Provide stationarity, reciprocity, and refinement consistency.)}%
  \begin{enumerate}[leftmargin=2.2em,label=\alph*.]
}{%
  \end{enumerate}\medskip
}

\title{\textbf{Module 2: Kinematic Closure and Splines} \\
\large Validating G2 and G3: The Minimal Curvature Condition}
\author{G2S — Module Contract Fulfillment (V.tex Compliant)}
\date{\today}

\begin{document}
\maketitle

\section{Input Node: Reciprocity and Variational Setup}

This module formalizes the unique analytic closure that minimally interpolates the discrete events guaranteed by consistency. Following the contract (V.tex Defs. 7--9), the search for the unique smooth field $\U(x)$ is cast as a minimization problem on the space of admissible histories $\Hspace$.

\begin{definition}[Informational Curvature Action $\Acal$]
Let $\U \in \Hspace \subset C^2(\mathbb{R})$ be an admissible history (Universe Tensor). The minimization of "bending energy" is defined by the quadratic action functional:
\[
\Acal[\U] \equiv \mathcal{R}[\U] = \frac{1}{2} \int (\DeltaOp^{(2)} \U)^2\, dx.
\]
The $\mathbf{Minimal\; Curvature\; Condition}$ is found by finding the stationary point of this action.
\end{definition}

\begin{definition}[Bilinear Form of Reciprocity]
The symmetric bilinear form $\Bform: \Hspace \times \Test \to \mathbb{R}$ associated with the action $\Acal$ is:
\[
\Bform(\U, \V) = \int (\DeltaOp^{(2)} \U) (\DeltaOp^{(2)} \V)\, dx.
\]
\end{definition}

\section{Theorem: Kinematic Closure (Validating G2 and G3)}

The condition of minimal informational friction (stationarity of $\Acal$) is mathematically identical to solving the cubic spline problem, ensuring unique analytic consistency.
\begin{theorem}[Kinematic Closure $\mathbf{U}^{(4)} = 0$]
The unique minimal-curvature solution $\U^\star$ compatible with Event Selection is the stationary point of $\Acal[\U]$, which yields the Euler-Lagrange equation in the continuum limit:
\[
\frac{\delta \Acal}{\delta \U} = \DeltaOp^{(4)} \U = 0.
\]
This condition is the necessary and sufficient condition for Kinematic Closure (G2).
\end{theorem}

\begin{proof}[\textbf{G2 Proof Obligation Fulfillment}]
The minimization procedure must satisfy the four core requirements set forth in V.tex.


\begin{varobligation}
  \item \textbf{Stationarity (Euler-Lagrange):} The first variation of the action $\Acal$ at $\U$ with respect to a test function $\V \in \Test$ (V.tex Def. 8) is:
    \[
    \deltaVar\Acal[\U](\V) = \left.\frac{d}{d\epsilon}\Acal[\U+\epsilon \V]\right|_{\epsilon=0} = \int (\DeltaOp^{(2)} \U) (\DeltaOp^{(2)} \V) dx.
    \]
    Integrating by parts twice (assuming appropriate boundary conditions $\bc$ on $\V$) yields:
    \[
    \deltaVar\Acal[\U](\V) = \int (\DeltaOp^{(4)} \U) \V dx.
    \]
    Setting $\deltaVar\Acal[\U](\V)=0$ for all $\V \in \Test$ implies the strong form $\DeltaOp^{(4)} \U = 0$.
  \item \textbf{Reciprocity (Bilinear Form Properties):} The required form $\Bform(\U, \V) = \int (\DeltaOp^{(2)} \U) (\DeltaOp^{(2)} \V) dx$ is inherently symmetric: $\Bform(\U, \V) = \Bform(\V, \U)$. It is also coercive, $\Bform(\U, \U) \ge \alpha \|\U\|^2_{\Hspace}$, for an appropriate Sobolev space $\Hspace$ and boundary conditions $\bc$ (e.g., clamped or natural splines) that prevent $\U''$ from vanishing non-trivially.
  \item \textbf{Weak Form:} The problem is written as finding $\U \in \Hspace$ such that $\Bform(\U, \V) = \Res(\V)$ for all $\V \in \Test$. Since there are no external sources or constraints (other than the fixed knots/boundaries), the residual term vanishes, $\Res(\V)=0$. The problem reduces to:
    \[
    \text{Find } \U \in \Hspace \text{ such that } \Bform(\U, \V) = 0 \quad \text{for all } \V \in \Test.
    \]
    The existence and uniqueness of $\U^\star$ are guaranteed by the Lax-Milgram theorem, c.f. V.tex Lemma 2.
  \item \textbf{Refinement Consistency (Spline):} The cubic spline interpolant $\Spline_\mesh \U_\mesh$ is the classical result of this minimization. Compliance with V.tex Def. 10 is asserted: the discrete solution $\U_\mesh$ converges to the unique continuous stationary solution $\U^\star$ (i.e., $\|\Spline_{\mesh} \U_{\mesh} - \U^\star\|\to 0$ as $\mesh\to 0$), validating the continuum limit of the discrete informational record.
\end{varobligation}
\end{proof}

\section{Output Node: Conclusion and Gold Check}

The established kinematic closure $\mathbf{U}^{(4)} = 0$ is the necessary structural input for subsequent Noether derivations (Chapter 4), defining the invariant under which translational symmetry is proven.

\begin{GTwoSpec}
  \item \gpass\; \textbf{Spaces}: $\Hspace \subset C^2(\mathbb{R})$ and $\Test \subset C^\infty_c$ defined; $\bc$ (fixed interpolation nodes and boundary values/slopes) is specified.
  \item \gpass\; \textbf{Action}: $\Acal[\U] = \frac{1}{2}\Bform(\U, \U)$ is quadratic (Gateaux-differentiable); first variation $\deltaVar\Acal[\U](\V)$ calculated, yielding $\DeltaOp^{(4)}\U=0$.
  \item \gpass\; \textbf{Reciprocity}: $\Bform(\U, \V) = \int \U'' \V'' dx$ is symmetric and certified coercive on the constrained space $\Hspace$.
  \item \gpass\; \textbf{Weak Form}: $\Bform(\U, \V)=\Res(\V)$ is stated with $\Res(\V)=0$, confirming the null source term.
  \item \gpass\; \textbf{Spline Consistency}: $\Spline_\mesh$ is the cubic spline operator; convergence assumptions verified by the classical solution.
  \item \gpass\; \textbf{Claim}: $\mathbf{U}^{(4)} = 0$ results from stationarity, and V.tex Theorem 3 guarantees existence/uniqueness.
  \item \gpass\; \textbf{Edge Cases}: Symmetry and coercivity rule out failure cases (V.tex Remark 4).
\end{GTwoSpec}

\begin{center}
\textit{Conclusion for Module 2: Smooth motion is the unique analytic form of consistency, established via a reciprocal variational principle.}
\end{center}

\end{document}
% \section*{G3S: Discrete–Continuum Reciprocity}\addcontentsline{toc}{subsection}{G3S: Discrete–Continuum Reciprocity}
% \documentclass[12pt]{article}
\usepackage{amsthm,amssymb,amsmath,mathtools,setspace}
\onehalfspacing

% --- Module Dependencies ---
\newtheorem{theorem}{Theorem}
\newtheorem{definition}{Definition}

% --- Load Rosetta Layer ---
% V_macros.tex — ultra-minimal Rosetta (Gemini-safe)
% Only simple \providecommand definitions. No packages, no \makeatletter.

% --- Core symbols used across G1S/G2S/G3S/P/V_article ---
\providecommand{\U}{\mathbf{U}}         % Admissible history / universe field
\providecommand{\V}{\mathbf{V}}         % Test / variation (generic handle)
\providecommand{\ph}{\varphi}           % Test function (lowercase alias)
\providecommand{\Test}{\mathcal{V}}     % Test space
\providecommand{\Bform}{\mathsf{B}}     % Bilinear/reciprocity pairing (weak form)
\providecommand{\grad}{\nabla}          % Gradient / abstract differential operator
\providecommand{\C}{\mathcal{C}}        % Causal class / partition carrier

% --- Entropy & closures (title-safe via \ensuremath) ---
\providecommand{\EntropySymbol}{\mathrm{S}}
% \Entropy[sub] -> S_sub (subscript optional), safe in text/headings
\providecommand{\Entropy}[1][]{\ensuremath{\EntropySymbol_{\!#1}}}
% Fourth derivative shorthand (1D strong closure)
\providecommand{\DFour}{\U^{(4)}}

% --- Named statement handles (contracts), safe in text/headings ---
\providecommand{\GOneStatement}{\ensuremath{\Delta\,\Entropy \ge 0}}
\providecommand{\GTwoStatement}{\ensuremath{\DFour = 0}}
\providecommand{\GTwoWeakForm}{\ensuremath{\Bform(\U,\ph)=0\ \forall\,\ph\in\Test}}
\providecommand{\GThreeStatement}{\ensuremath{\text{Discrete–continuum reciprocity via }\Bform}}

% --- Noether bridge (index-free API primitives) ---
\providecommand{\Xi}{\boldsymbol{\xi}}                                      % symmetry generator (abstract)
\providecommand{\NoetherTensor}[3]{\mathsf{N}\!\left[#1,#2;#3\right]}       % N[L,U;Xi]
\providecommand{\StressEnergy}{\mathbf{T}}                                   % stress–energy (abstract)
\providecommand{\Current}{\mathbf{J}}                                        % Noether current (abstract)
\providecommand{\Div}[1]{\ensuremath{\grad\!\cdot\! #1}}                     % divergence via ∇· (uses \grad)


% Local Macro Definitions (if V_macros didn't load them)
\providecommand{\DFour}{\U^{(4)}} 
\providecommand{\Bform}{\mathsf{B}} 
\providecommand{\Acal}{\mathcal{A}}
\providecommand{\Uh}{\mathbf{U}_{\mathbf{h}}} 
\providecommand{\mesh}{\mathsf{h}}      
\providecommand{\DeltaFourh}{\Delta_{\!\mathbf{h}}^{(4)}} 
\providecommand{\norm}[1]{\left\lVert #1 \right\rVert} 

\title{\textbf{Module 3: Discrete Kinematics and Consistency} \\
\large Validating G3: Discrete–Continuum Reciprocity \texorpdfstring{\GThreeStatement}{}}
\author{G3S — Module Contract Fulfillment (V.tex Compliant)}
\date{\today}

\begin{document}
\maketitle

\section{Input Node: The Discrete Energy Functional}
The continuum action $\Acal[\U]$ is the smooth limit of the discrete action $\Acalh[\Uh]$, whose stationarity yields the discrete Euler-Lagrange equation $\DeltaFourh \Uh = 0$.

\section{Theorem: The Kinematic Closure Chain (G3.Chain)}

\begin{theorem}[Discrete Kinematic Convergence]
The discrete Euler-Lagrange operator converges to the continuous solution:
\[
\DeltaFourh \Uh = 0 \quad \implies \quad \lim_{\mesh\to 0} \frac{\DeltaFourh \Uh}{\mesh^2} = \DFour.
\]
\end{theorem}

\section*{G3: Discrete–Continuum Reciprocity}

\paragraph{Summary.}
The validation of this module confirms the self-consistent closure chain by asserting that the discrete scheme accurately reflects the continuum system:
$$\GThreeStatement.$$

\paragraph{Compatibility with conservation (index-free).}
The validity of the discrete scheme hinges on its use of a pairing ($\Bform(\cdot,\cdot)$ or $\Bformh(\cdot,\cdot)$) that ensures **reciprocity** holds for the field $\U$.
This reciprocity guarantees that the conservation law derived from Noether's theorem is preserved across the numerical limit. Specifically, the abstract conservation current $\Current$ must satisfy:
\[
\Div{\Current} = 0,
\]
and this identity is invariant under the continuous interpolation and discrete sampling defined by the core reciprocity pairing. The consistent convergence of the field $\U$ means that the continuity laws for $\StressEnergy$ are preserved.

\paragraph{Comment.}
The convergence proof ensures that the flow of distinguishability—the source of $\StressEnergy$ and $\Current$—is numerically stable and topologically invariant as the mesh $\mesh \to 0$. This sets the necessary foundation for subsequent derivations of metric structure and field dynamics.

\end{document}
% \section*{G4S: Noether Bridge}\addcontentsline{toc}{subsection}{G4S: Noether Bridge}
% \documentclass[12pt]{article}
\usepackage{amsthm,amssymb,amsmath,mathtools,setspace}
\usepackage{hyperref} % for \texorpdfstring
\onehalfspacing

% --- Module Dependencies ---
\newtheorem{theorem}{Theorem}
\newtheorem{definition}{Definition}
\newtheorem{proposition}{Proposition}

% --- Load Rosetta Layer (Contracts & Shared Symbols) ---
% V_macros.tex — ultra-minimal Rosetta (Gemini-safe)
% Only simple \providecommand definitions. No packages, no \makeatletter.

% --- Core symbols used across G1S/G2S/G3S/P/V_article ---
\providecommand{\U}{\mathbf{U}}         % Admissible history / universe field
\providecommand{\V}{\mathbf{V}}         % Test / variation (generic handle)
\providecommand{\ph}{\varphi}           % Test function (lowercase alias)
\providecommand{\Test}{\mathcal{V}}     % Test space
\providecommand{\Bform}{\mathsf{B}}     % Bilinear/reciprocity pairing (weak form)
\providecommand{\grad}{\nabla}          % Gradient / abstract differential operator
\providecommand{\C}{\mathcal{C}}        % Causal class / partition carrier

% --- Entropy & closures (title-safe via \ensuremath) ---
\providecommand{\EntropySymbol}{\mathrm{S}}
% \Entropy[sub] -> S_sub (subscript optional), safe in text/headings
\providecommand{\Entropy}[1][]{\ensuremath{\EntropySymbol_{\!#1}}}
% Fourth derivative shorthand (1D strong closure)
\providecommand{\DFour}{\U^{(4)}}

% --- Named statement handles (contracts), safe in text/headings ---
\providecommand{\GOneStatement}{\ensuremath{\Delta\,\Entropy \ge 0}}
\providecommand{\GTwoStatement}{\ensuremath{\DFour = 0}}
\providecommand{\GTwoWeakForm}{\ensuremath{\Bform(\U,\ph)=0\ \forall\,\ph\in\Test}}
\providecommand{\GThreeStatement}{\ensuremath{\text{Discrete–continuum reciprocity via }\Bform}}

% --- Noether bridge (index-free API primitives) ---
\providecommand{\Xi}{\boldsymbol{\xi}}                                      % symmetry generator (abstract)
\providecommand{\NoetherTensor}[3]{\mathsf{N}\!\left[#1,#2;#3\right]}       % N[L,U;Xi]
\providecommand{\StressEnergy}{\mathbf{T}}                                   % stress–energy (abstract)
\providecommand{\Current}{\mathbf{J}}                                        % Noether current (abstract)
\providecommand{\Div}[1]{\ensuremath{\grad\!\cdot\! #1}}                     % divergence via ∇· (uses \grad)


% Local symbols defined via V_macros for use in the body
\providecommand{\Lcal}{\mathcal{L}}
\providecommand{\Scal}{\mathcal{S}}
\providecommand{\Xi}{\boldsymbol{\xi}}

\title{\textbf{Module 4: The Noether Bridge} \\
\large Validating G4: Derivation of Conserved Currents \texorpdfstring{\Div{\StressEnergy}=0}{div T = 0}}
\author{G4S — Proof of Conservation (V.tex Compliant)}
\date{\today}

\begin{document}
\maketitle

\bigskip\hrule\bigskip

\section{Input Node: Stationarity from Kinematic Closure (G2)}
The conservation law relies on the \emph{action principle} being stationary. The kinematic closure \(\GTwoStatement\) guarantees that the fundamental field \(\U\) is a solution to the Euler–Lagrange equations in the sense needed to invoke stationarity \(\delta \Scal = 0\).

\bigskip\hrule\bigskip

\begin{definition}[Action Functional \(\Scal\)]
The action functional is the integral of the local Lagrangian density \(\Lcal\) that encapsulates informational curvature:
\[
  \Scal[\U] \;=\; \int \Lcal\!\big(\U,\, \grad \U\big)\, d\tau ,
\]
where the measure \(d\tau\) is left abstract (index-free).
\end{definition}

\bigskip\hrule\bigskip

\section{Theorem: Conservation Law (Index-Free Noether)}

\begin{theorem}[Noether Conservation]
For every continuous symmetry \(\Xi\) that leaves the action invariant \((\delta \Scal = 0)\), there exists an associated current \(\Current\) whose divergence vanishes:
\[
  \Div{\Current} \;=\; 0.
\]
\end{theorem}

\bigskip\hrule\bigskip

\begin{proof}[\textbf{G4 Proof Obligation Fulfillment (Translational Case)}]
\begin{enumerate}
  \item \textbf{Hypothesis (stationarity).} The closure \(\GTwoStatement\) positions \(\U\) at a stationary point: \(\delta \Scal = 0\).
  \item \textbf{Symmetry selection.} We select the translational symmetry \(\Xi_{\mathrm{trans}}\).
  \item \textbf{Noether bridge (API).} Invoke the abstract construction
  \[
    \StressEnergy \;\leftarrow\; \NoetherTensor{\Lcal}{\U}{\Xi_{\mathrm{trans}}}.
  \]
  \item \textbf{Conservation identity.} Symmetry implies the resulting current is conserved, establishing the global bookkeeping constraint:
  \[
    \Div{\StressEnergy} \;=\; 0.
  \]
\end{enumerate}
Thus the Noether bridge carries the geometric constraint \(\GTwoStatement\) into the required structural conservation statement.
\end{proof}

\bigskip\hrule\bigskip

\end{document}

% \section*{G5S: Statistical Closure}\addcontentsline{toc}{subsection}{G5S: Statistical Closure}
% \documentclass[12pt]{article}
\usepackage{amsthm,amssymb,amsmath,mathtools,setspace}
\usepackage{hyperref} 
\onehalfspacing

% --- Module Dependencies ---
\newtheorem{theorem}{Theorem}
\newtheorem{definition}{Definition}
\newtheorem{proposition}{Proposition}
% V_macros.tex — ultra-minimal Rosetta (Gemini-safe)
% Only simple \providecommand definitions. No packages, no \makeatletter.

% --- Core symbols used across G1S/G2S/G3S/P/V_article ---
\providecommand{\U}{\mathbf{U}}         % Admissible history / universe field
\providecommand{\V}{\mathbf{V}}         % Test / variation (generic handle)
\providecommand{\ph}{\varphi}           % Test function (lowercase alias)
\providecommand{\Test}{\mathcal{V}}     % Test space
\providecommand{\Bform}{\mathsf{B}}     % Bilinear/reciprocity pairing (weak form)
\providecommand{\grad}{\nabla}          % Gradient / abstract differential operator
\providecommand{\C}{\mathcal{C}}        % Causal class / partition carrier

% --- Entropy & closures (title-safe via \ensuremath) ---
\providecommand{\EntropySymbol}{\mathrm{S}}
% \Entropy[sub] -> S_sub (subscript optional), safe in text/headings
\providecommand{\Entropy}[1][]{\ensuremath{\EntropySymbol_{\!#1}}}
% Fourth derivative shorthand (1D strong closure)
\providecommand{\DFour}{\U^{(4)}}

% --- Named statement handles (contracts), safe in text/headings ---
\providecommand{\GOneStatement}{\ensuremath{\Delta\,\Entropy \ge 0}}
\providecommand{\GTwoStatement}{\ensuremath{\DFour = 0}}
\providecommand{\GTwoWeakForm}{\ensuremath{\Bform(\U,\ph)=0\ \forall\,\ph\in\Test}}
\providecommand{\GThreeStatement}{\ensuremath{\text{Discrete–continuum reciprocity via }\Bform}}

% --- Noether bridge (index-free API primitives) ---
\providecommand{\Xi}{\boldsymbol{\xi}}                                      % symmetry generator (abstract)
\providecommand{\NoetherTensor}[3]{\mathsf{N}\!\left[#1,#2;#3\right]}       % N[L,U;Xi]
\providecommand{\StressEnergy}{\mathbf{T}}                                   % stress–energy (abstract)
\providecommand{\Current}{\mathbf{J}}                                        % Noether current (abstract)
\providecommand{\Div}[1]{\ensuremath{\grad\!\cdot\! #1}}                     % divergence via ∇· (uses \grad)


\title{\textbf{Module 5: Statistical Closure}\\
\large Validating G5: Predictability Bounds \texorpdfstring{\GFourDPI}{DPI}}
\author{G5S — Statistical Closure (V.tex Compliant)}
\date{\today}

\begin{document}
\maketitle

\section{Inputs: Conservation and Order Monotonicity}

The statistical closure of the causal field relies on the global stability established by the preceding modules. The inputs are the $\mathbf{Conservation\; Law}$ (from G4S) and the $\mathbf{Thermodynamic\; Constraint}$ (from G1S). Both constraints must hold for the probability measures underlying the causal evolution:
\[
  \Div{\Current}=0 \qquad\text{and}\qquad \GOneStatement.
\]

\section{Contracts: Information-Theoretic Closures}

The statistical foundation of the model is formalized by two main contracts governing the behavior of inference under causal refinement.
\begin{itemize}
  \item Data processing / coarse-graining: \GFourDPI.
  \item Score unbiasedness at truth: \GFourScoreZero.
\end{itemize}

\section{Theorem: DPI under admissible coarse-graining}

\begin{theorem}[Causal Data Processing Inequality]
The divergence $\KL{P}{Q}$ does not increase under admissible maps consistent with $\C$.
\end{theorem}

\begin{proof}[Proof Sketch]
Any admissible refinement of the causal partition $\C$ corresponds to an information-preserving, non-injective map between the sets of micro-orderings. Since information cannot be created by mere rearrangement of distinguishable elements, the distinguishability between any two probability distributions $P$ and $Q$ of these elements can only decrease or remain constant under the map. This is a direct consequence of the injective count monotonicity established in $\GOneStatement$.
\end{proof}

\section{Theorem: Unbiased score and Fisher lower bound}

\begin{theorem}
Estimation of causal parameters must adhere to the unbiased score constraint:
\[
  \GFourScoreZero.
\]
\end{theorem}

Estimation accuracy is bounded via $\Fisher$; residual uncertainty $\Res$ remains. The structural stability imposed by the kinematic and Noether constraints ($\GTwoStatement$ and $\Div{\StressEnergy}=0$) ensures that parameter estimation efficiency is maximized, but is still fundamentally limited by the local informational curvature ($\Fisher$).

\section{Invariance and Stable Statistics}

The conservation laws established in G4S imply that conserved quantities must yield stable statistical estimates regardless of measurement location.

Let $\Invariant{\Xi}$ denote an index-free invariant statistic associated with $\Xi$. The conservation law guarantees the stability of this invariant:
\[
  \Div{\Current}=0 \;\Longrightarrow\; \text{stability of } \Invariant{\Xi}.
\]

\section{Predictability Bounds}

The combined closures define the limit of predictive power in the causal system.
\[
  \KL{P}{Q}\ \text{nonincreasing}, \quad \E[\Score]=0, \quad \text{residual } \Res \ \text{is constrained noise.}
\]
The thermodynamic constraint ($\GOneStatement$ / monotonicity) and the statistical constraint ($\GFourDPI$ / distinguishability cannot increase) together state that finer partitions cannot increase predictive power beyond the invariant structure defined by the conserved current. The remaining uncertainty is not random noise, but the structurally limited residual ($\Res$) dictated by the analytic closure $\GTwoStatement$.

\end{document}

\end{document}
