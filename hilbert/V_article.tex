\documentclass[12pt]{article}
\usepackage{geometry}
\usepackage{setspace}
\usepackage{amsthm,amssymb,amsmath,mathtools}
\usepackage{microtype}
\usepackage{hyperref}
\usepackage{enumitem}
\geometry{margin=1in}
\onehalfspacing

% Import the Rosetta layer (guards only; safe if re-imported elsewhere)
% V_macros.tex — ultra-minimal Rosetta (Gemini-safe)
% Only simple \providecommand definitions. No packages, no \makeatletter.

% --- Core symbols used across G1S/G2S/G3S/P/V_article ---
\providecommand{\U}{\mathbf{U}}         % Admissible history / universe field
\providecommand{\V}{\mathbf{V}}         % Test / variation (generic handle)
\providecommand{\ph}{\varphi}           % Test function (lowercase alias)
\providecommand{\Test}{\mathcal{V}}     % Test space
\providecommand{\Bform}{\mathsf{B}}     % Bilinear/reciprocity pairing (weak form)
\providecommand{\grad}{\nabla}          % Gradient / abstract differential operator
\providecommand{\C}{\mathcal{C}}        % Causal class / partition carrier

% --- Entropy & closures (title-safe via \ensuremath) ---
\providecommand{\EntropySymbol}{\mathrm{S}}
% \Entropy[sub] -> S_sub (subscript optional), safe in text/headings
\providecommand{\Entropy}[1][]{\ensuremath{\EntropySymbol_{\!#1}}}
% Fourth derivative shorthand (1D strong closure)
\providecommand{\DFour}{\U^{(4)}}

% --- Named statement handles (contracts), safe in text/headings ---
\providecommand{\GOneStatement}{\ensuremath{\Delta\,\Entropy \ge 0}}
\providecommand{\GTwoStatement}{\ensuremath{\DFour = 0}}
\providecommand{\GTwoWeakForm}{\ensuremath{\Bform(\U,\ph)=0\ \forall\,\ph\in\Test}}
\providecommand{\GThreeStatement}{\ensuremath{\text{Discrete–continuum reciprocity via }\Bform}}

% --- Noether bridge (index-free API primitives) ---
\providecommand{\Xi}{\boldsymbol{\xi}}                                      % symmetry generator (abstract)
\providecommand{\NoetherTensor}[3]{\mathsf{N}\!\left[#1,#2;#3\right]}       % N[L,U;Xi]
\providecommand{\StressEnergy}{\mathbf{T}}                                   % stress–energy (abstract)
\providecommand{\Current}{\mathbf{J}}                                        % Noether current (abstract)
\providecommand{\Div}[1]{\ensuremath{\grad\!\cdot\! #1}}                     % divergence via ∇· (uses \grad)


% Theorem environments for the article
\newtheorem{theorem}{Theorem}
\newtheorem{definition}{Definition}
\newtheorem{proposition}{Proposition}
\newtheorem{remark}{Remark}

\title{\textbf{V: Reader’s Guide and Rosetta Layer}\\
\large Causal Order, Reciprocity, and the Two Closures}
\author{(Verification Tensor)}
\date{\today}

\begin{document}
\maketitle

\begin{abstract}
\noindent
This document (\textbf{V}) serves as both a \emph{reader’s guide} and a \emph{Rosetta layer}.
It states the minimal contracts that make the monograph legible across files:
(1) the \emph{thermodynamic closure} (\(\GOneStatement\)),
(2) the \emph{kinematic closure} (\(\GTwoStatement\)), and
(3) \emph{reciprocity} as a weak-form bridge (\(\GTwoWeakForm\)) between discrete and continuous models.
The same macros compile harmlessly when \texttt{V\_macros.tex} is \verb|\input| by other chapters.
\end{abstract}

\section{Purpose and Scope}
\texttt{V\_macros.tex} provides a guarded symbol table and statement handles used by
\texttt{P\_clean.tex}, \texttt{P.tex}, and the proof sketches \texttt{G1S.tex}, \texttt{G2S.tex}, \texttt{G3S.tex}.
All commands are declared with \verb|\providecommand|, so no package or environment conflicts occur.

\section{Causal Primitives and Notation}
We write \(\U\) for the admissible history (or field) in a function space \(\Hspace\), and \(\V\) for test/variation in \(\Test\).
For readability we use shorthands \(\DTwo = \U''\) and \(\DFour = \U^{(4)}\).
Entropy is denoted \(\Entropy\) unless specialized elsewhere.
We also expose lightweight utilities: \(\ip{\cdot}{\cdot}\) and \(\norm{\cdot}\).

\paragraph{Contracts.} Three centralized statement handles avoid duplication:
\begin{equation*}
  \textbf{G1: }\GOneStatement,\qquad
  \textbf{G2: }\GTwoStatement,\qquad
  \textbf{G2 (weak): }\GTwoWeakForm,
\end{equation*}
and a reciprocity summary for discretizations:
\begin{equation*}
  \textbf{G3: }\GThreeStatement.
\end{equation*}

\section{The Two Closures}
\subsection{Thermodynamic Closure (G1)}
\textbf{Statement.} Along any admissible evolution of \(\U\), the entropy does not decrease:
\begin{equation}\label{eq:G1}
  \GOneStatement.
\end{equation}
\textbf{Interpretation.} Monotonic distinguishability: coarse-grainings and communications preserve or increase effective state count.

\subsection{Kinematic Closure (G2)}
\textbf{Strong form (minimal curvature).}
\begin{equation}\label{eq:G2-strong}
  \GTwoStatement.
\end{equation}
\textbf{Weak form via reciprocity.} For all \(\varphi\in\Test\),
\begin{equation}\label{eq:G2-weak}
  \GTwoWeakForm.
\end{equation}
Concrete realizations of \(\Bform\) (integral kernels, boundary terms) live in the chapters; \eqref{eq:G2-strong} follows from \eqref{eq:G2-weak} under standard regularity assumptions.

\section{Reciprocity and Discrete--Continuum Alignment (G3)}
Discretizations (e.g., finite elements, splines) inherit the weak form by Galerkin projection.
The contract \(\GThreeStatement\) emphasizes symmetric, consistent discrete bilinear forms \(\Bform_h\) approximating \(\Bform\).

\section{Crosswalk for Readers}
\begin{itemize}[leftmargin=2em]
  \item \textbf{\texttt{P\_clean.tex}}: structured exposition referencing \(\Acal\), \(\Bform\), and closures.
  \item \textbf{\texttt{G1S.tex}}: establishes \eqref{eq:G1}; use \verb|\GOneStatement| for consistency.
  \item \textbf{\texttt{G2S.tex}}: develops \eqref{eq:G2-strong}/\eqref{eq:G2-weak}; use \verb|\GTwoStatement| and \verb|\GTwoWeakForm|.
  \item \textbf{\texttt{G3S.tex}}: proves stability/consistency of discrete counterparts; use \verb|\GThreeStatement|.
  \item \textbf{\texttt{P.tex}}: production build; inherits the same notation via \verb|% V_macros.tex — ultra-minimal Rosetta (Gemini-safe)
% Only simple \providecommand definitions. No packages, no \makeatletter.

% --- Core symbols used across G1S/G2S/G3S/P/V_article ---
\providecommand{\U}{\mathbf{U}}         % Admissible history / universe field
\providecommand{\V}{\mathbf{V}}         % Test / variation (generic handle)
\providecommand{\ph}{\varphi}           % Test function (lowercase alias)
\providecommand{\Test}{\mathcal{V}}     % Test space
\providecommand{\Bform}{\mathsf{B}}     % Bilinear/reciprocity pairing (weak form)
\providecommand{\grad}{\nabla}          % Gradient / abstract differential operator
\providecommand{\C}{\mathcal{C}}        % Causal class / partition carrier

% --- Entropy & closures (title-safe via \ensuremath) ---
\providecommand{\EntropySymbol}{\mathrm{S}}
% \Entropy[sub] -> S_sub (subscript optional), safe in text/headings
\providecommand{\Entropy}[1][]{\ensuremath{\EntropySymbol_{\!#1}}}
% Fourth derivative shorthand (1D strong closure)
\providecommand{\DFour}{\U^{(4)}}

% --- Named statement handles (contracts), safe in text/headings ---
\providecommand{\GOneStatement}{\ensuremath{\Delta\,\Entropy \ge 0}}
\providecommand{\GTwoStatement}{\ensuremath{\DFour = 0}}
\providecommand{\GTwoWeakForm}{\ensuremath{\Bform(\U,\ph)=0\ \forall\,\ph\in\Test}}
\providecommand{\GThreeStatement}{\ensuremath{\text{Discrete–continuum reciprocity via }\Bform}}

% --- Noether bridge (index-free API primitives) ---
\providecommand{\Xi}{\boldsymbol{\xi}}                                      % symmetry generator (abstract)
\providecommand{\NoetherTensor}[3]{\mathsf{N}\!\left[#1,#2;#3\right]}       % N[L,U;Xi]
\providecommand{\StressEnergy}{\mathbf{T}}                                   % stress–energy (abstract)
\providecommand{\Current}{\mathbf{J}}                                        % Noether current (abstract)
\providecommand{\Div}[1]{\ensuremath{\grad\!\cdot\! #1}}                     % divergence via ∇· (uses \grad)
|.
\end{itemize}

\section{Implementation Notes}
\begin{enumerate}[leftmargin=2em]
  \item \texttt{V\_macros.tex} introduces no packages or environments.
  \item Legacy \verb|\Vt| aliases to \verb|\V|.
  \item \(\Acal\) and \(\Bform\) are names only; concrete forms belong in the chapters that prove them.
\end{enumerate}

\end{document}
