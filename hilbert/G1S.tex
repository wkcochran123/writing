\documentclass[12pt]{article}
\usepackage{amsthm,amssymb,amsmath,mathtools,setspace}
\onehalfspacing

% --- Environment Definitions ---
\newtheorem{theorem}{Theorem}
\newtheorem{definition}{Definition}
\newtheorem{axiom}{Axiom}
% FIX: Define the 'remark' environment for use in the body.
\theoremstyle{remark} 
\newtheorem{remark}{Remark}

% --- Module Dependencies (Must align with main P.tex definitions) ---
\newcommand{\U}{\mathbf{U}}        % Universe tensor
\newcommand{\grad}{\nabla}
\newcommand{\T}{\mathbf{T}}        % Stress tensor / Coherence Tensor
\newcommand{\C}{\mathcal{C}}       % Causal set
\newcommand{\OmegaSet}{\Omega}     % Set of admissible micro-orderings
\newcommand{\kB}{k_{\mathrm{B}}}   % Boltzmann constant

\title{\textbf{Module 3: The Second Law as a Theorem} \\
\large Proving $\Delta S \ge 0$ from Causal Consistency (Validating $\mathbf{G1}$)}
\author{Independent Logical Section: Validating G1}
\date{\today}

\begin{document}
\maketitle

\section{Input Node: The Coherence Invariant}

The preceding module (Chapter 4) established the **conservation law** as the consistency requirement for a smooth, kinematically closed system. This invariant expresses that the flow of distinguishability (coherence) is locally conserved:
\[
\grad_\mu \mathbf{T}^{\mu\nu} = 0.
\]
This **conservation of coherence** serves as the necessary pre-condition for proving the monotonicity of causal order. It guarantees that the system's ledger of events is closed under translation, ensuring that all subsequent statements of order are built upon a stable, physically admissible substrate.

\section{Entropy as the Count of Distinctions}

To bridge the conservation of coherence ($\grad_\mu \mathbf{T}^{\mu\nu} = 0$) to the monotonic growth of time, we must formally define **Entropy** ($S$) not as a thermodynamic postulate, but as a direct, combinatorial measure of order within the causal set $\C$.

\begin{definition}[Entropy (Count of Distinctions)]
Let $\C$ denote the causal set of distinguishable events accessible to an observer, and let $\OmegaSet(\C)$ be the set of all admissible micro–orderings (histories) of those events consistent with the **Reciprocity Law** and the **Kinematic Closure** ($\mathbf{U}^{(4)}=0$).
The \emph{entropy} associated with $\C$ is the logarithm of this count:
\[
S[\C] = \kB \ln |\OmegaSet(\C)|.
\]
Operationally, $S$ measures the total number of distinct internal causal configurations that yield the same observable macroscopic coherence $\mathbf{T}^{\mu\nu}$.
\end{definition}

\begin{remark}
Entropy in this framework is a measure of the system's capacity for unresolved order. It is the number of ways the discrete events could have happened while still appearing smooth and conserved to the macroscopic observer.
\end{remark}

\section{Theorem: Monotonicity of Causal Entropy (Validating $\mathbf{G1}$)}

The proof rests on the structural consequence of the **Axiom of Event Selection** ($\mathbf{G5}$), which ensures that any extension of the causal set remains logically consistent.

\begin{theorem}[Second Law of Causal Order: $\mathbf{G1}$]
For any sequence of **Martin–consistent** causal set refinements $\C_1 \subseteq \C_2 \subseteq \cdots$ that preserves the conservation of coherence ($\grad_\mu \mathbf{T}^{\mu\nu} = 0$), the associated entropies satisfy the monotonicity condition:
\[
\Delta S_n \equiv S[\C_{n+1}] - S[\C_n] \ge 0.
\]
\end{theorem}

\begin{proof}
Let $\C_{n} \to \C_{n+1}$ be an admissible refinement of the causal set.
By the **Axiom of Event Selection** (G5), the extension $\C_{n+1}$ is guaranteed to exist and must maintain consistency with all local causal constraints defined by $\C_n$.

A refinement means that the new set $\C_{n+1}$ contains all the events and causal relations of $\C_n$, plus at least one new distinguishable event or relation. Since $\C_{n+1}$ is an extension of $\C_n$, any micro-ordering $\omega_n \in \OmegaSet(\C_n)$ that was admissible under the rules of $\C_n$ must also be representable and admissible under the rules of the larger set $\C_{n+1}$. The new set $\C_{n+1}$ may, however, admit new orderings $\omega_{n+1} \notin \OmegaSet(\C_n)$ due to the exposure of previously entangled (indistinguishable) elements.

Therefore, the set of admissible orderings can never shrink:
\[
\OmegaSet(\C_{n}) \subseteq \OmegaSet(\C_{n+1}).
\]
This leads directly to the non-decreasing cardinality:
\[
|\OmegaSet(\C_{n+1})| \ge |\OmegaSet(\C_{n})|.
\]
Since the entropy function $S[\C] = \kB \ln |\OmegaSet(\C)|$ is strictly monotonic, taking the difference yields the theorem:
\[
\Delta S = \kB \ln \left( \frac{|\OmegaSet(\C_{n+1})|}{|\OmegaSet(\C_{n})|} \right) \ge 0.
\]
The inequality is strict ($\Delta S > 0$) whenever the refinement exposes at least one new distinguishable causal configuration. The equality ($\Delta S = 0$) holds only if the refinement was informationally redundant.
\end{proof}

\section{Logical Interpretation: The Arrow of Time}

The Second Law of Causal Order is thus a theorem of logical necessity, not a statistical tendency. The **Arrow of Time** is defined by the monotonic growth of $S$, which simply indexes the unique direction in which the universe can consistently accumulate its own record of distinctions. Time is the ordinal index of this irreversible increase in total distinguishable order.

\section{Output Node: The Fixed Point}

The proven monotonicity constraint $\mathbf{\Delta S \ge 0}$ (G1) establishes the global requirement for the universe's long-term evolution. This result is the necessary input for the final module (Chapter 6), where this monotonic growth is unified with the geometry and fields derived from all preceding consistency requirements, identifying the total structure as a coherent, thermodynamically fixed point.

\begin{center}
\textit{Conclusion for Module 3 (Chapter 5): Order implies dynamics, and dynamics imply monotonic order. The logical structure is thermodynamically irreversible.}
\end{center}

\end{document}
