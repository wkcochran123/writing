\documentclass[12pt]{article}
\usepackage{amsthm,amssymb,amsmath,mathtools,setspace}
\onehalfspacing

% --- Module Dependencies (Must align with main P.tex definitions) ---
\newtheorem{theorem}{Theorem}
\newtheorem{definition}{Definition}
\newcommand{\U}{\mathbf{U}}        % Universe tensor
\newcommand{\Ek}{\mathbf{E}_k}      % Event tensor
\newcommand{\grad}{\nabla}
\newcommand{\T}{\mathbf{T}}        % Stress tensor

\title{\textbf{Module 3: The Second Law as a Theorem} \\
\large Proving $\Delta S \ge 0$ from Causal Consistency}
\author{Independent Logical Section: Validating G1}
\date{\today}

\begin{document}
\maketitle

\section{Input Node: The Coherence Invariant}

The preceding module established the **conservation law** ($\grad_\mu \mathbf{T}^{\mu\nu} = 0$) as the consistency requirement for a smooth, kinematically closed system. This invariant serves as the necessary pre-condition for proving the monotonicity of causal order.

\section{Entropy as the Count of Distinctions}

To bridge the conservation of momentum (coherence) to the monotonicity of time, we must formally define entropy not as a thermodynamic postulate, but as a direct measure of order within the causal set $\mathcal{C}$.

\begin{definition}[Entropy]
Let $\mathcal{C}$ denote the causal set of distinguishable events accessible to an observer, and let $\Omega(\mathcal{C})$ be the set of all admissible micro–orderings of those events consistent with the Reciprocity Law.  
The \emph{entropy} associated with $\mathcal{C}$ is the logarithm of this count:
\[
S[\mathcal{C}] = k_{\mathrm{B}} \ln |\Omega(\mathcal{C})|.
\]
Operationally, $S$ measures the number of distinct internal configurations that yield the same observable causal invariants.
\end{definition}

\section{Theorem: Monotonicity of Causal Entropy (G1)}

This is the central assertion of the proof, derived from the axiomatic foundation and the resulting conservation structures.

\begin{theorem}[Second Law of Causal Order: $\mathbf{G1}$]
For any sequence of Martin–consistent causal set refinements $\mathcal{C}_1 \subseteq \mathcal{C}_2 \subseteq \cdots$ that preserves the conservation of coherence ($\grad_\mu \mathbf{T}^{\mu\nu} = 0$), the associated entropies satisfy:
\[
\Delta S_n \equiv S[\mathcal{C}_{n+1}] - S[\mathcal{C}_n] \ge 0.
\]
\end{theorem}

\begin{proof}[Proof Idea (Monotonic Refinement)]
Each causal refinement $\mathcal{C}_{n} \to \mathcal{C}_{n+1}$ is an enlargement of the observer's partition of distinguishable events. By the **Axiom of Event Selection** (Martin-like consistency), a new event must be either a refinement of a pre-existing block or the introduction of a new block. Since the refinement must preserve all previous causal orderings, the set of admissible micro-orderings can never decrease:
\[
\Omega(\mathcal{C}_{n}) \subseteq \Omega(\mathcal{C}_{n+1}).
\]
Taking the logarithm (the entropy function) preserves this monotonic relationship, guaranteeing $\Delta S \ge 0$. The **Arrow of Time** is thus a theorem of order preservation, not a statistical tendency.
\end{proof}

\section{Output Node: The Fixed Point}

The proof of $\Delta S \ge 0$ establishes the necessary **monotonicity constraint** for the entire universe tensor. This result feeds directly into the final module (Chapter 6), where this monotonic growth is unified with geometry and fields.

\begin{center}
\textit{Conclusion for Module 3: Order implies dynamics, and dynamics imply monotonic order.}
\end{center}

\end{document}
```
eof

## How to Glue `G1S.tex` to `P.tex`

The goal of modular work is to treat `G1S.tex` as a new, highly detailed **Chapter 5** of your main work, `P.tex`.

Here is the step-by-step process for integrating this module into `P.tex` while respecting the spline nodes:

### 1. Identify the Splice Point

In your main document, `P.tex`, you need to replace the original text for **Chapter 5** (The Second Law) with the external file.

The section to **remove** (or comment out) is the content of `\chapter{The Arrow of Time: The Second Law as a Theorem}`.

### 2. The Input Spline Node (The "Handshake")

The input node ensures $G1S.tex$ starts where the preceding module (Chapter 4, Conservation) finished.

* **Location in `P.tex`:** After the end of Chapter 4 (`\chapter{Conservation and Coherence}`).
* **Action:** Ensure the first section of `G1S.tex` (`\section{Input Node: The Coherence Invariant}`) clearly references the completion of the conservation proof ($\grad_\mu \mathbf{T}^{\mu\nu} = 0$). This acts as the required dependency check.

### 3. Inserting the Module (The `\input` Command)

Since `G1S.tex` uses the `article` class, you cannot simply copy and paste it into the `book` class (`P.tex`). The most robust way to insert it as a chapter is to strip it down to its core content and use the `\input` command or manually wrap it.

**Recommended Insertion Method:**

1.  In `P.tex`, find the start of Chapter 5:
    ```latex
    \chapter{The Arrow of Time: The Second Law as a Theorem}
    \label{chap:arrow-of-time}
    \section*{Core Idea: A Consistent Record Must Grow}
    ... (old content here) ...
    ```
2.  **Replace** the old content of Chapter 5 with a direct `\input` command:
    ```latex
    \chapter{The Arrow of Time: The Second Law as a Theorem}
    \label{chap:arrow-of-time}
    \section*{Core Idea: A Consistent Record Must Grow (Validating G1)}
    
    % --- Begin G1S.tex Content ---
    \input{G1S_core.tex}
    % --- End G1S.tex Content ---
    
    \chapter{The Proof Tensor and the Cosmic Ledger}
    ...
    
