\documentclass[12pt]{article}
\usepackage{amsthm,amssymb,amsmath,setspace,mathtools}
\onehalfspacing

% --- Module Dependencies (Must align with main P.tex definitions) ---
\newtheorem{theorem}{Theorem}
\newtheorem{definition}{Definition}
\newtheorem{axiom}{Axiom}

% --- Core Symbols (Needed if V_macros.tex is missing) ---
\newcommand{\kB}{k_{\mathrm{B}}}
\newcommand{\U}{\mathbf{U}}
\newcommand{\V}{\mathbf{V}}
\newcommand{\EntropySymbol}{\mathrm{S}}

% --- Optional Rosetta Layer Input ---
\IfFileExists{V_macros.tex}{% V_macros.tex — ultra-minimal Rosetta (Gemini-safe)
% Only simple \providecommand definitions. No packages, no \makeatletter.

% --- Core symbols used across G1S/G2S/G3S/P/V_article ---
\providecommand{\U}{\mathbf{U}}         % Admissible history / universe field
\providecommand{\V}{\mathbf{V}}         % Test / variation (generic handle)
\providecommand{\ph}{\varphi}           % Test function (lowercase alias)
\providecommand{\Test}{\mathcal{V}}     % Test space
\providecommand{\Bform}{\mathsf{B}}     % Bilinear/reciprocity pairing (weak form)
\providecommand{\grad}{\nabla}          % Gradient / abstract differential operator
\providecommand{\C}{\mathcal{C}}        % Causal class / partition carrier

% --- Entropy & closures (title-safe via \ensuremath) ---
\providecommand{\EntropySymbol}{\mathrm{S}}
% \Entropy[sub] -> S_sub (subscript optional), safe in text/headings
\providecommand{\Entropy}[1][]{\ensuremath{\EntropySymbol_{\!#1}}}
% Fourth derivative shorthand (1D strong closure)
\providecommand{\DFour}{\U^{(4)}}

% --- Named statement handles (contracts), safe in text/headings ---
\providecommand{\GOneStatement}{\ensuremath{\Delta\,\Entropy \ge 0}}
\providecommand{\GTwoStatement}{\ensuremath{\DFour = 0}}
\providecommand{\GTwoWeakForm}{\ensuremath{\Bform(\U,\ph)=0\ \forall\,\ph\in\Test}}
\providecommand{\GThreeStatement}{\ensuremath{\text{Discrete–continuum reciprocity via }\Bform}}

% --- Noether bridge (index-free API primitives) ---
\providecommand{\Xi}{\boldsymbol{\xi}}                                      % symmetry generator (abstract)
\providecommand{\NoetherTensor}[3]{\mathsf{N}\!\left[#1,#2;#3\right]}       % N[L,U;Xi]
\providecommand{\StressEnergy}{\mathbf{T}}                                   % stress–energy (abstract)
\providecommand{\Current}{\mathbf{J}}                                        % Noether current (abstract)
\providecommand{\Div}[1]{\ensuremath{\grad\!\cdot\! #1}}                     % divergence via ∇· (uses \grad)
}{}

\title{\textbf{Module 1: Order Monotonicity and The Second Law} \\
\large Validating G1: Thermodynamic Closure $\GOneStatement$}
\author{G1S — Module Contract Fulfillment (V.tex Compliant)}
\date{\today}

\begin{document}
\maketitle

\section{Axiomatic Foundation: Causal Order and Event Selection}

The proof begins by grounding the concepts of entropy and distinction within ZFC set theory, as outlined in \texttt{P.tex}[cite: 202, 211].

\begin{definition}[Causal Set and Admissible Micro-Orderings]
Let $\C$ be the causal set of distinguishable events accessible to an observer[cite: 881, 1004]. Let $\Omega(\C)$ be the set of all **admissible micro-orderings** of those events consistent with the Axiom of Event Selection (Martin-like consistency)[cite: 81, 881, 1328].
\end{definition}

\begin{definition}[Causal Entropy $\Entropy$]
The $\mathbf{Entropy}$ associated with a causal set $\C$ is the logarithm of this count, where $\kB$ is Boltzmann's constant:
\[
\Entropy[\C] = \kB \ln |\Omega(\C)|.
\]
Operationally, $\Entropy$ measures the number of distinct internal configurations that yield the same observable causal invariants[cite: 883].
\end{definition}

\section{Theorem: The Second Law of Causal Order $(\GOneStatement)$}

The principle that information, and thus entropy, can never decrease is proved as a theorem of causal consistency rather than assumed as a postulate[cite: 78, 79].

\begin{theorem}[Monotonicity of Causal Entropy $\mathbf{(\GOneStatement)}$]
In any extension of a finite causal order that remains globally consistent (via Event Selection/Martin-like consistency), the count of distinguishable states cannot decrease.
\[
\GOneStatement.
\]
\end{theorem}

\begin{proof}[\textbf{G1 Proof Obligation Fulfillment}]
\begin{enumerate}
    \item \textbf{Causal Refinement:} Let $\C_n$ be a causal set, and $\C_{n+1}$ be an admissible extension (refinement) that introduces new distinctions while remaining consistent with the Axiom of Event Selection[cite: 1306]. By definition, $\C_n \subseteq \C_{n+1}$.
    \item \textbf{Monotonicity of $\Omega$:} Since the refinement preserves the partial order and is globally consistent, every micro-ordering that was admissible in $\C_n$ must remain admissible in $\C_{n+1}$[cite: 92]. Therefore, the set of admissible orderings cannot shrink:
    $$
    \Omega(\C_n) \subseteq \Omega(\C_{n+1}).
    $$
    \item \textbf{Logarithmic Closure:} Taking the natural logarithm and multiplying by $\kB$:
    $$
    \kB \ln |\Omega(\C_n)| \le \kB \ln |\Omega(\C_{n+1})|.
    $$
    \item \textbf{Conclusion (Entropy Invariance):} This directly implies that the entropy is non-decreasing[cite: 1307, 1308]:
    $$
    \Entropy[\C_{n+1}] - \Entropy[\C_n] = \Delta \Entropy \ge 0.
    $$
\end{enumerate}
The equality holds only if the refinement $\C_{n+1}$ does not expose any previously indistinguishable configurations. Thus, the Arrow of Time is a theorem of order monotonicity[cite: 93].
\end{proof}

\end{document}