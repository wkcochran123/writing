\documentclass[12pt]{article}
\usepackage{amsthm,amssymb,amsmath,setspace,mathtools}
\usepackage{hyperref} % for \texorpdfstring
\onehalfspacing

% --- Module Dependencies ---
\newtheorem{theorem}{Theorem}
\newtheorem{definition}{Definition}
\newtheorem{axiom}{Axiom}
\newtheorem{example}{Example}

% --- Minimal inline-cite shim (replace later with biblatex/natbib) ---
\newcommand{\pcite}[1]{\textsuperscript{[#1]}}

% --- Load Rosetta Layer (Contracts & Shared Symbols) ---
% V_macros.tex — ultra-minimal Rosetta (Gemini-safe)
% Only simple \providecommand definitions. No packages, no \makeatletter.

% --- Core symbols used across G1S/G2S/G3S/P/V_article ---
\providecommand{\U}{\mathbf{U}}         % Admissible history / universe field
\providecommand{\V}{\mathbf{V}}         % Test / variation (generic handle)
\providecommand{\ph}{\varphi}           % Test function (lowercase alias)
\providecommand{\Test}{\mathcal{V}}     % Test space
\providecommand{\Bform}{\mathsf{B}}     % Bilinear/reciprocity pairing (weak form)
\providecommand{\grad}{\nabla}          % Gradient / abstract differential operator
\providecommand{\C}{\mathcal{C}}        % Causal class / partition carrier

% --- Entropy & closures (title-safe via \ensuremath) ---
\providecommand{\EntropySymbol}{\mathrm{S}}
% \Entropy[sub] -> S_sub (subscript optional), safe in text/headings
\providecommand{\Entropy}[1][]{\ensuremath{\EntropySymbol_{\!#1}}}
% Fourth derivative shorthand (1D strong closure)
\providecommand{\DFour}{\U^{(4)}}

% --- Named statement handles (contracts), safe in text/headings ---
\providecommand{\GOneStatement}{\ensuremath{\Delta\,\Entropy \ge 0}}
\providecommand{\GTwoStatement}{\ensuremath{\DFour = 0}}
\providecommand{\GTwoWeakForm}{\ensuremath{\Bform(\U,\ph)=0\ \forall\,\ph\in\Test}}
\providecommand{\GThreeStatement}{\ensuremath{\text{Discrete–continuum reciprocity via }\Bform}}

% --- Noether bridge (index-free API primitives) ---
\providecommand{\Xi}{\boldsymbol{\xi}}                                      % symmetry generator (abstract)
\providecommand{\NoetherTensor}[3]{\mathsf{N}\!\left[#1,#2;#3\right]}       % N[L,U;Xi]
\providecommand{\StressEnergy}{\mathbf{T}}                                   % stress–energy (abstract)
\providecommand{\Current}{\mathbf{J}}                                        % Noether current (abstract)
\providecommand{\Div}[1]{\ensuremath{\grad\!\cdot\! #1}}                     % divergence via ∇· (uses \grad)


\title{\textbf{Module 1: Order Monotonicity and the Second Law}\\
\large Validating G1: Thermodynamic Closure \texorpdfstring{\GOneStatement}{Delta S >= 0}}
\author{G1S — Module Contract Fulfillment (V.tex Compliant)}
\date{\today}

\begin{document}
\maketitle

\section{Axiomatic Foundation: Causal Order and Entropy}

The proof begins by grounding entropy and distinction within the causal classes \(\C\) of admissible events\pcite{31,32}.

\begin{definition}[Causal Entropy \(\Entropy\)]
The entropy associated with a causal set \(\C\) is defined via the number of admissible micro-orderings \(\Omega(\C)\):
\[
  \Entropy[\C] = k_{\mathrm{B}} \ln |\Omega(\C)|.
\]
Operationally, \(\Entropy\) quantifies the number of distinct internal configurations consistent with the underlying field \(\U\)\pcite{808,809}.
\end{definition}

\section{Theorem: The Second Law of Causal Order (\GOneStatement)}

\begin{theorem}[Monotonicity of Causal Entropy]
In any extension of a finite causal order that remains globally consistent, the count of distinguishable states cannot decrease\pcite{1259}.
\[
  \GOneStatement.
\]
\end{theorem}

\bigskip\hrule\bigskip

\begin{proof}[\textbf{G1 Proof Obligation Fulfillment}]
\begin{enumerate}
  \item \textbf{Causal refinement:} Let \(\C_n\) be a causal set and \(\C_{n+1}\) an admissible extension (refinement). By the Axiom of Event Selection (Martin-like consistency), the refinement preserves the partial order\pcite{29}.
  \item \textbf{Monotonicity of \(\Omega\):} Every micro-ordering admissible in \(\C_n\) remains admissible in \(\C_{n+1}\). Therefore, the set of admissible orderings cannot shrink:
  \[
    \Omega(\C_n) \subseteq \Omega(\C_{n+1}) \, .
  \]\pcite{1233}
  \item \textbf{Closure:} Taking logarithms and using the definition of \(\Entropy\) yields
  \[
    \Entropy[\C_{n+1}] - \Entropy[\C_n] \;=\; \Delta \Entropy \;\ge\; 0 \, .
  \]\pcite{1234}
\end{enumerate}
Thus \(\Entropy\) is nondecreasing under admissible refinement, establishing the thermodynamic closure\pcite{28,59}.
\end{proof}

\bigskip\hrule\bigskip

\section{Interpretation: Maxwell's Demon and Non-Commutativity}

The \GOneStatement{} is maintained even in the presence of an information-processing agent, providing an operational link to generalized information theory.

\begin{example}[Maxwell's Demon as Non-Commutative Selection]
The Demon's attempt to reduce entropy fails because the act of measurement (\(M\)) and the resulting system evolution (\(U\)) do not commute. The necessary entropy production comes directly from the act of distinction itself:
\[
  \text{Entropy production from non-commuting measurement/evolution } [M,U]\neq 0 \, .
\]\pcite{1240,1244}
The Demon exemplifies the theorem that informational refinement in one domain must be offset globally, ensuring the total causal order remains consistent\pcite{1245}.
\end{example}

\end{document}