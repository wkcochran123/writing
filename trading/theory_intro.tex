\chapter{The Three Trading Systems}

One simple way to classify trading systems is by the type of instrument they trade.
Equitites engines, futures engines, and currency/crypto trading are some examples of
types of trading engines.  But, defining instruments turns out to be a very difficult
problem and one reserved for another book.

Rather, we can look at the type of \emph{risk} the instrument being traded
represents.  There are three fundamental types of risk that most, if not all,
instruments fall into:
\begin{enumerate}
\item \emph{Absolute risk instruments} are instruments where it is \emph{not
possible} to lose more than the principal on the instrument.
\item \emph{Relative risk instruments} are instruments where it is \emph{possible}
to lose more than principle on the instrument.
\item \emph{Cross risk "instruments"} are not really instruments at all.  These
are crossrates that compute the value of one thing in terms of the other.  EURUSD,
for instance, is how many dollars it takes to buy 1 euro. There is no "EURUSD"
instrument. When I refer to instruments in this book, I inclusively include crossrates
and crypto trading, even though these are not instruments by any definition.
\end{enumerate}
At the heart of these definitions is the idea of risk. Throughout this book, I will
measure risk in dollars.  Value at Risk (VaR) or appropriate surrogate is the idea
being captured.  Some idea of the magnitude of the first derivative is another way
to conecptualize this. But, to measure risk, we need to understand another concept.

Holding period, or how long should the trading system expect before any particular
position is unwound, is the other concept necessary to describe the type of risk
being taken on.  Given a holding period and a variance, the total amount of
\emph{traded risk} can be modeled.

As the trading horizon decreases, the total amount of traded risk that can be
accummulated safely decreases. Having a highly volatile short term view is incosistent
with having a large position and expectation of profit.  As such, the quantitative
trading strategies fundamentally change.  The algorithms used to manage portfolios
get more and more streamlined. There becomes a very real tradeoff between performance
and flexibility.  For this reason, trading systems can be divided into three different
holding horizons based on focus.
\begin{enumerate}
\item \emph{Mid-long frequency} strategies routinely hold informed overnight positions.
By informed, I mean exactly that, the trade was designed to incur overnight volatilies.
Holding times are long enough to make ownership limit positions possible for long
periods of time while less directional trading can still playout intraday.
\item \emph{Low-latency} trading has evolved and added appropriate "ultra" and "super,"
but their technology has become robust enough and flexible enough that low-latency
trading engines are filtering into broader and broader trading strategies.  These
are mid-to-high frequency trading signals that allow for accumulation intraday and
possibility of overnight holding but rely on market making strategies to improve
market impact.
\item \emph{Ultra Low Latency} We are almost to the point where there is the minimal
entropy inside the switch that triggers the outgoing order.  The entire fight is
finding clever ways to cut the number of bits required of an incoming message to
trigger a profitable trade on the outbound.
\end{enumerate}

These are three very different kinds of trading systems--algorithmic trading at
a maximal rate related to traded risk, high speed trading using very narrowly
applicable technology, and the hybrid system that blends them.  This chapter will
explore risk and trading horizon in depth to understand how a trading system

\section{Risk}

Before we start with trading engines, let's start with trading.  Why do people trade?
The reasons are as diverse as the wisdom or folly behind them.  This book is not
concerned with \textit{why} people trade but with \textit{how} people trade. Or,
rather as the case is today, how computers trade.

For instance, consider a retail order placed on a app that enables day-trading.
That order is then presented to a wholesale market maker that will decide whether they
should fill from inventory or exhaust.  There can be several layers of this exhaust,
depending on the calibre of the retail trader and the liquidity of the name being
traded. The eventual final exhaust is, of course, the open market.  There can be
several trading systems tied together in a loose collaboration through FIX that
eventually all agree on how to fill the order.

Of course, trading can get much more sophisticated than this.  It would be very
difficult to give an exhaustive list of the potential responsibilities of a trading
system.  Not only that, but rules and regulations change fairly frequently around
the world.  Profitable trading strategies ebb and flow as macroeconomic conditions
gyrate. The trading system should fill orders at \emph{minimal market impact} without
impacting orderly trading while being able to seamlessly absorbed changes required
by the ever evolving landscape. Should a new marketplace appear desirable, it should be 
serving that marketplace within a matter of weeks to race opportunity decay.

This is the financial dimension of trading systems.  The trading system negotiates the
exchange of something of value for something else of value for the purposes of 
transferring some risk.  Exposure to profits and losses of a company, the point 
value of the S\&P 500 will increase or decrease, inflation risk, default risk,
etc.  The list 
goes on and on, but the fact remains, all trading incurs risk.  All trading systems 
manage risk.  The trading system facilitates the lifecycle transistions of risk.

As such, this book divides these trading systems into three different types of 
financial instrument: \emph{absolute risk instruments} such as equities, 
\emph{relative risk instruments} such as derivatives, and \emph{cross risk 
instruments} such as currency pairs.  

A trade can decide to take a risk.  The trader can calculate how much risk they would
like to take, find an instrument that represents that risk, compute the amount of the
instrument to purchase, and enter the trade. At this point, the trader owns rights
granted by the instrument.  But, the trade won't turn a profit or book a loss until..

\section{Horizon}

They flatten their position.  The length of time the trader plans to hold the position
before re-evaluation is the \emph{horizon}.  






\begin{lstlisting}
   while 
\end{lstlisting}

