

\chapter{Instruments}

\section{Instruments, Porfolios, and Risk}

Trading, the primary means of allowing wide spread division of labor, simultaneously privides
a service called \emph{price discovery}.  These activities are far too complex to try to
review here.  Rather, we care more about the mechanism by which this takes place.  At its heart
a trade is simply an exchange of two things by two people.  Let's dissect how hard this
actually is.

Suppose Alice and Bob would like to trade an apple for a banana.  They could simultaneously hand
each other the fruit.  Otherwise, without an intermediary, there will be an instant in time
where Alice or Bob will be in possession of both pieces of fruit.  If Alice and Bob were to
try to exchange something that didn't fit in each other's hand, then there must be a time
of ambiguous possession.  This ambiguous possession either involves Alice and Bob simultaneously
both or neither in possession of the fruit.  This instant of ambiguous possession begins with 
agreement trade and ends when each party has clear title.  What this means is trading large,
complicated things only gets harder and more complicated.

One of the most complicated things traded are financial instruments used to invest capital in
order to make commercial gains.  As you can imagine, the trading that happens here is
extraordinarily sophisticated due to how complex the trading needs to be as well as how
complex the financial instruments are.  This book goes into excruciating detail about how
complex trading can get. Understanding how to evaluate the value of financial instruments is
left as an exercise for the reader.

What is a financial instrument? Suppose you take a very large problem and divide up paying 
for the solution among a large number of people because the problem is that large.  Well, the 
people that \emph{invest} in
solving this problem are due the relative proportion of any commercial gain achieved by that
solution.  These little divisions of financing are called financial instruments.  At their
heart, there is a risk being undertaken that may provide commercial gain.  This risk is being
divided up and distributed to the investor at a rate commensurate to how much they invest
relative to the total value of the endeavor.

A financial \emph{instrument} is little more than the expression of some commercial or 
economic risk or other risk that can be measured in currency. The instrument indicates the
owner has particular rights and priveleges that have value.  The currency exchanged for the
instrument is a measure of that value.  Not the only measure, as value as a very relative
meaning.  What one person values highly, another may value cheaply.

Getting at the value of an instrument is rather challenging.  It is said that the primary
purpose of trade is to "discover the price."  This is generally considered to be the price
at which the most transactions would take place.  The philosophy and practice of trading
are left as an exercise for the reader.  Suffice it to say, understanding the value of
an instrument is at the heart of all trading: two people disagree on the value of an asset
owned by one.

The conviction to which the valuation applies, the time horizon on the term of ownership,
any number of other considerations also go into the decision.  Even the proverbial throwing
a dart a the wall can go into the consideration.  At the end of the day two things change
hands: one is the instrument that represents some risk, tangible or intangible, and 
\emph{currency} that measures a value.



\section{The Lifecycle of a Trade}

\begin{enumerate}
\item{Trigger} A need to trade has been identified, whatever that need maybe.  This need
is expressed as a desired portfolio of instruments.
\item{Execution}
\item{Exit}
\end{enumerate}
